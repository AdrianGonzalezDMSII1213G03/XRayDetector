%%%%%%%%%%%%%%%%%%%%%%%%%%%%%%%%%%%%%%%%%%%%%%%%%%%%%%%%%%%%%%%%%%
%%%%%%%%%%%%%%%%%%%%%%%%%%%%%%%%%%%%%%%%%%%%%%%%%%%%%%%%%%%%%%%%%%
\chapter{Aspectos relevantes del desarrollo del proyecto}
%%%%%%%%%%%%%%%%%%%%%%%%%%%%%%%%%%%%%%%%%%%%%%%%%%%%%%%%%%%%%%%%%%
%%%%%%%%%%%%%%%%%%%%%%%%%%%%%%%%%%%%%%%%%%%%%%%%%%%%%%%%%%%%%%%%%%

En este apartado se detallan los aspectos más relevantes que se han encontrado en el proceso de desarrollo del proyecto.

\section{Problemas y retos}
El tema principal del proyecto era totalmente desconocido. Para la obtención de los defectos en las radiografías se han utilizado técnicas novedosas que no se enseñan en ninguna asignatura de la carrera. Hay que añadir que algunas de esas técnicas se usan actualmente en proyectos de investigación en universidades y centros de investigación con mucha más experiencia en este campo que la Universidad de Burgos.

Se busca resolver un problema real, con imágenes (cedidas por el Grupo Antolín) obtenidas desde los puntos de vista y condiciones que obtiene una máquina real, piezas reales y con formas muy complejas. Es un problema mucho más complicado que el que se aborda en los artículos relacionados donde la imagen suele ser mucho mas uniforme y sencilla. En la imagen \ver{radiografia_articulos} se puede ver una comparativa de algunas de las imágenes usadas en otros artículos. En la figura \ver{radiografia_antolin} se puede ver un ejemplo de una de las imágenes que usamos en nuestro proyecto.

Por lo anterior, este ha sido un proyecto de gran incertidumbre y riesgo que ha hecho que:

\begin{enumerate}
\item Como se preveía que los requisitos del proyecto iban a cambiar mucho durante el desarrollo del mismo, ya que surgen nuevas ideas, nuevas aproximaciones para resolver el problema, se decidió usar la metodología Scrum, que está pensada para este tipo de entornos, en los que los requisitos cambian y es difícil hacer una planificación.

\item Haya sido necesario adquirir muchos nuevos conocimientos.
\end{enumerate}

\figura{1}{imgs/radiografia_articulos.png}{Radiografías usadas en otros artículos \cite{anand2009flaw} \cite{tridi} \cite{DomingoMery}}{radiografia_articulos}{}

\figura{1}{imgs/radiografia_antolin.png}{Ejemplo de radiografía usada en nuestro proyecto}{radiografia_antolin}{}


\section{Mejoras respecto a la versión previa}
Uno de los objetivos del proyecto era mejorar a la versión previa presentada el curso pasado. Esto supuso algunos problemas, ya que arreglar ciertos aspectos no ha sido tan sencillo.

Las mejoras introducidas han sido:

\begin{enumerate}

\item \textbf{Mejoras sobre el código:} se ha mejorado el código en general, procurando que sea más claro y evitando ciertos defectos de código, en la medida de lo posible. Se han refactorizado, por ejemplo, algunas clases en las que había métodos exageradamente largos, procurando dividirlos en métodos más pequeños, lo que mejora mucho la legibilidad.

\item \textbf{Mejoras en la estructura:} se ha cambiado completamente la estructura de la aplicación, introduciendo un diseño en tres capas y algunos patrones de diseño que permiten mejorar el mantenimiento y la extensibilidad de la aplicación.

\item \textbf{Mejoras de rendimiento:} se ha intentado mejorar el rendimiento general de la aplicación. Básicamente, esto lo hemos conseguido mediante el proceso paralelo de los distintos trozos de la imagen mediante multihilo, aunque también se han cambiado algunos cálculos para que sean más eficientes.

\item \textbf{Mejoras en la interfaz:} la interfaz se cambió casi completamente, buscando una mejor intuitividad para el usuario. Por ejemplo, los botones son ahora más claros y se desactivan cuando no se pueden usar, cosa que antes no pasaba y podía llegar a causar problemas.

\item \textbf{Mejoras en la documentación y ayuda:} se ha mejorado la calidad de la API, extendiéndola a todos los elementos del código y arreglando fallos del lenguaje. Además, se ha añadido un módulo de ayuda en línea para permitir al usuario consultar cualquier duda de una forma rápida y sencilla.

\item \textbf{Mejoras en la precisión:} se ha buscado mejorar la detección de defectos implementando nuevas aproximaciones y mejorando la que ya existía.

\item \textbf{Ampliación de funcionalidad:} no sólo se han añadido nuevas aproximaciones para la detección de defectos, sino que también se ha ampliado la funcionalidad de la aplicación en otras formas, como por ejemplo, el cálculo de características geométricas, la interactividad con los defectos detectados, las medidas de \emph{precision \& recall},...

\end{enumerate}

En definitiva, se ha intentado que, a partir de la buena base que representaba el proyecto del año pasado, se pueda ampliar esta idea de una forma mucho más sencilla. Se ha buscado, por tanto, crear una aplicación mucho más fácil de ampliar, ya que este proyecto es candidato a recibir una innumerable cantidad de mejoras prometedoras en un futuro.


\section{Conocimientos adquiridos}
Durante el desarrollo del proyecto se han ido aprendiendo y perfeccionando distintas disciplinas, las cuales aparecen detalladas a continuación.

\subsection{Minería de datos}
Al inicio del proyecto el conocimiento sobre la minería de datos estaba limitado a los conocimientos adquiridos en la asignatura de Minería de Datos de 5º curso de Ingeniería Informática.

Por este motivo cuando se expuso el proyecto se entendió como un reto y una manera de poder aprender sobre esta interesante rama de la informática en la que entran en juego grandes volúmenes de datos.

De este modo y a base de leer artículos, se han asimilado y refinado multitud de conceptos y técnicas.

\subsection{Weka}
En la asignatura de Minería de Datos, de la que ya hemos hablado en el apartado anterior, se utilizó \weka{} para realizar las prácticas. Es por ello que ya teníamos algunos conocimientos sobre esta herramienta.

Este proyecto nos ha permitido usar \weka{} de una forma que no habíamos considerado hasta ahora, y es incluir alguno de sus métodos dentro de nuestra propia aplicación, aprovechando las posibilidades que nos brinda la herramienta.

\subsection{Metodología Scrum}
Las metodologías ágiles han adquirido un gran éxito dentro del desarrollo de software. Por este motivo el proyecto de final de carrera se presentaba como una buena base sobre la que aplicar una de estas metodologías y aprender de ella.

Se eligió \scrum{} por el hecho de que está pensado para entornos en los que cambian los requisitos y es difícil hacer una buena planificación.

Además, \scrum{} se ha explicado en la asignatura de Planificación y Gestión de Proyectos de 4º curso de Ingenería Informática. De este modo la realización del proyecto bajo esta nueva metodología ha aportado una experiencia adicional a los desarrollos clásicos en cascada que son utilizados en multitud de empresas.

\subsection{Programación multihilo}
Para poder mejorar el rendimiento de la aplicación, se hizo necesaria la programación multihilo. Ya poseíamos algunos conocimientos de algunas asignaturas de la carrera, pero nunca lo habíamos usado en Java. Por ello, ha sido necesario leer documentación al respecto y lidiar con algunos problemas que pueden presentar este tipo de aplicaciones.

\subsection{Documentación en \LaTeX{}}
Al principio, la temida documentación se iba a desarrollar con uno de los clásicos compositores
de texto, pero pronto nuestros tutores nos recomendaron utilizar encarecidamente la herramienta
\LaTeX{} \cite{lamport1989latex}. Este hecho, que a priori parecía un reto sencillo, se convirtió en un proceso de cierta complejidad y con una curva de aprendizaje larga e intensa.

En estos momentos damos gracias a nuestros tutores por empeñarse en convencernos a utilizar
\LaTeX{}, ya no solo por el resultado estético que se obtiene, sino por darnos un nuevo reto a superar
que añade valor a la realización de este proyecto y la documentación, además del valor para el
futuro laboral.

Hemos utilizado una plantilla creada por el alumno Álvar Arnáiz González en el proyecto «Biblioteca
de algoritmos de selección de instancias y aplicación orientada a su docencia» \cite{proyectoAlvar}. Futuros alumnos podrán disfrutarla y mejorarla. Tarde o temprano, el hecho de que cada año nuevo
alumnos utilicen y mejoren esta plantilla, hará que la Universidad de Burgos tenga una plantilla
estándar para la realización de cualquier memoria escrita en \LaTeX{}.
