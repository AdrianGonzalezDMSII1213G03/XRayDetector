%%%%%%%%%%%%%%%%%%%%%%%%%%%%%%%%%%%%%%%%%%%%%%%%%%%%%%%%%%%%%%%%%%
%%%%%%%%%%%%%%%%%%%%%%%%%%%%%%%%%%%%%%%%%%%%%%%%%%%%%%%%%%%%%%%%%%
\chapter{Especificación de requisitos}
%%%%%%%%%%%%%%%%%%%%%%%%%%%%%%%%%%%%%%%%%%%%%%%%%%%%%%%%%%%%%%%%%%
%%%%%%%%%%%%%%%%%%%%%%%%%%%%%%%%%%%%%%%%%%%%%%%%%%%%%%%%%%%%%%%%%%

%Introducción
\section{Introducción}
Este anexo tiene como objetivo analizar y documentar las necesidades funcionales que deberán ser soportadas por el sistema a desarrollar. Para  ello, hay que identificar los requisitos que debe satisfacer el nuevo sistema, el estudio de los problemas de las unidades afectadas y sus necesidades actuales.

El objetivo de esta fase es describir lo que el sistema deberá ser capaz de hacer pero no cómo debe hacerlo.

Los tutores del proyecto toma un papel fundamental en este punto del proyecto, ya que asume el rol de cliente, es el encargado de especificar la funcionalidad que la aplicación debe implementar y establecer los plazos para lograrla.

Otra de las tareas a llevar a cabo en el presente anexo es la de priorizar cada una de las funcionalidades a implementar. Teniendo claras cuales son las tareas más críticas se podrá prestar un especial interés en ellas y asignar una cantidad mayor de recursos.

Hay que destacar la importancia de este documento, ya que sirve como punto de partida para el resto del trabajo. Por eso hay que hacer un esfuerzo extraordinario a la hora de definir y declarar correctamente todos los artefactos de ingeniería de software, para no tener que volver sobre nuestros pasos en etapas posteriores del trabajo. En primer lugar y de forma general, se recogen los objetivos que se quieren alcanzar con este trabajo. Estos objetivos son de muy alto nivel y estarán expresados por el cliente en lenguaje natural, siendo tarea del ingeniero su desglose y formalización. A continuación se detalla una lista de los usuarios que han participado en la toma de requisitos así como su relación con el sistema a construir.

Además, se incluye un catálogo con los requisitos que deberá cumplir el sistema a construir, junto con una breve explicación de cada uno, su tipo y su importancia. De estos requisitos se puede confeccionar la lista de los usuarios que interactúan con él, llamados actores. Se debe conocer el perfil de los usuarios para los que está destinada la aplicación, ya que esto condicionará en el futuro muchos aspectos del diseño, como por ejemplo las interfaces o la ayuda.
\newpage



%Objetivos del proyecto
\section{Objetivos del proyecto}\label{obj}
El objetivo principal de este proyecto es el de mejorar la aplicación en diversos aspectos, entre los que se encuentran:

\begin{itemize}
\item Mejora del rendimiento, mediante inclusión de programación multihilo y sustitución de algunos cálculos por otros más eficientes.
\item Mejora de la precisión en la detección de defectos, mediante la implementación de nuevos enfoques.
\item Mejora de la interfaz gráfica, haciéndola más intuitiva y funcional.
\item Mejora del diseño arquitectónico, enfocándolo hacia el mantenimiento y ampliación y solucionando fallos.
\item Mejora del código.
\item Implementación de nuevas funcionalidades, como el cálculo de características geométricas.
\end{itemize}

El proyecto, como vemos, está basado en uno anterior, pero se han tenido que cambiar un gran número de aspectos. Tenemos una parte muy orientada al mantenimiento, como es la mejora de todos los aspectos que ya existen, y otra de adición de nuevas funcionalidades.

Las nuevas funcionalidades incluyen la detección de defectos usando nuevos enfoques, es decir, mediante el uso de los filtros de umbrales locales y nuevas formas de determinar cuándo una ventana es defecto, así como una detección de defectos más interactiva, permitiendo al usuario interactuar con los defectos dibujados. También se incluye la implementación del cálculo de características geométricas y la inclusión de estos cálculos en una tabla interactiva. Estos cálculos permitirán, de una forma sencilla, incluir una clasificación de los defectos en tipos.
\newpage



%Usuarios participantes
\section{Usuarios participantes}
Durante la fase de análisis han participado diversos usuarios, cada uno desempeñando uno o varios papeles. De este modo han sido establecidos los requisitos necesarios para el desarrollo del proyecto software. Cabe destacar la participación de los tutores de este proyecto.
\begin{itemize}
 \item José Francisco Díez Pastor y César Ignacio García Osorio, tutores del proyecto, han asumido diversos roles durante el proceso:
 \begin{itemize}
  \item Como \textit{cliente}, han participado en la fase de análisis describiendo las funcionalidades y el comportamiento del sistema a desarrollar. Asimismo, han organizado el calendario y marcado los plazos de entrega.
  \item Como técnicos, ofreciendo sus conocimientos sobre la minería de datos. Además, César ha aportado sus conocimientos de LaTeX y experiencia en la realización de proyectos previos, y José su experiencia con la biblioteca Weka y con la programación en Java y sus conocimientos en las técnicas de procesamiento digital de imagen.
 \end{itemize}
 \item Por último, Adrián González Duarte y Joaquín Bravo Panadero han asumido el rol de \textit{analista}. Este rol agrupa las siguientes responsabilidades: analizar y describir el problema planteado por el cliente y realizar el diseño con la solución propuesta.
\end{itemize}
\newpage

\section{Descripción del sistema actual}
En este apartado se pretende hacer referencia a la aplicación en la que hemos integrado las funcionalidades descritas anteriormente.

El proyecto en el que está basada nuestra herramienta es \underline{X-Ray Detector: Detección de defectos en piezas metálicas mediante análisis de radiografías}, por Alan Blanco Álamo y Víctor Barbero García, junio de 2012. Este proyecto incluía la funcionalidad de entrenamiento, detección y dibujado de defectos que tiene nuestros proyecto. En un principio, se pensó en simplemente refactorizar su código e ir incluyendo sobre él las nuevas modificaciones, pero al poco tiempo se vio que esto no era del todo sencillo, ya que el diseño y el propio código no eran de la calidad esperada (diseño sin ningún tipo de patrón que facilite la inclusión de nuevas funcionalidades, defectos de código o \textit{bad smells}...), por lo que decidimos rediseñar la aplicación desde el principio prácticamente.

Se ha aprovechado todo lo que se ha podido del proyecto anterior, como son los cálculos de características, si bien se han cambiado un poco para permitir, por ejemplo, que puedan ser usadas a modo de librería. También se ha mejorado alguno de estos cálculos, bien usando funciones e ImageJ o bien usando otras librerías, como EJML.

Una vez rediseñada la aplicación, se ha procedido a incluir las nuevas funcionalidades o mejoras poco a poco sobre este nuevo diseño.

\newpage

%Factores de riesgo
\section{Factores de riesgo}
En este apartado se analizan las dificultades a las que se va a tener que enfrentar el desarrollo de este proyecto software. Identificar los riesgos servirá para estar alerta sobre los posibles problemas que puedan surgir y los retrasos que éstos desencadenen.

Los factores de riesgo que han sido identificados son:
\begin{itemize}
 \item Falta de conocimiento: la primera dificultad encontrada es el desconocimiento de los fundamentos teóricos sobre temas como análiss de imágenes, etc. 
 \item Complejidad de la documentación existente: la documentación disponible se basa en los artículos sobre los algoritmos a implementar y en algún que otro artículo de investigación, por lo que la labor de comprensión de información se presenta difícil.
 \item Complejidad de la aplicación del proyecto del año pasado: se hace difícil comprender el proyecto del año pasado, sobre todo en cuanto al código.
 \item Desconocimiento sobre herramientas: nunca habíamos trabajado con una herramienta como ImageJ, o con librerías complejas como EJML, lo que hace complicado su uso.
 \item Desconocimiento sobre la metodología de desarrollo: si bien es cierto que teníamos una buena base teórica al respecto, nunca habíamos podido poner en práctica los conocimientos sobre las metodologías ágiles, como Scrum, lo que hace algo complicado crear un \textit{backlog}, usar un \textit{tracker}, etc.
 \item Nuevo sistema de composición de texto: para la documentación se va a utilizar \LaTeX{} lo que provocará una carga extra al desarrollo de la memoria del proyecto. Se utilizará para lograr una apariencia más sólida y profesional.
\end{itemize}

Se hace patente que para la creación de la aplicación se requerirá de una gran labor de investigación y análisis que posibilite superar todos los riesgos identificados y que concluyan con la superación de los objetivos marcados en los plazos establecidos.
\newpage



%Catálogo de requisitos del sistema
\section{Catálogo de requisitos del sistema}
El objetivo de este apartado es definir de forma clara, precisa, completa y verificable todas las funcionalidades y restricciones del sistema a construir.



\subsection{Funciones del producto}
Una de las características más importantes de la aplicación que pretende desarrollarse es la facilidad de uso. Dado que su empleo está orientado a convertirse en una herramienta empresarial que ayude a los operarios a identificar piezas defectuosas, debe ser fácil de usar y tener la versatilidad para poder entrenar nuevos conjuntos de imágenes de cualquier nueva pieza que pueda desarrollar la empresa.

La aplicación cuenta con un panel de Log que muestra al usuario datos sobre el progreso del proceso que se está ejecutando.

Para cargar y guardar los conjuntos de datos se usará el formato nativo de \weka{} (\arff{}).


\subsubsection{Características a calcular}
La aplicación deberá calcular las siguientes características, tanto para la imagen normal como para la imagen con Saliency Map aplicado.
\begin{itemize}
 \item Características estándar: 
 \begin{enumerate}
 \item Media
 \item Desviación estándar
 \item Primera derivada
 \item Segunda derivada
 \end{enumerate}
 \item Características de Haralick:
 \begin{enumerate}
 \item Segundo Momento Angular
 \item Contraste
 \item Correlación
 \item Suma de cuadrados
 \item Momento Diferencial Inveros
 \item Suma Promedio
 \item Suma de Entropías
 \item Suma de Varianzas
 \item Entropía
 \item Diferencia de Varianzas
 \item Diferencia de Entropías
 \item Medidas de Información de Correlación 1
 \item Medidas de Información de Correlación 2
 \item Coeficiente de Correlación
 \end{enumerate}
 \item Local Binary Patterns
\end{itemize}

Además, debe ser capaz de calcular las siguientes características geométricas sobre los defectos encontrados:
\begin{itemize}
\item Área
\item Perímetro
\item Circularidad
\item Redondez
\item Semieje mayor
\item Semieje menor
\item Ángulo
\item Distancia Feret
\end{itemize}


\subsubsection{Entrenar el clasificador}
Esta parte deberá posibilitar las siguientes funciones:
\begin{itemize}
 \item Se permitirá al usuario abrir un directorio que contenga radiografías etiquetadas y listas para entrenar.
  \item Se informará al usuario en caso de que el directorio no sea apropiado, ya sea porque el número de imágenes originales y máscaras no coinciden, sus nombres no son los mismos, etc.
 \item Una vez cargado el directorio, se podrá crear un fichero \arff{} que contendrá el cálculo de todas las características para las imágenes de dicho directorio. Este fichero podrá ser guardado donde elija el usuario.
 \item El usuario podrá decidir si quiere que la extracción de características se haga mediante regiones aleatorias o con una ventana deslizante que recorra toda la imagen.
 \item Durante el cálculo de características, se informará al usuario del progreso y de los pasos que se van ejecutando mediante un Log, que podrá ser guardado en el disco duro. Asimismo, una barra de progreso le indicará el porcentaje aproximado de proceso ejecutado.
 \item La ejecución del cálculo de características podrá ser detenida en cualquier momento a petición del usuario.
 \item El entrenamiento se puede realizar también con un \arff{} generado previamente.
\end{itemize}


\subsubsection{Usar el clasificador para detectar defectos}
En este apartado se incluyen las siguientes funciones:
\begin{itemize}
 \item El usuario podrá abrir una imagen para detectar sus defectos.
 \item Se podrá iniciar la detección de defectos, teniendo que cargar un modelo entrenado.
 \item Se podrá seleccionar un área de la imagen para detectar los defectos contenidos en él. Si no se selecciona un área, se debe avisar al usuario de que el proceso puede llevar mucho tiempo.
 \item La detección de defectos podrá ser detenida en cualquier momento a petición del usuario.
 \item Se mostrará la imagen con los defectos marcados en un panel y la información de las características geométricas en una tabla.
 \item Se podrá seleccionar un defecto en la imagen para identificar la fila de la tabla de características geométricas que contiene sus datos. La inversa, seleccionar una fila y que se ilumine el defecto, también debe ser posible.
\end{itemize}

\subsection{Requisitos de usuario}
El usuario debe ser capaz de utilizar la aplicación mediante la interfaz gráfica cargando imágenes, detectando defectos y visualizando los resultados.

La ayuda será un punto a tener en cuenta y en todo momento el usuario podrá solicitarla para orientarse y resolver dudas que le puedan surgir durante el manejo de la misma.


\subsection{Requisitos de sistema}
Al tratarse de una aplicación de escritorio será necesario contar con un hardware que tenga unos requisitos mínimos.

La aplicación se va a desarrollar en \java{} por lo que no debe ser muy exigente, es decir, cualquier máquina capaz de hacer correr la máquina virtual de \java{} debe ser capaz de ejecutar la aplicación.

No obstante, dependiendo del tamaño del fichero \arff{} que se vaya a cargar, o del número de características que se quieran calcular, será recomendable una mayor capacidad hardware para reducir el tiempo de ejecución.

Los requisitos mínimos serán:
\begin{itemize}
 \item Equivalente a 2.8 GHz.
 \item 2 GB de memoria RAM.
 \item Resolución de pantalla igual o superior a 1200 x 720
\end{itemize}

Mientras que los requisitos recomendados serán:
\begin{itemize}
 \item Equivalente a 2.8 GHz(doble núcleo).
 \item 4 GB de memoria RAM.
 \item Resolución de pantalla igual o superior a 1280 x 960
\end{itemize}

Aunque estos requisitos dependerán en gran medida de el número de instancias o características con el que se trabaje.

En cuanto al sistema operativo, deberá ser posible la ejecución en todos los que exista una versión de máquina virtual de \java{}. Uno de los beneficios de que el proyecto se implemente en un lenguaje interpretado es que maximiza las posibilidades de uso en distintos entornos.

La versión de \java{} sobre la que se desarrollará el proyecto es la 1.7 (o 7 directamente).
\newpage



%Especificación de los requisitos
\section{Especificación de los requisitos}
Esta sección del anexo recoge los casos de uso que representan la funcionalidad que debe cubrir la aplicación a desarrollar. Cada caso de uso es una descripción de una secuencia de acciones que se ejecutan en un sistema para producir la salida esperada por un actor.

El primer lugar quedan definidos los requisitos inherentes a la tecnología utilizada y después los requisitos de información. Con este primer análisis se detallan a continuación los requisitos funcionales, apoyados en los casos de uso, y por último los requisitos no funcionales.


\subsection{Especificación de requisitos inherentes a la tecnología utilizada}
Son los requisitos que vienen <<heredados>> por la tecnología a utilizar.

Se ha elegido el lenguaje de programación \java{} para el desarrollo del proyecto, concretamente la versión 7. El requisito necesario para poder ejecutar la aplicación será que el terminal tenga instalada una máquina virtual de \java{}.


\subsection{Especificación de requisitos de información}
El objetivo de este apartado es determinar la información que se debe almacenar para cumplir los objetivos anteriormente descritos y que el programa funcione.
Con objeto de identificar cada uno de los requisitos de información se les ha asignado un código único y un nombre descriptivo. A continuación \vertabla{tablaRequisitosInformacion} se detallan de manera tabulada cada uno de los requisitos en plantillas.

\tablaSinCabecera{Requisitos de información}{p{3.5cm} p{10cm}}{2}{tablaRequisitosInformacion}{
  \rowcolor[gray]{.8}RI - 01       & Información cálculo de características \\\hline
  Descripción                      & Almacena la información resultante del cálculo de características de una radiografía mediante ficheros \arff{}\\\hline
  Datos específicos                & Características calculadas a partir de una radiografía \\\hline
  Importancia                      & Alta \\\hline
  Comentarios                      & Esta información se utilizará para entrenar un clasificador \\\\

  \rowcolor[gray]{.8}RI - 02       & Información de configuración de la ventana de análisis \\\hline
  Descripción                      & Almacena la información necesaria para analizar la imagen mediante ventanas\\\hline
  \multirow{4}*{Datos específicos} & Tamaño de la ventana \\\cline{2-2} 
                                   & Salto entre ventanas \\\cline{2-2}
                                   & Tipo de ventana (aleatoria, deslizante) \\\cline{2-2}	
                                   & Cuándo una ventana es defectuosa (entrenamiento) \\\hline
  Importancia                      & Alta \\\hline
  Comentarios                      & Se almacena todo en un mismo fichero de configuración \\\\
  
  \rowcolor[gray]{.8}RI - 03       & Información para el entrenamiento del clasificador o para el dibujado de características \\\hline
  Descripción                      & Almacena la información necesaria para entrenar un clasificador o para dibujar características \\\hline
 Datos específicos				   & Fichero \arff{} generado al calcular las características de una radiografía  \\\hline
  Importancia                      & Alta \\\hline
  Comentarios                      & Es obligatorio para poder entrenar al clasificador\\\\
  
  \rowcolor[gray]{.8}RI - 04       & Información para la detección de defectos \\\hline
  Descripción                      & Almacena la información necesaria detectar los defectos de una radiografía \\\hline
 Datos específicos				   & Modelo entrenado que utilizará la aplicación para detectar los defectos de una radiografía  \\\hline
  Importancia                      & Alta \\\hline
  Comentarios                      & Es obligatorio para poder detectar los defectos \\\\
  }


\subsection{Requisitos funcionales}
La definición de los requisitos funcionales servirá de ayuda para el diseño de la herramienta.

En la lista aparecen todos los requisitos identificados en la aplicación:
\begin{itemize}
 \item Cargar imágenes: el usuario podrá cargar imágenes para detectar defectos.
 \item Detención del proceso: se podrá detener el proceso en cualquier momento.
 \item Cargar instancias: la aplicación debe estar capacitada para abrir conjuntos de instancias en el formato \arff{} soportado por \weka{}.
 \item Habilitar árbol: se podrán volver a activar las características desactivadas si se quiere calcular otro fichero.
 \item Cargar directorio: se podrá cargar un directorio que contenga radiografías etiquetadas listas para entrenar.
 \item Entrenar clasificador: se podrá entrenar un clasificador para detectar los defectos de nuevas radiografías.
 \item Detectar defectos: se podrá realizar la detección de defectos de cualquier imagen que el usuario cargue, pudiendo seleccionar la región exacta de la imagen que se quiera analizar.
 \item Visualización de resultados: se podrán visualizar los defectos detectados, así como una tabla con las características geométricas.
 \item Interacción con los resultados: se podrá seleccionar un defecto y se iluminará la fila de la tabla correspondiente a ese defecto. Se podrá seleccionar también una fila de la tabla y se iluminará el defecto.
 \item Guardar Log: el usuario podrá visualizar el progreso en un log y guardarlo en el disco duro.
 \item Visualización de la ayuda: las interfaces deberán ofrecer la ayuda al usuario que necesite para poder utilizarlas.
\end{itemize}

Para la definición formal de los casos de uso se hace uso de los diagramas de casos de uso. Para completar la información de los diagramas se utilizan las plantillas donde quedan reflejados los aspectos funcionales.

En el primer diagrama \ver{DiagramaCasosDeUsoGeneralGUI} se muestra el diagrama de casos de uso principal.

%Diagrama general de casos de uso GUI.
\figura{0.9}{imgs/DiagramaCasosDeUso_GeneralGUI.png}{Diagrama de casos de uso principal}{DiagramaCasosDeUsoGeneralGUI}{}

En el diagrama \ver{DiagramaCasosDeUsoVisualizarImagen} se muestra de una manera más detallada la interacción del usuario para la visualización de imágenes.

%Diagrama visualizar imagen
\figura{1}{imgs/DiagramaCasosDeUso_VisualizarImagen.png}{Caso de uso Visualizar Imagen}{DiagramaCasosDeUsoVisualizarImagen}{}

En el diagrama \ver{DiagramaCasosDeUsoCalcularCaracteristicas} se muestran los casos de uso relacionados para calcular las características de una radiografía.

%Diagrama calcular características
\figura{1}{imgs/DiagramaCasosDeUso_CalcularCaracteristicas.png}{Caso de uso Calcular Características}{DiagramaCasosDeUsoCalcularCaracteristicas}{}

En el diagrama \ver{DiagramaCasosDeUsoDibujarCaracteristicas} se muestran los casos de uso relacionados para dibujar las características de una radiografía.

%Diagrama dibujar características
\figura{1}{imgs/DiagramaCasosDeUso_DibujarCaracteristicas.png}{Caso de uso Dibujar Caracteristicas}{DiagramaCasosDeUsoDibujarCaracteristicas}{}

En el diagrama \ver{DiagramaCasosDeUsoEntrenarClasificador} se muestran los casos de uso relacionados para dibujar las características de una radiografía.

%Diagrama entrenar clasificador
\figura{1}{imgs/DiagramaCasosDeUso_EntrenarClasificador.png}{Caso de uso Entrenar Clasificador}{DiagramaCasosDeUsoEntrenarClasificador}{}

En el diagrama \ver{DiagramaCasosDeUsoDetectarDefectos} se muestran los casos de uso relacionados para detectar los defectos de una radiografía.

%Diagrama detectar defectos
\figura{1}{imgs/DiagramaCasosDeUso_DetectarDefectos.png}{Caso de uso Detectar Defectos}{DiagramaCasosDeUsoDetectarDefectos}{}


\newpage
A continuación, aparecen los casos de uso listados y cada una de las plantillas:
\begin{itemize}
 \item RF1 Cargar ayuda \vertabla{tablaRF1}
 \item RF2 Visualizar imagen \vertabla{tablaRF2}
 \item RF3 Calcular características \vertabla{tablaRF3}
 \item RF4 Dibujar características \vertabla{tablaRF4}
 \item RF5 Entrenar clasificador \vertabla{tablaRF5}
 \item RF6 Detectar defectos \vertabla{tablaRF6}
 \item RF7  Abrir imagen \vertabla{tablaRF7}
 \item RF8  Limpiar imagen \vertabla{tablaRF8}
 \item RF9  Hacer Zoom \vertabla{tablaRF9}
 \item RF10 Guardar imagen \vertabla{tablaRF10}
 \item RF11 Seleccionar características \vertabla{tablaRF11}
 \item RF12 Detener proceso \vertabla{tablaRF12}
 \item RF13 Habilitar árbol \vertabla{tablaRF13}
 \item RF14 Guardar log \vertabla{tablaRF14}
 \item RF15 Abrir directorio \vertabla{tablaRF15}
 \item RF16 Dibujar saliency map \vertabla{tablaRF16}
 \item RF17 Cargar fichero \arff{} \vertabla{tablaRF17}
 \item RF18 Normalizar fichero \arff{} \vertabla{tablaRF18}
 \item RF19 Dibujar PCA \vertabla{tablaRF19}
\end{itemize}

\tablaSmallSinColores{Caso de uso: RF1 Cargar ayuda}{p{3cm} p{.75cm} p{9.5cm}}{tablaRF1}{
  \multicolumn{3}{l}{RF1  Cargar ayuda} \\
 }
 {
  Descripción                            & \multicolumn{2}{l}{Cargar ayuda en línea} \\\hline
  Precondiciones                         & \multicolumn{2}{l}{La interfaz debe estar cargada} \\\hline
  \multirow{2}{3.5cm}{Secuencia normal}  & Paso & Acción \\\cline{2-3}
                                         & 1    & El usuario carga la ayuda \\\hline
  Postcondiciones                        & \multicolumn{2}{l}{La ayuda habrá aparecido en pantalla} \\\hline
  \multirow{2}{3.5cm}{Excepciones}       & Paso & Acción \\\cline{2-3}
                                         &      &  \\\hline
  Rendimiento                            &      & \\\hline
  Frecuencia                             & \baja{2} \\\hline
  Importancia                            & \media{2} \\\hline
  Urgencia                               & \baja{2} \\\hline
  Comentarios                            & \multicolumn{2}{l}{La ayuda solo estará disponible en castellano} \\
}

\tablaSmallSinColores{Caso de uso: RF2 Visualizar imagen}{p{3cm} p{.75cm} p{9.5cm}}{tablaRF2}{
  \multicolumn{3}{l}{RF2 Visualizar imagen} \\
 }
 {
  Descripción                            & \multicolumn{2}{l}{Visualizar una imagen} \\\hline
  Precondiciones                         & \multicolumn{2}{l}{La interfaz debe estar cargada} \\\hline
  \multirow{3}{3.5cm}{Secuencia normal}  & Paso & Acción \\\cline{2-3}
                                         & 1    & El usuario abre una imagen  \\\cline{2-3}
                                         & 2    & La imagen se muestra en la pantalla \\\hline
  Postcondiciones                        & \multicolumn{2}{l}{La interfaz deberá mostrar la imagen} \\\hline
  \multirow{2}{3.5cm}{Excepciones}       & Paso & Acción \\\cline{2-3}
                                         &      & \\\hline
  Rendimiento                            &      & \\\hline
  Frecuencia                             & \alta{2} \\\hline
  Importancia                            & \alta{2} \\\hline
  Urgencia                               & \media{2} \\\hline
  Comentarios                            & \multicolumn{2}{l}{La imagen será redimensionada para adaptarla al panel de visualización} \\
}

\tablaSmallSinColores{Caso de uso: RF3 Calcular características}{p{3cm} p{.75cm} p{9.5cm}}{tablaRF3}{
  \multicolumn{3}{l}{RF3 Calcular características} \\
 }
 {
  Descripción                            & \multicolumn{2}{p{10cm}}{Cálculo de las características de una imagen o de todas las imágenes contenidas en un directorio} \\\hline
  Precondiciones                         & \multicolumn{2}{l}{Debe haberse cargado una imagen o un directorio} \\\hline
  \multirow{3}{3.5cm}{Secuencia normal}  & Paso & Acción \\\cline{2-3}
                                         & 1    & El usuario selecciona una o varias características \\\cline{2-3}
                                         & 2    & El usuario inicia el cálculo de características \\\cline{2-3} 							 							 & 3    & Comienza el cálculo de características \\\hline
  Postcondiciones                        & \multicolumn{2}{p{10cm}}{Se guardarán las características calculadas en un fichero \arff{}, se crea una instancia por cada ventana de la imagen.} \\\hline
  \multirow{3}{3.5cm}{Excepciones}       & Paso & Acción \\\cline{2-3}
                                         & 2    & Si el usuario no ha cargado una imagen o un directorio, se mostrará un error \\\cline{2-3}
                                         & 2    & Si el usuario no ha seleccionado ninguna característica, se mostrará un error \\\hline
  Rendimiento                            &      & \\\hline
  Frecuencia                             & \media{2} \\\hline
  Importancia                            & \alta{2} \\\hline
  Urgencia                               & \media{2} \\\hline
  Comentarios                            & \multicolumn{2}{l}{Al calcular las características se mostrará un log y una barra de progreso} \\
}

\tablaSmallSinColores{Caso de uso: RF4 Dibujar características}{p{3cm} p{.75cm} p{9.5cm}}{tablaRF4}{
  \multicolumn{3}{l}{RF4 Dibujar características} \\
 }
 {
  Descripción                            & \multicolumn{2}{l}{Dibujar una interpretación visual de las características} \\\hline
  Precondiciones                         & \multicolumn{2}{l}{Debe haberse cargado una imagen} \\\hline
  \multirow{3}{3.5cm}{Secuencia normal}  & Paso & Acción \\\cline{2-3}
                                         & 1    & El usuario carga un fichero \arff{} \\\cline{2-3}
                                         & 2    & El usuario normaliza el fichero arff \\\cline{2-3} 																 & 3    & El usuario selecciona una característica \\\cline{2-3} 
                                         & 4    & El usuario elige dibujar las características \\\cline{2-3} 
                                         & 5    & Las características se dibujan sobre la imagen \\\hline 		
  Postcondiciones                        & \multicolumn{2}{l}{Las características deberán aparecer dibujadas sobre la imagen} \\\hline
  \multirow{3}{3.5cm}{Excepciones}       & Paso & Acción \\\cline{2-3}
                                         & 1    & Si el usuario no ha cargado una imagen, se mostrará un error \\\cline{2-3}
                                         & 1    & Si el fichero está corrupto, se mostrará un error \\\cline{2-3}
                                         & 2    & Si el usuario no ha cargado un fichero \arff{}, se mostrará un error \\\cline{2-3}
                                         & 4    & Si el usuario no ha cargado una imagen, se mostrará un error \\\cline{2-3}
                                         
                                         & 4    & Si el usuario no ha cargado un fichero \arff{}, se mostrará un error \\\cline{2-3}                      
                                         & 4    & Si el fichero \arff{} no se encuentra normalizado, se mostrará un error \\\cline{2-3}
                                         & 4    & Si no hay ninguna característica seleccionada, se mostrará un error  \\\cline{2-3}
                                         & 4    & Si hay más de una característica seleccionada, se mostrará un error\\\hline
  Rendimiento                            &      & \\\hline
  Frecuencia                             & \alta{2} \\\hline
  Importancia                            & \alta{2} \\\hline
  Urgencia                               & \media{2} \\\hline
  Comentarios                            & \multicolumn{2}{p{10cm}}{Cada línea del fichero arff corresponde a una región analizada por la ventana, y contiene los valores de las características de esa región y las coordenadas de su píxel central. Se va pintando el píxel central de cada región transformando el valor de la característica seleccionada a los tres valores RGB necesarios para pintar.} \\
}

\tablaSmallSinColores{Caso de uso: RF5 Entrenar clasificador}{p{3cm} p{.75cm} p{9.5cm}}{tablaRF5}{
  \multicolumn{3}{l}{RF5 Entrenar clasificador} \\
 }
 {
  Descripción                            & \multicolumn{2}{l}{Entrenar un clasificador} \\\hline
  Precondiciones                         & \multicolumn{2}{l}{Disponer de un fichero \arff{} cargado} \\\hline
  \multirow{2}{3.5cm}{Secuencia normal}  & Paso & Acción \\\cline{2-3}
                                         & 1    & El usuario pulsa el botón <<Entrenar clasificador>> \\\cline{2-3}
                                         & 2    & El clasificador es entrenado con el fichero cargado \\\hline
  Postcondiciones                        & \multicolumn{2}{l}{Se habrá entrenado un clasificador} \\\hline
  \multirow{2}{3.5cm}{Excepciones}       & Paso & Acción \\\cline{2-3}
                                         & 1    & Si no hay un fichero cargado, se mostrará un error\\\hline
  Rendimiento                            &      & \\\hline
  Frecuencia                             & \media{2} \\\hline
  Importancia                            & \alta{2} \\\hline
  Urgencia                               & \media{2} \\\hline
  Comentarios                            & \multicolumn{2}{p{10cm}}{Se informará al usuario mediante un mensaje cuando se haya terminado de entrenar el clasificador} \\
}

\tablaSmallSinColores{Caso de uso: RF6 Detectar defectos}{p{3cm} p{.75cm} p{9.5cm}}{tablaRF6}{
  \multicolumn{3}{l}{RF6 Detectar defectos} \\
 }
 {
  Descripción                            & \multicolumn{2}{l}{Detectar los defectos de una radiografía} \\\hline
  Precondiciones                         & \multicolumn{2}{l}{Debe haberse cargado una imagen} \\\hline
  \multirow{2}{3.5cm}{Secuencia normal}  & Paso & Acción \\\cline{2-3}
										 & 1 	& El usuario selecciona una región de la imagen para analizar \\\cline{2-3}                                         
                                         & 2    & El usuario pulsa el botón <<Detectar defectos>> \\\cline{2-3}
                                         & 3    & Se pide al usuario que seleccione un modelo entrenado \\\cline{2-3} 
                                         & 4    & Se produce la detección de defectos\\\hline
  Postcondiciones                        & \multicolumn{2}{l}{Una vez finalizada la detección, los defectos se mostrarán sobre la imagen} \\\hline
  \multirow{2}{3.5cm}{Excepciones}       & Paso & Acción \\\cline{2-3}
                                         & 1    & Si no hay una imagen abierta, se mostrará un error \\\cline{2-3} 
                                         & 2 	& Si no hay una región seleccionada, se mostrará un error \\\hline
  Rendimiento                            &      & \\\hline
  Frecuencia                             & \alta{2} \\\hline
  Importancia                            & \alta{2} \\\hline
  Urgencia                               & \alta{2} \\\hline
  Comentarios                            & \multicolumn{2}{l}{Al detectar los defectos se mostrará un log y una barra de progreso} \\
}

\tablaSmallSinColores{Caso de uso: RF7 Abrir imagen}{p{3cm} p{.75cm} p{9.5cm}}{tablaRF7}{
  \multicolumn{3}{l}{RF7  Abrir imagen} \\
 }
 {
  Descripción                            & \multicolumn{2}{l}{Abrir una imagen} \\\hline
  Precondiciones                         &      & \\\hline
  \multirow{2}{3.5cm}{Secuencia normal}  & Paso & Acción \\\cline{2-3}
                                         & 1    & El usuario pulsa el botón <<Abrir imagen>> \\\cline{2-3} 
                                         & 2    & El usuario elige la imagen mediante una ventana de selección \\\hline
  Postcondiciones                        & \multicolumn{2}{l}{La imagen se habrá cargado} \\\hline
  \multirow{2}{3.5cm}{Excepciones}       & Paso & Acción \\\cline{2-3}
                                         &      &  \\\hline
  Rendimiento                            &      & \\\hline
  Frecuencia                             & \alta{2} \\\hline
  Importancia                            & \alta{2} \\\hline
  Urgencia                               & \media{2} \\\hline
  Comentarios                            & \multicolumn{2}{l}{Se mostrará una vista previa de las imágenes antes de abrirlas} \\
}

\tablaSmallSinColores{Caso de uso: RF8 Limpiar imagen}{p{3cm} p{.75cm} p{9.5cm}}{tablaRF8}{
  \multicolumn{3}{l}{RF8  Limpiar imagen} \\
 }
 {
  Descripción                            & \multicolumn{2}{l}{Limpiar la imagen} \\\hline
  Precondiciones                         & \multicolumn{2}{l}{Debe haberse cargado una imagen} \\\hline
  \multirow{2}{3.5cm}{Secuencia normal}  & Paso & Acción \\\cline{2-3}
                                         & 1    & El usuario pulsa el botón <<Limpiar imagen>> \\\cline{2-3} 
                                         & 2    & Se limpiará la imagen \\\hline
  Postcondiciones                        & \multicolumn{2}{l}{La imagen se mostrará tal y como era originalmente} \\\hline
  \multirow{2}{3.5cm}{Excepciones}       & Paso & Acción \\\cline{2-3}
                                         &      &  \\\hline
  Rendimiento                            &      & \\\hline
  Frecuencia                             & \media{2} \\\hline
  Importancia                            & \baja{2} \\\hline
  Urgencia                               & \baja{2} \\\hline
  Comentarios                            & \multicolumn{2}{p{10cm}}{Con <<limpiar imagen>> se quiere decir que se volverá a cargar la imagen en caso de que se hayan dibujado características y se quiera volver a verla sin las características dibujadas. \newline Si no se ha dibujado nada en la imagen, el botón no tendrá ningún efecto.} \\
}

\tablaSmallSinColores{Caso de uso: RF9 Hacer Zoom}{p{3cm} p{.75cm} p{9.5cm}}{tablaRF9}{
  \multicolumn{3}{l}{RF9  Hacer Zoom} \\
 }
 {
  Descripción                            & \multicolumn{2}{l}{Hacer zoom sobre la imagen} \\\hline
  Precondiciones                         & \multicolumn{2}{l}{Debe haberse cargado una imagen} \\\hline
  \multirow{2}{3.5cm}{Secuencia normal}  & Paso & Acción \\\cline{2-3}
                                         & 1    & El usuario pulsa el botón <<Zoom>> \\\cline{2-3}
                                         & 2    & Se mostrará la imagen aumentada \\\hline
  Postcondiciones                        & \multicolumn{2}{l}{Se mostrará una nueva ventana en la que se podrá hacer zoom} \\\hline
  \multirow{2}{3.5cm}{Excepciones}       & Paso & Acción \\\cline{2-3}
                                         &      &  \\\hline
  Rendimiento                            &      & \\\hline
  Frecuencia                             & \media{2} \\\hline
  Importancia                            & \baja{2} \\\hline
  Urgencia                               & \baja{2} \\\hline
  Comentarios                            & \multicolumn{2}{l}{Se podrá aumentar y reducir el zoom aplicado sobre la imagen} \\
}

\tablaSmallSinColores{Caso de uso: RF10 Guardar imagen}{p{3cm} p{.75cm} p{9.5cm}}{tablaRF10}{
  \multicolumn{3}{l}{RF10 Guardar imagen} \\
 }
 {
  Descripción                            & \multicolumn{2}{l}{Guardar la imagen en el disco duro} \\\hline
  Precondiciones                         & \multicolumn{2}{l}{Deberá cargarse una imagen y dibujar alguna característica en ella} \\\hline
  \multirow{2}{3.5cm}{Secuencia normal}  & Paso & Acción \\\cline{2-3}
                                         & 1    & El usuario pulsa el botón <<Guardar imagen>> \\\cline{2-3} 
                                         & 2    & El usuario selecciona el nombre y el lugar donde guardar la imagen \\\hline
  Postcondiciones                        & \multicolumn{2}{l}{La imagen con las características dibujadas se guardará en el disco duro} \\\hline
   \multirow{2}{3.5cm}{Excepciones}       & Paso & Acción \\\cline{2-3}
                                         & 1    & Si no se ha dibujado ninguna característica sobre la imagen, no se podrá guardar\\\hline
  Rendimiento                            &      & \\\hline
  Frecuencia                             & \media{2} \\\hline
  Importancia                            & \media{2} \\\hline
  Urgencia                               & \media{2} \\\hline
  Comentarios                            & \multicolumn{2}{l}{Las imágenes se guardarán con formato jpg} \\
}

\tablaSmallSinColores{Caso de uso: RF11 Seleccionar características}{p{3cm} p{.75cm} p{9.5cm}}{tablaRF11}{
  \multicolumn{3}{l}{RF11 Seleccionar características} \\
 }
 {
  Descripción                            & \multicolumn{2}{l}{Seleccionar las características de una imagen} \\\hline
  Precondiciones                         &      & \\\hline
  \multirow{2}{3.5cm}{Secuencia normal}  & Paso & Acción \\\cline{2-3}
                                         & 1    & El usuario selecciona las características que desea calcular o dibujar \\\hline
  Postcondiciones                        & \multicolumn{2}{p{8.5cm}}{Las características seleccionadas aparecerán marcadas en el árbol de selección} \\\hline
    \multirow{2}{3.5cm}{Excepciones}       & Paso & Acción \\\cline{2-3}
                                         &      &  \\\hline
  Rendimiento                            &      & \\\hline
  Frecuencia                             & \alta{2} \\\hline
  Importancia                            & \alta{2} \\\hline
  Urgencia                               & \alta{2} \\\hline
  Comentarios                            & \multicolumn{2}{p{10cm}}{Al cargar un fichero \arff{}, se desactivarán las características no contenidas en él. \newline Las características aparecen ordenadas por categoría: Estándar, Haralick y Local Binary Patterns.
   \newline Se pueden seleccionar de una en una, todas o una categoría completa.} \\
}

\tablaSmallSinColores{Caso de uso: RF12 Detener proceso}{p{3cm} p{.75cm} p{9.5cm}}{tablaRF12}{
  \multicolumn{3}{l}{RF12 Detener proceso} \\
 }
 {
  Descripción                            & \multicolumn{2}{l}{Detener el proceso} \\\hline
  Precondiciones                         & \multicolumn{2}{p{8.5cm}}{Tiene que estar en ejecución el cálculo de características o la detección de defectos} \\\hline
  \multirow{2}{3.5cm}{Secuencia normal}  & Paso & Acción \\\cline{2-3}
                                         & 1    & El usuario pulsa el botón <<Detener proceso>> \\\cline{2-3}
                                         & 2    & Se pregunta al usuario si realmente quiere detener el proceso \\\cline{2-3}
                                         & 3    & El usuario confirma que quiere detener el proceso \\\hline
  Postcondiciones                        & \multicolumn{2}{l}{El proceso se habrá detenido} \\\hline
    \multirow{2}{3.5cm}{Excepciones}       & Paso & Acción \\\cline{2-3}
                                         &      &  \\\hline
  Rendimiento                            &      & \\\hline
  Frecuencia                             & \baja{2} \\\hline
  Importancia                            & \media{2} \\\hline
  Urgencia                               & \media{2} \\\hline
  Comentarios                            &      & \\ \\
}

\tablaSmallSinColores{Caso de uso: RF13 Habilitar árbol}{p{3cm} p{.75cm} p{9.5cm}}{tablaRF13}{
  \multicolumn{3}{l}{RF13 Habilitar árbol} \\
 }
 {
  Descripción                            & \multicolumn{2}{l}{Habilitar el árbol de selección de características} \\\hline
  Precondiciones                         & \multicolumn{2}{l}{Deberá haber alguna característica desactivada en el árbol} \\\hline
  \multirow{2}{3.5cm}{Secuencia normal}  & Paso & Acción \\\cline{2-3}
                                         & 1    & El usuario pulsa el botón <<Habilitar árbol>> \\\hline
  Postcondiciones                        & \multicolumn{2}{l}{Se volverán a activar todas las características} \\\hline
    \multirow{2}{3.5cm}{Excepciones}       & Paso & Acción \\\cline{2-3}
                                         &      &  \\\hline
  Rendimiento                            &      & \\\hline
  Frecuencia                             & \baja{2} \\\hline
  Importancia                            & \baja{2} \\\hline
  Urgencia                               & \baja{2} \\\hline
  Comentarios                            & \multicolumn{2}{l}{Si todas las características están activadas, no pasará nada} \\
}

\tablaSmallSinColores{Caso de uso: RF14 Guardar log}{p{3cm} p{.75cm} p{9.5cm}}{tablaRF14}{
  \multicolumn{3}{l}{RF14 Guardar log} \\
 }
 {
  Descripción                            & \multicolumn{2}{l}{Guardar el log} \\\hline
 Precondiciones                         &      & \\\hline
  \multirow{2}{3.5cm}{Secuencia normal}  & Paso & Acción \\\cline{2-3}
                                         & 1    & El usuario pulsa el botón <<Guardar log>> \\\hline
  Postcondiciones                        & \multicolumn{2}{l}{El log se guarda en un archivo de texto plano} \\\hline
    \multirow{2}{3.5cm}{Excepciones}       & Paso & Acción \\\cline{2-3}
                                         &      &  \\\hline
  Rendimiento                            &      & \\\hline
  Frecuencia                             & \baja{2} \\\hline
  Importancia                            & \baja{2} \\\hline
  Urgencia                               & \baja{2} \\\hline
  Comentarios                            & \multicolumn{2}{p{10cm}}{En el log se guardarán datos como el tamaño de la imagen a analizar, el número de instancias que se van analizando, etc.} \\
}

\tablaSmallSinColores{Caso de uso: RF15 Abrir directorio}{p{3cm} p{.75cm} p{9.5cm}}{tablaRF15}{
  \multicolumn{3}{l}{RF15 Abrir directorio} \\
 }
 {
  Descripción                            & \multicolumn{2}{l}{Cargar todas las imágenes que están dentro de un directorio} \\\hline
  Precondiciones                         &      & \\\hline
  \multirow{2}{3.5cm}{Secuencia normal}  & Paso & Acción \\\cline{2-3}
                                         & 1    & El usuario pulsa el botón <<Abrir directorio>> \\\cline{2-3} 
                                         & 2    & El usuario elige el directorio mediante una ventana de selección \\\hline
  Postcondiciones                        & \multicolumn{2}{l}{Se cargarán las imágenes que contenga dicho directorio} \\\hline
   \multirow{3}{3.5cm}{Excepciones}       & Paso & Acción \\\cline{2-3}
                                         & 2    & Si el nombre de la carpeta no es <<originales>>, mostrará un error \\\cline{2-3}
                                         & 2    & Si no coinciden el número de máscaras y originales, mostrará un error \\\cline{2-3}
                                         & 2    & Si no coinciden los nombres de máscaras y originales, mostrará un error \\\hline
  Rendimiento                            &      & \\\hline
  Frecuencia                             & \alta{2} \\\hline
  Importancia                            & \alta{2} \\\hline
  Urgencia                               & \alta{2} \\\hline
  Comentarios                            &      & \\
}

\tablaSmallSinColores{Caso de uso: RF16 Dibujar saliency map}{p{3cm} p{.75cm} p{9.5cm}}{tablaRF16}{
  \multicolumn{3}{l}{RF16 Dibujar saliency map} \\
 }
 {
  Descripción                            & \multicolumn{2}{l}{Dibujar saliency map} \\\hline
  Precondiciones                         & \multicolumn{2}{l}{Debe haberse cargado una imagen} \\\hline
  \multirow{2}{3.5cm}{Secuencia normal}  & Paso & Acción \\\cline{2-3}
                                         & 1    & El usuario pulsa el botón <<Dibujar Saliency Map>> \\\hline
  Postcondiciones                        & \multicolumn{2}{l}{Se mostrará el resultado de realizar saliency map sobre la imagen} \\\hline
   \multirow{3}{3.5cm}{Excepciones}       & Paso & Acción \\\cline{2-3}
                                         & 1    & Si no se ha cargado una imagen, se informará al usuario \\\hline
  Rendimiento                            &      & \\\hline
  Frecuencia                             & \media{2} \\\hline
  Importancia                            & \alta{2} \\\hline
  Urgencia                               & \media{2} \\\hline
  Comentarios                            & \multicolumn{2}{p{10cm}}{Para ver una definición de Saliency Map, ir al apartado \ref{saliency}} \\
}

\tablaSmallSinColores{Caso de uso: RF17 Cargar fichero \arff{}}{p{3cm} p{.75cm} p{9.5cm}}{tablaRF17}{
  \multicolumn{3}{l}{RF17 Cargar fichero \arff{}} \\
 }
 {
  Descripción                            & \multicolumn{2}{l}{Cargar un fichero \arff{}} \\\hline
  Precondiciones                         &      & \\\hline
  \multirow{2}{3.5cm}{Secuencia normal}  & Paso & Acción \\\cline{2-3}
                                         & 1    & El usuario pulsa el botón <<Cargar Fichero \arff{}>> \\\cline{2-3}
                                          & 2    & El usuario elige el fichero \arff{} \\\hline
  Postcondiciones                        & \multicolumn{2}{l}{El fichero es cargado} \\\hline
  \multirow{3}{3.5cm}{Excepciones}       & Paso & Acción \\\cline{2-3}
                                         & 2    & Si el fichero está corrupto, mostrará un error \\\cline{2-3}
                                         & 2    & Si el fichero tiene una extensión no soportada, informará al usuario \\\hline
  Rendimiento                            &      & \\\hline
  Frecuencia                             & \alta{2} \\\hline
  Importancia                            & \alta{2} \\\hline
  Urgencia                               & \alta{2} \\\hline
  Comentarios                            & \multicolumn{2}{p{10cm}}{Se informará al usuario de que el fichero se ha cargado correctamente.
  \newline El fichero \arff{} contiene las características calculadas a partir de una radiografía.} \\
}

\tablaSmallSinColores{Caso de uso: RF18 Normalizar fichero \arff{}}{p{3cm} p{.75cm} p{9.5cm}}{tablaRF18}{
  \multicolumn{3}{l}{RF18 Normalizar fichero \arff{}} \\
 }
 {
  Descripción                            & \multicolumn{2}{l}{Normalizar un fichero \arff{}} \\\hline
  Precondiciones                         & \multicolumn{2}{l}{Disponer de un fichero \arff{} cargado} \\\hline
  \multirow{2}{3.5cm}{Secuencia normal}  & Paso & Acción \\\cline{2-3}
                                         & 1    & El usuario pulsa el botón <<Normalizar Fichero \arff{}>> \\\cline{2-3}
                                         & 2    & Se normaliza el fichero y se guarda en el disco duro \\\hline
  Postcondiciones                        & \multicolumn{2}{l}{El fichero normalizado se guardará en el disco duro} \\\hline
  \multirow{3}{3.5cm}{Excepciones}       & Paso & Acción \\\cline{2-3}
                                         & 1    & Si no hay un fichero \arff{} cargado, se mostrará un error\\\hline
  Rendimiento                            &      & \\\hline
  Frecuencia                             & \media{2} \\\hline
  Importancia                            & \alta{2} \\\hline
  Urgencia                               & \media{2} \\\hline
  Comentarios                            & \multicolumn{2}{p{10cm}}{
 Normalizar un fichero \arff{} consiste en, para cada atributo (o columna), coger todos los valores, buscar el máximo y el mínimo, y ajustar el resto de valores para que estén entre ese máximo y ese mínimo. \newline Este paso es necesario para poder dibujar las características.} \\
}

\tablaSmallSinColores{Caso de uso: RF19 Dibujar PCA}{p{3cm} p{.75cm} p{9.5cm}}{tablaRF19}{
  \multicolumn{3}{l}{RF19 Dibujar PCA} \\
 }
 {
  Descripción                            & \multicolumn{2}{l}{Dibujar PCA sobre una imagen} \\\hline
  Precondiciones                         & \multicolumn{2}{l}{Debe haberse cargado una imagen} \\\hline
  \multirow{3}{3.5cm}{Secuencia normal}  & Paso & Acción \\\cline{2-3}
                                         & 1    & El usuario carga un fichero \arff{} \\\cline{2-3}
                                         & 2    & El usuario normaliza el fichero arff \\\cline{2-3} 																 & 3    & El usuario selecciona una o varias característica \\\cline{2-3} 
                                         & 4    & El usuario pulsa el botón <<Dibujar PCA>> \\\cline{2-3} 
                                         & 5    & Se dibuja el PCA sobre la imagen \\\hline 		
  Postcondiciones                        & \multicolumn{2}{l}{La representación del PCA deberá aparecer dibujada en la imagen} \\\hline
  \multirow{3}{3.5cm}{Excepciones}       & Paso & Acción \\\cline{2-3}
                                         & 1    & Si el usuario no ha cargado una imagen, se mostrará un error \\\cline{2-3}
                                         & 1    & Si el fichero está corrupto, se mostrará un error \\\cline{2-3}
                                         & 2    & Si el usuario no ha cargado un fichero \arff{}, se mostrará un error \\\cline{2-3}
                                         & 4    & Si el usuario no ha cargado una imagen, se mostrará un error \\\cline{2-3}
                                         & 4    & Si el fichero \arff{} no se encuentra normalizado, se mostrará un error \\\cline{2-3}
                                         & 4    & Si el usuario no ha cargado un fichero \arff{}, se mostrará un error \\\cline{2-3}
                                         & 4    & Si no hay ninguna característica seleccionada, se mostrará un error \\\hline
  Rendimiento                            &      & \\\hline
  Frecuencia                             & \alta{2} \\\hline
  Importancia                            & \alta{2} \\\hline
  Urgencia                               & \media{2} \\\hline
  Comentarios                            & \multicolumn{2}{p{10cm}}{Para ver una definición de PCA, ir al apartado \ref{pca}} \\
}


\newpage
\subsection{Requisitos no funcionales}
Una vez analizados los requisitos de información y los funcionales falta un tipo de requisitos que, normalmente, suelen ser de carácter técnico y se engloban dentro de requisitos no funcionales.

A continuación aparecen listados los requisitos no funcionales a tener en cuenta para el diseño de la aplicación:

\begin{itemize}
 \item Extensible: debe estar pensado para que se pueda ampliar añadiendo nuevas características a calcular, utilizando la misma interfaz.
 \item Máxima información: la herramienta deberá mostrar la máxima información al usuario, ya que la parte de dibujar características podría llegar a emplearse con fines didácticos. Por ello, en el diseño se debe tener en cuenta este requisito.
 \item Documentado: si se pretende que un software sea ampliado por terceros, o continuado en un futuro, éste debe estar debidamente documentado.
\end{itemize}
\newpage


%Interfaz de usuario
\section{Interfaz de usuario}
Como se ha comentado con anterioridad, la aplicación tendrá una interfaz con dos pestañas, una para dibujar las características y otra para la detección de defectos.

La interfaz gráfica de usuario requiere un tratamiento especial y un análisis en profundidad para hacer de la aplicación una herramienta útil y fácil de utilizar.

En la imagen \ver{prototype1} se muestra la pestaña <<Pintar Características>>. En ella se podrá abrir una radiografía, calcular sus características, dibujarlas de una en una o varias mediante PCA...

\figuraConPosicionSinMarco{1}{imgs/prototipo1.png}{Prototipo de la pestaña Pintar Características}{prototype1}{}{H}

En la imagen \ver{prototype2} se muestra la pestaña <<Detectar Defectos>>. En ella se podrá abrir un directorio con imágenes etiquetadas para entrenar un clasificador y luego usarlo para detectar los defectos de una radiografía cualquiera.

\figuraConPosicionSinMarco{1}{imgs/prototipo2.png}{Prototipo de la pestaña Detección de Defectos}{prototype2}{}{H}