% Fuentes de la documentación del proyecto 

\documentclass{memoriaPFC}

% Incluir el estilo definido para la Universidad de Burgos por Álvar Arnáiz 2009/12/20.
\usepackage{ubupro}

%\usepackage{cgotest}

%
% Incluir los comandos para las palabras t'ecnicas.
%
% Almacenar los comandos propios del proyecto de Selección de Instancias.
\newcommand{\scrum}{\ubuemph{Scrum}}
\newcommand{\sprint}{\ubuemph{Sprint}}
\newcommand{\sprints}{\ubuemph{Sprints}}
\newcommand{\productbacklog}{\ubuemph{Product Backlog}}
\newcommand{\backlogitem}{\ubuemph{Backlog Item}}
\newcommand{\backlogitems}{\ubuemph{Backlog Items}}
\newcommand{\productitem}{\ubuemph{Product Item}}
\newcommand{\sprintbacklog}{\ubuemph{Sprint Backlog}}
\newcommand{\burndownchart}{\ubuemph{Burndown Chart}}

\newcommand{\is}{\ubuemph{Instance Selection}}
\newcommand{\enn}{\ubuemph{ENN}}
\newcommand{\cnn}{\ubuemph{CNN}}
\newcommand{\rnn}{\ubuemph{RNN}}
\newcommand{\mss}{\ubuemph{MSS}}
\newcommand{\icf}{\ubuemph{ICF}}
\newcommand{\drop}{\ubuemph{DROP}}
\newcommand{\dropuno}{\ubuemph{DROP1}}
\newcommand{\dropdos}{\ubuemph{DROP2}}
\newcommand{\droptres}{\ubuemph{DROP3}}
\newcommand{\dropcuatro}{\ubuemph{DROP4}}
\newcommand{\dropcinco}{\ubuemph{DROP5}}

\newcommand{\dis}{\ubuemph{DIS}}
\newcommand{\demois}{\ubuemph{Democratic Instance Selection}}
\newcommand{\gt}{\ubuemph{Grand Tour}}

\newcommand{\clang}{\ubuemph{C}}
\newcommand{\cpp}{\ubuemph{C++}}
\newcommand{\java}{\ubuemph{Java}}
\newcommand{\javaversion}{\ubuemph{Java 6}}
\newcommand{\linux}{\ubuemph{Linux}}

\newcommand{\weka}{\ubuemph{Weka}}
\newcommand{\arff}{\ubuemph{ARFF}}
\newcommand{\xrff}{\ubuemph{XRFF}}
\newcommand{\csv}{\ubuemph{CSV}}

\newcommand{\gui}{\ubuemph{GUI}}
\newcommand{\cli}{\ubuemph{CLI}}

\newcommand{\uml}{\ubuemph{UML}}
\newcommand{\jude}{\ubuemph{Jude}}
\newcommand{\ant}{\ubuemph{Ant}}

\newcommand{\voronoi}{\ubuemph{Voronoi}}

\newcommand{\gnu}{\ubuemph{GNU}}
\newcommand{\epl}{\ubuemph{EPL}}


%
% DATOS DEL DOCUMENTO
%\autores{Nombre1 Apellidos1}{Nombre2 Apellidos2}
\title{X-RayDetector: Detección automática de defectos en piezas
metálicas mediante análisis de radiografías}
\autores{Adrián González Duarte}{Joaquín Bravo Panadero}
% Modificado por C´esar Garc´ia Osorio 6/12/2012
%\director{César I. García Osorio, Juan José Rodríguez Díez}
\directores{Dr. César I. García Osorio}{José Francisco Díez Pastor}

\date{junio de 2012}
\titulacion{ii}

\resumen{%
La inspección de defectos es una tarea muy importante para asegurar la seguridad y la fiabilidad de los procesos industriales. Una de las últimas tendencias en la inspección y control de calidad de los procesos industriales son las técnicas de prueba no destructiva (NDT - Non-Destructive Testing). En estas técnicas se evalúan los defectos superficiales o internos a través de distintos métodos como son la radiografía y la ecografía, entre otros.

La radiografía se ha mostrado como uno de los métodos más efectivos a la hora de identificar defectos en procesos de inyección de material en molde. Este método se basa en que las zonas con algún tipo de defecto absorben distinta cantidad de energía que el resto de zonas, lo que ocasiona regiones más oscuras o claras en la imagen.

La etapa de interpretación de los defectos en imágenes de radiografía es un proceso costoso en tiempo y subjetivo, produciéndose contradicciones entre las evaluaciones de los distintos inspectores de la imagen. Este trabajo es la segunda iteración de un proyecto que pretende desarrollar un método para la detección y clasificación de defectos en imágenes de radiografía de piezas de magnesio, que acelere el proceso de inspección de fallos y elimine la ambigüedad y subjetividad del proceso manual.
}
\descriptores{Detección de defectos, minería de datos, procesamiento digital de la imagen, visión artificial, aprendizaje automático, radiografías}

\abstract{%
The defect inspection is an important task to ensure the safety and reliability of industrial processes.
Some of the latest trends in inspection and quality control of industrial processes are the non-destructive testing techniques (NDT - Non-Destructive Testing). These techniques are evaluated on the surface or internal defects through different methods such as radiography and ultrasound, among others.

Radiography has proven to be one of the most effective techniques to identify defects in the process of injection-molded material. This method is based on the fact that areas with some kind of defect absorb different amounts of energy than other areas, resulting in darker or lighter regions in the image.

The stage of defects' interpretation in radiographic images is a subjective and time consuming process, resulting in contradictions between different operators examining the same. This work is the second iteration of a project for the detection and classification of defects in radiographic images of magnesium parts, which aim is to speed up the process of defect detection and the elimination of the ambiguity and subjectivity of the human inspection.
}
\keywords{Defect detection, Data Mining, digital image processing, computer vision, Machine Learning, radiography}


%
% COMIENZO DEL DOCUMENTO
\begin{document}

%
% Portada, resumen, indices
\frontmatter
\ubuportada
\hacerresumen
\tableofcontents
\listoffigures % Opcional
\listoftables % Opcional
%\lstlistoflistings % Opcional

%
% Contenidos %%%%%%%%%%%%%%%%%%%%%%%%%%%%%%%%%%%%%%%%%%%%%%%%%%%%%%%%%
\mainmatter

\subsection*{Agradecimientos}
Aquí irán los agradecimientos.




%Esto hay que cambiarlo. Meter nuestros objetivos y tal.

El objetivo principal de este proyecto consiste en el desarrollo de una aplicación de escritorio que implemente y facilite la comprensión de los algoritmos de \textit{Selección de Instancias clásicos}.

La selección de instancias tiene como objetivo disminuir el tamaño de las muestras para posibilitar su uso posterior. Los algoritmos de selección de instancias pretenden disminuir la complejidad mediante la reducción del número de instancias, extrayendo así las más significativas y desechando las que no aporten información al conjunto.

La necesidad de la selección de instancias se hace patente cuando se examinan los conjuntos de datos que se manejan en la vida real. Intentar entrenar un clasificador, por ejemplo, a partir de mil millones de instancias puede ser una tarea difícil e, incluso, imposible. Por ello la selección de instancias se presenta como una buena alternativa para reducir la complejidad de la muestra posibilitando su posterior tratamiento.

Actualmente los algoritmos de selección de instancias que se manejan tienen un gran inconveniente y es su complejidad. Si utilizamos la notación de Landau, el mejor de los algoritmos clásicos de selección de instancias es de complejidad $ O(n^{2}) $ por lo que la simple tarea de seleccionar las instancias puede ser inabordable.

En este proyecto se va a implementar el algoritmo \demois{} \cite{democratic_instance_selection}, \dis{} a partir de ahora, el cual es un nuevo algoritmo de selección de instancias basado en la técnica \textit{divide y vencerás} y cuya complejidad es $ O(n \log(n)) $. Este algoritmo ha sido creado por César Ignacio García Osorio, tutor de este proyecto, y Nicolás Pedrajas García.

Con todo lo anteriormente expuesto, el desarrollo de este proyecto se presenta interesante desde dos puntos de vista: el primero es la poca orientación pedagógica de la información existente sobre los algoritmos de selección de instancias, y el segundo el desarrollo e implementación del nuevo algoritmo \dis{}. Del primer punto se extrae que el desarrollo del software debe estar diseñado para el aprendizaje de dichos algoritmos, facilitando la comprensión de los mismos y posibilitando su seguimiento. Del segundo punto se extrae que la herramienta debe tener una arquitectura pensada para ser extensible, es decir, puedan añadirse nuevos algoritmos que sigan una interfaz común.

El desarrollo se implementará sobre el lenguaje de programación \javaversion{} y será capaz de ejecutar los algoritmos paso a paso, facilitando la información de las estructuras de datos que éste maneje durante su funcionamiento. Tras un análisis de alternativas, se utilizará la librería \weka{} (Waikato Environment for Knowledge Analysis) para el tratamiento y manejo de las instancias y muestras debido a su versatilidad y potencia.

Durante el desarrollo, se utilizará la metodología ágil \scrum{}. El principal motivo de escoger una metodología ágil es el desconocimiento por parte del equipo de desarrollo del entorno, haciendo imposible una planificación estructurada del mismo desde el momento del inicio del proyecto. Gracias a \scrum{} el proyecto avanzará planificándose, en cada \sprint{}, las tareas que se vayan extrayendo del \productbacklog{}.

%%%%%%%%%%%%%%%%%%%%%%%%%%%%%%%%%%%%%%%%%%%%%%%%%%%%%%%%%%%%%%%%%%
%%%%%%%%%%%%%%%%%%%%%%%%%%%%%%%%%%%%%%%%%%%%%%%%%%%%%%%%%%%%%%%%%%
\chapter{Objetivos del proyecto}
%%%%%%%%%%%%%%%%%%%%%%%%%%%%%%%%%%%%%%%%%%%%%%%%%%%%%%%%%%%%%%%%%%
%%%%%%%%%%%%%%%%%%%%%%%%%%%%%%%%%%%%%%%%%%%%%%%%%%%%%%%%%%%%%%%%%%

Los objetivos principales del proyecto son principalmente cuatro:
\begin{enumerate}
 \item Rediseñar completamente la aplicación que presentaron los alumnos del año pasado, buscando un diseño mucho más modular y reutilizable.
 \item Refactorizar todo el código que reutilicemos, buscando un mejor estilo para favorecer su comprensión y reutilización.
 \item Mejorar el rendimiento de la aplicación anterior, buscando que la misma se pueda aprovechar de los procesadores actuales de más de un núcleo de proceso.
 \item Mejorar la precisión a la hora de localizar los defectos.
 \item Ampliación de la funcionalidad de la aplicación anterior en diversos aspectos, como la implementación de algoritmos de segmentación que permitan obtener un resumen de las características geométricas de los defectos encontrados que, en un futuro, permitirán clasificar los defectos en varios y tipos y decidir si la pieza es defectuosa y debe ser retirada o si es correcta.
\end{enumerate}

En el proyecto se van a implementar los algoritmos de selección de características más relevantes o significativos que hemos encontrado, prestando especial atención al diseño para que sea fácil la inclusión de nuevos algoritmos en el futuro. Esto posibilitará que el proyecto crezca en el tiempo y amplíe su capacidad.

\section{Objetivos Técnicos}
Este proyecto se va a desarrollar utilizando el paradigma de la \textit{Orientacion a Objetos} el cual nos permitirá un diseño fácilmente comprensible por futuros desarrolladores que deseen proseguir con el presente trabajo. 

El lenguaje de programación escogido para el proyecto es \javaversion{}. El motivo de su elección ha sido el conocimiento del mismo, su sencillez, portabilidad, extensa documentación y la posibilidad de utilizar la librería de \weka{}. La completa \textit{API (Application Programming Interface)} con la que cuenta y los conocimientos adquiridos durante la carrera sobre este lenguaje de programación harán posible que el desarrollo se base en el estudio e implementación de los algoritmos evitando retrasos por tener que aprender un nuevo lenguaje de programación

El lenguaje de modelado escogido ha sido \uml{} (\textit{Unified Modeling Language}) que se trata de un lenguaje unificado y muy extendido en el diseño de aplicaciones Orientadas a Objetos (OO).

Para la creación de los diagramas se utilizará \jude{}, se trata de una herramienta para el modelado de diagramas \uml{}. Al estar enfocado a la OO posibilita todo tipo de diagramas de una forma cómoda y rápida. Todos los diagramas creados a lo largo del proyecto se realizarán con dicho programa.

Para resolver algunos de los problemas comunes a los que habrá que enfrentarse se utilizarán diversos patrones de diseño \cite{patrones} estudiados que posibilitarán ofrecer una solución única estándar sobre problemas comunes que puedan surgir. La aplicación de patrones consigue diseños de calidad y, en consecuencia, mejores resultados en la fase de implementación.

También, trabajaremos con programación multihilo, buscando el poder aprovecharnos de las posibilidades que ofrecen los procesadores actuales, con más de un núcleo de proceso. Con esto, aumentaremos notablemente el rendimiento.

Para la memoria se va a utilizar \LaTeX{} \citeotras{definicion_latex}. Las ventajas que ofrece respecto a otros sistemas de composición de textos (como los clásicos \textit{WYSIWYG} \citeotras{wysiwyg}) son muchas, entre ellas nos permitirá la creación de una documentación uniforme, es decir, la salida que se obtenga será la misma con independencia del dispositivo o sistema operativo empleado para su visualización o impresión.

Dado que nunca se ha trabajado con \LaTeX{} ha sido necesaria la utilización de documentación \citeotras{cervantex}, manuales \citeotras{introduccion_latex} y \citeotras{latex_wikibook} que han facilitado el aprendizaje del mismo. Como plantilla para la memoria se utilizará la de la Universidad de Deusto \citeotras{plantilla_deusto} modificándola para adaptarla a la Universidad de Burgos.

Para trabajar con las radiografías se ha elegido la librería de ImageJ. Se trata de un programa de procesamiento de imagen digital desarrollado en Java que permite analizar y procesar imágenes, mediante la utilización de diversas técnicas como filtros e histogramas.

\section{Objetivos personales}
Además de los objetivos propios de la realización de la aplicación, también se pretende conseguir una serie de objetivos personales.
Nos gustaría poder poner en práctica todos los conocimientos teóricos adquiridos durante estos años de carrera. Además, con la realización de este proyecto queremos adquirir nuevos conocimientos en áreas tan diversas como la gestión de proyectos y el uso de metodologías ágiles, la inteligencia artificial y la visión por computador, la minería de datos, el uso de sistemas de control de versiones, etc.
Por último, queremos llevar a cabo con éxito la realización de este proyecto siendo capaces de planificarnos y trabajar en equipo.


%%%%%%%%%%%%%%%%%%%%%%%%%%%%%%%%%%%%%%%%%%%%%%%%%%%%%%%%%%%%%%%%%%
%%%%%%%%%%%%%%%%%%%%%%%%%%%%%%%%%%%%%%%%%%%%%%%%%%%%%%%%%%%%%%%%%%
\chapter{Conceptos teóricos}
%%%%%%%%%%%%%%%%%%%%%%%%%%%%%%%%%%%%%%%%%%%%%%%%%%%%%%%%%%%%%%%%%%
%%%%%%%%%%%%%%%%%%%%%%%%%%%%%%%%%%%%%%%%%%%%%%%%%%%%%%%%%%%%%%%%%%

En este apartado se van a describir, por una parte, conceptos básicos de la visión por computador, comenzado con nociones básicas del tratamiento digital de imágenes como la Adquisición (\ref{adquisicion}), el preprocesamiento (\ref{preprocesamiento}) y la segmentación (\ref{segmentacion}). A continuación, se describirán distintas técnicas de análisis de imágenes como la extracción de características (\ref{descriptores}) y el reconocimiento e interpretación (\ref{reconocimiento}). Por otra parte se hará una breve introducción a cómo se utilizan técnicas de visión artificial en los ensayos no destructivos y una descripción más pormenorizada de estás técnicas dentro de nuestro proyecto (\ref{aplicacion}).


\section{Descripción del problema: Ensayos no destructivos}
Los ensayos no destructivos \cite{wiki:EnsayoNoDestructivo} son pruebas que, practicadas sobre un material, no alteran de forma permanente sus propiedades físicas, químicas, mecánicas o dimensionales. Los ensayos no destructivos implican un daño imperceptible o nulo.

Los diferentes métodos de ensayos no destructivos se basan en la aplicación de fenómenos físicos tales como ondas electromagnéticas, acústicas, elásticas, emisión de partículas subatómicas, capilaridad, absorción y cualquier tipo de prueba que no implique un daño considerable a la muestra examinada.

Los datos aportados por este tipo de ensayos suelen ser menos exactos que los de los ensayos destructivos. Sin embargo, suelen ser más baratos, ya que no implican la destrucción de la pieza a examinar. En ocasiones los ensayos no destructivos buscan únicamente verificar la homogeneidad y continuidad del material analizado, por lo que se complementan con los datos provenientes de los ensayos destructivos.

Uno de los aspectos más importantes de cualquier método de ensayo no destructivo es que todo el personal debe estar entrenado, cualificado y certificado. El personal debe estar familiarizado con las técnicas, el equipamiento, el objeto a ensayar y cómo interpretar los resultados.

El objetivo de este proyecto es proporcionar una herramienta que automatice en la medida de lo posible (o que en última instancia, sirva de asistencia al operador humano) el proceso de ensayo radiográfico.

\subsection{Radiografía}
Una radiografía \cite{wiki:Radiografia} es una imagen registrada en una placa o película fotográfica, o de forma digital en una base de datos. La imagen se obtiene al exponer al receptor de imagen radiográfica a una fuente de radiación de alta energía, comúnmente rayos X o radiación gamma procedente de isótopos radiactivos. Al interponer un objeto entre la fuente de radiación y el receptor, las partes más densas aparecen con diferentes tonos dentro de una escala de grises, en función inversa a la densidad del objeto. Por ejemplo, si la radiación incide directamente sobre el receptor, se registra un tono negro.

Sus usos pueden ser tanto médicos, para detectar fisuras en huesos, como industriales en la detección de defectos en materiales y soldaduras, tales como grietas, poros, rebabas, etc.

La radiografía se usa para ensayar una variedad de productos, tales como objetos de fundición, objetos forjados y soldaduras. Es también muy usada en la industria aeroespacial para la detección de grietas (fisuras) en las estructuras de los aviones, la detección de agua en las estructuras tipo panal y detección de objetos extraños. Los objetos a ensayar se exponen a rayos X o gamma y se procesa un film o se visualiza digitalmente. El personal de ensayos radiográficos instala, expone y procesa la película o digitalmente procesa las señales e interpreta las imágenes de acuerdo con códigos.

\subsection{Ventajas del ensayo radiográfico}
Las ventajas del ensayo radiográfico \cite{wiki:Radiografia} son, entre otras, las siguientes:

\begin{enumerate}
\item Puede usarse con la mayoría de los materiales.
\item Puede usarse para proporcionar un registro visual permanente del objeto ensayado o un registro digital con la subsiguiente visualización en un monitor de computadora.
\item Puede revelar algunas discontinuidades dentro del material.
\item Revela errores de fabricación y a menudo indica la necesidad de acciones correctivas.
\end{enumerate}

\subsection{Limitaciones del ensayo radiográfico}
Las limitaciones de la radiografía \cite{wiki:Radiografia} incluyen consideraciones físicas y económicas.

\begin{enumerate}
\item Deben seguirse siempre los procedimientos de seguridad para las radiaciones.
\item La accesibilidad puede estar limitada. El operador debe tener acceso a ambos lados del objeto a ensayar.
\item Las discontinuidades que no son paralelas con el haz de radiación son difíciles de localizar.
\item La radiografía es un método caro de ensayo.
\item Es un método de ensayo que consume mucho tiempo. Después de tomar la radiografía, la película debe ser procesada, secada e interpretada (aunque este problema desaparece cuando la imagen de rayos X se registra digitalmente).
\item Algunas discontinuidades superficiales pueden ser difíciles, si no imposible, de detectar.
\end{enumerate}

\subsection{Objetivos del ensayo radiográfico}
El objetivo del ensayo radiográfico \cite{wiki:Radiografia} es asegurar la confiabilidad del producto. Esto puede lograrse sobre la base de los siguientes factores.

\begin{enumerate}
\item La radiografía permite al técnico ver la calidad interna del objeto ensayado o evidencia la
configuración interna de los componentes.
\item Revela la naturaleza del objeto ensayado sin perjudicar la utilidad del material.
\item Revela discontinuidades estructurales, fallas mecánicas y errores de montaje.
\end{enumerate}

La realización del ensayo radiográfico es sólo una parte del procedimiento. Los resultados del ensayo deben ser interpretados de acuerdo con normas de aceptación, y luego el objeto ensayado es aceptado o rechazado.

\subsection{Principios del ensayo radiográfico}
Los rayos X y gamma \cite{wiki:Radiografia} tienen la capacidad de penetrar los materiales incluso los materiales que no transmiten la luz. Al pasar a través de un material, algunos de esos rayos son absorbidos. La cantidad de radiación que se transmite a través de un objeto ensayado varía dependiendo del espesor y densidad del material y del tamaño de la fuente que se use. La radiación transmitida a través del objeto produce una imagen radiográfica. El objeto ensayado absorbe radiación, pero hay menos absorción donde el objeto es más fino o donde se presenta un vacío. Las porciones más gruesas del objeto o las inclusiones más densas se verán como imágenes más claras en la radiografía porque aumenta el espesor y la absorción es mayor.


\section{Visión por computador}

\subsection{Adquisición de la imagen}\label{adquisicion}
La primera etapa del proceso es la adquisición de la imagen. Para ello se necesitarán dos elementos:

\begin{itemize}

\item Un sensor de imágenes, es decir, un dispositivo físico sensible a una determinada banda del espectro de energía electromagnética (como las bandas de rayos X, ultravioleta, visible o
infrarrojo). En nuestro caso será un sistema de rayos X.

\item Un digitalizador, dispositivo que permitirá convertir la señal de salida del sensor a forma digital.

\end{itemize}

%Figura de Adquisicón de imágenes
\figura{1}{imgs/adquisicion_imagen}{Proceso de adquisición de imágenes de radiografía \cite{wiki:EnsayoNoDestructivo}}{img_rad}{}

En este proyecto no se está realizando la adquisición de la imagen, ya que trabajamos con imágenes cedidas por el Grupo Antolín.


\subsection{Preprocesamiento}\label{preprocesamiento}
El preprocesamiento de la imagen es una etapa que consiste en reducir la información de la
misma, de forma que pueda ser interpretada por una computadora, facilitando así la posterior fase de análisis.

Se utiliza un conjunto de técnicas que, aplicadas a las imágenes digitales, mejoran su calidad o facilitan la búsqueda de información. A partir de una imagen origen, se obtiene otra imagen final cuyo resultado sea más adecuado para una aplicación específica, optimizando ciertas características de la misma que hagan posible realizar operaciones de procesado sobre ella.

A continuación se explican algunas de las técnicas que comprenden esta etapa.


\subsubsection{Binarización}
La binarización de una imagen consiste en un proceso mediante el cual los valores de gris de una imagen quedan reducidos a dos (que podrían interpretarse como falso y verdadero). En una imagen digital, estos valores pueden representarse por los valores 0 y 1 o, más frecuentemente, por los colores negro (valor de gris 0) y blanco (valor de gris 255).

Para hacer esto, primero se debe convertir la imagen a escala de grises. Después hay que fijar
un valor umbral entre 0 y 255. Una vez que se tenga dicho umbral, se convertirán a 255 todos
los valores de la imagen superiores al umbral, mientras que los inferiores se convertirán a 0. El resultado será una imagen en blanco y negro que permitirá realizar tareas como la detección de contornos, separar regiones u objetos de interés del resto de la imagen, etc.

La binarización es, además, un método de segmentación. Hablaremos más sobre este tema en la sección \ref{segmentacion}.


\subsubsection{Saliency}\label{saliencymap}
El «Saliency Map» o «Mapa de Prominencia» \cite{Niebur:2007} es un mapa topográfico que permite representar
la prominencia visual de una determinada imagen.

Uno de los mayores problemas de la percepción es la sobrecarga de información. Se hace necesario identificar qué partes de la información disponible merecen ser seleccionadas para ser analizadas y qué partes deben descartarse. Este algoritmo busca solucionar este problema.

Koch y Ulman propusieron en 1985 \cite{koch1985shifts} que las diferentes características visuales que contribuyen a la selección de atención ante un estímulo (color, orientación, movimiento, etc.) fueran combinadas en un único mapa topográfico, el Saliency Map, que integraría la información normalizada de los mapas de características individuales en una medida global de visibilidad.

La saliencia de una posición dada es determinada principalmente por cómo de diferente es dicha
localización de las que la rodean, en color, orientación, movimiento, profundidad, etc.

La implementación del mapa de saliencia usada en este proyecto está basada en la variante
descrita en el artículo \emph{Human Detection Using a Mobile Platform and Novel Features Derived
From a Visual Saliency Mechanism} \cite{montabone2010human}.

%Figura de Saliency Map
\figura{1}{imgs/saliency.png}{Ejemplo de Saliency Map. La imagen original a la izquierda,
con el correspondiente saliency map a la derecha}{saliency}{}

\subsection{Segmentación}\label{segmentacion}
Además de calcular ciertas características de las imágenes, como hemos visto en el apartado anterior, se hace necesaria realizar una segmentación de la imagen, para aumentar la precisión de la detección de los defectos.

La segmentación de una imagen \cite{wiki:segmentation} consiste en particionar esa imagen en múltiples segmentos (conjuntos de píxeles). El objetivo es simplificar y/o cambiar la representación de una imagen en algo que sea más significativo o fácil de analizar. Se suele usar típicamente para localizar objetos y bordes. Más precisamente, la segmentación de una imagen es el proceso de asignar una etiqueta a cada píxel de una imagen, con lo que los píxeles que tengan la misma etiqueta compartirán ciertas características.

De entre todos los posibles métodos de segmentación que existen, nosotros hemos usado el denominado \textit{Thresholding} \cite{wiki:thresholding}. Es uno de los métodos más simples. Está basado en considerar un valor, llamado \textbf{umbral}, que se usa para convertir una imagen en escala de grises a una binaria (es decir, binarizar la imagen). La clave está, por tanto, en el valor umbral. De acuerdo a la taxonomía definida por Sezgin and Sankur \cite{sezgin}, existen varias opciones:

\begin{itemize}
\item Basados en la forma del histograma.
\item Basados en clustering.
\item Basados en entropía.
\item Basados en atributos de objetos.
\item Métodos espaciales.
\item Métodos locales.
\end{itemize}

De todos estos, nosotros hemos usado los locales. Estos métodos se basan en adaptar el valor umbral en cada píxel, de acuerdo a las características del vecindario de ese píxel. El radio del vecindario es un parámetro que afecta al funcionamiento del método.

Hay varios de estos métodos. A continuación, vamos a describir los que hemos probado en este proyecto.

\subsubsection{MidGrey}
Este método selecciona el umbral de acuerdo a la media del máximo y mínimo valor de la distribución local de escala de grises.

Por lo tanto, el umbral viene dado por la siguiente fórmula \cite{autolocal}:

\[T = \left(\frac{\max+\min}{2}\right)-c\]

Donde $c$ es una constante que sirve para afinar el método. Por defecto, es cero.

Para determinar a qué región pertenece el píxel, se comprueba con el umbral. Si es mayor que éste último, el píxel pertenece al objeto. Si no, pertenece al fondo.

\subsubsection{Mean}
En este caso, el umbral es la media de la distribución local en escala de grises \cite{autolocal}. Por lo tanto, para determinar a qué región pertenece el píxel, se compara con la media de la región de vecinos (menos el parámetro $c$, en caso de que se especifique). Si es mayor, es parte del objeto. Si no, es parte del fondo.


\subsection{Descriptores de regiones}\label{descriptores}
Una vez realizada la etapa de preprocesamiento, la imagen ya estará lista para ser analizada.
Nosotros vamos a trabajar directamente con los píxeles de la imagen, a diferencia de otros métodos de análisis, que extraen otros tipos de atributos. Para analizar las características de la imagen, se utilizarán los descriptores que se explican a continuación:

\begin{itemize}
\item Descriptores simples.
\item Descriptores de textura.
\item Descriptores geométricos.
\end{itemize}

\subsubsection{Descriptores simples}
También se les conoce como características estándar o de primer orden \cite{presutti2004matriz}. Son medidas que se calculan a partir de los valores de gris originales de la imagen y su frecuencia, como la media, varianza, desviación estándar, etc. En estas medidas no se considera la relación de co-ocurrencia entre los píxeles.

Las características más comunes y que se han usado en este proyecto son:

\paragraph*{Media}\mbox{} \\
\indent Se calcula el promedio de los niveles de intensidad de todos los píxeles de la imagen. Esta es una medida útil ya que nos permite determinar de forma sencilla la claridad de la imagen. Si la media es alta, la imagen será más clara, mientras que si la media es baja, será más oscura.

\paragraph*{Desviación estándar}\mbox{} \\
\indent Es una medida de dispersión que nos indica cuánto se alejan los valores respecto a la media. Nos sirve para apreciar el grado de variabilidad entre los valores de intensidad de los píxeles de una región.

\paragraph*{Primera y segunda derivadas}\mbox{} \\
\indent Se utilizan operadores de detección de bordes \cite{marcosmartin2004} basados en aproximaciones de la primera y segunda derivada de los niveles de grises de la imagen. La primera derivada del perfil de gris será positiva en el borde de entrada de la transición entre una zona clara y otra oscura. En el borde de salida será negativa, mientras que en las zonas de nivel de gris constante será cero. El módulo de la primera derivada podrá utilizarse, por lo tanto, para detectar la presencia de un borde en una imagen. En cuanto a la segunda derivada, será positiva en la parte de la transición asociada con el lado oscuro del borde, negativa en la parte de la transición asociada con el lado claro y cero en las zonas de nivel de gris constante. El signo de la segunda derivada nos permitirá determinar si un píxel perteneciente a un borde está situado en el lado oscuro o claro del mismo.


\subsubsection{Descriptores de textura}
La texturas son propiedades asociadas a las superficies, como rugosidad, suavizado, granularidad, regularidad. En el campo de las imágenes, significa la repetición espacial de ciertos patrones sobre una superficie.

Otra definición de la textura podría ser la variación entre píxeles en una pequeña vecindad de una imagen. Alternativamente, la textura puede describirse también como un atributo que representa la distribución espacial de los niveles de intensidad en una región dada de una imagen digital.


El análisis de la textura de las imágenes nos ofrecerá datos útiles para nuestro trabajo. Hemos utilizado los siguientes descriptores:

\paragraph*{Características de Haralick}\mbox{} \\
\indent Siguiendo la propuesta de Haralick \cite{haralick1992computer}, se extrae información de textura de la distribución de los valores de intensidad de los píxeles. Dichos valores se calculan utilizando matrices de coocurrencia que representan información de textura de segundo orden.

Haralick propuso un conjunto de 14 medidas de textura basada en la dependencia espacial de los tonos de grises. Esas dependencias están especificadas en la matriz de co-ocurrencia espacial (o de niveles de gris). Se sigue la descripción de \cite{presutti2004matriz} para calcular dicha matriz.

La matriz de co-ocurrencia, una vez normalizada, tiene las siguientes propiedades:

\begin{itemize}

\item Es cuadrada.

\item Tiene el mismo número de filas y columnas que el número de bits de la imagen. Con una imagen de 8 bits ($2^{8} = 256$ posibles valores) la matriz de co-ocurrencia será de 256$x$256, es decir, 65536 celdas.

\item Es simétrica con respecto a la diagonal.

\item Los elementos de la diagonal representan pares de píxeles que no tienen diferencias en su nivel de gris. Si estos elementos tienen probabilidades grandes, entonces la imagen no muestra mucho contraste, ya que la mayoría de los píxeles son idénticos a sus vecinos.

\item Sumando los valores de la diagonal tenemos la probabilidad de que un píxel tenga el mismo nivel de gris que su vecino.

\item Las líneas paralelas a la diagonal separadas por una celda, representan los pares de píxeles con una diferencia de un nivel de gris. De la misma manera sumando los elementos separados dos celdas de la diagonal, tenemos los pares de píxeles con dos valores de grises de diferencia. A medida que nos alejamos de la diagonal la diferencia entre niveles de grises es mayor.

\item Sumando los valores de estas diagonales secundarias (y paralelas a la diagonal principal) obtenemos la probabilidad de que un píxel tenga 1, 2, 3, etc niveles de grises de diferencia con su vecino.

\end{itemize}

Una vez construida la matriz de co-ocurrencia, de ella pueden derivarse diferentes medidas. Se obtendrán matrices para las direcciones 0º, 90º, 180º y 270º y para distancias 1, 2, 3, 4 y 5. Para cada una de estas distancias se calculará un vector con las medias de las cuatro direcciones y otro con los rangos. Las características a calcular a partir de la matriz son las siguientes:

\begin{enumerate}

\item \textbf{Segundo Momento Angular}

Mide la homogeneidad local. Cuanto más suave es la textura, mayor valor toma. Si la matriz de co-ocurrencia tiene pocas entradas de gran magnitud, toma valores altos. Es baja cuando todas las entradas son similares \cite{waveletdiscreta}.

\[f_1 = \sum_{i=1}^{N_g}\sum_{j=1}^{N_g} p(i,j)^2\]

$p(i,j)$ es el valor de la matriz de coocurrencia en la fila $i$ y la columna $j$
$N_g$ es la dimensión de la matriz

\item \textbf{Contraste}

Es lo opuesto a la homogeneidad, es decir, es una medida de la variación local en una imagen. Tiene un valor alto cuando la región tiene un alto contraste.

\[f_2 =\sum_{n=0}^{N_g-1}n^2 \sum_{i=1}^{N_g}\sum_{j=1}^{N_g} p(i,j)\]

\item \textbf{Correlación}

Mide las dependencias lineales de los niveles de grises, la similitud entre píxeles vecinos. Un objeto tiene mayor correlación dentro de él que con los objetos adyacentes. Píxeles cercanos están más correlacionados entre sí que los píxeles más distantes.

\[f_3 = \frac{\sum_i \sum_j (i,j)\cdot p(i,j) - v_xv_y}{\sigma_x\sigma_y}\]

Donde $v_x, v_y, \sigma_x, \sigma_y$ son las medias y desviaciones estándar de $p_x$ y $p_y$, las funciones de densidad de probabilidad parcial.

\item \textbf{Suma de cuadrados}

Es la medida del contraste del nivel de gris.

\[f_4 = \sum_{i=1}^{N_g}\sum_{j=1}^{N_g}(i-j)^2\cdot p(i,j)\]

\item \textbf{Momento Diferencial Inverso}

También llamado homogeneidad, es más alto cuando la matriz de co-ocurrencia se concentra a lo largo de la diagonal. Esto ocurre cuando la imagen es localmente homogénea de acuerdo al tamaño de la ventana.

\[f_5 = \sum_{i}\sum_{j}\frac{1}{1+(i-j)^2}\cdot p(i,j)\]

\item \textbf{Suma promedio}

\[f_6 = \sum_{i=2}^{2N_g}i \cdot p_{x+y}(i)\]

\item \textbf{Suma de Entropías}

\[f_7 = \sum_{i=2}^{2N_g}(i-f_8)^2 \cdot p_{x+y}(i)\]

\item \textbf{Suma de Varianzas}

\[f_8 = -\sum_{i=2}^{2N_g}p_{x+y}(i)\log(p_{x+y}(i))\]

\item \textbf{Entropía}

Es alta cuando los elementos de la matriz de co-ocurrencia tienen valores relativamente iguales. Es baja cuando los elementos son cercanos a 0 ó 1.

\[f_9 = -\sum_{i=1}^{N_g}\sum_{j=1}^{N_g}p(i,j)\log(p(i,j))\]

\item \textbf{Diferencia de Varianzas}

\[f_{10} = \sum_{i=0}^{N_g-1} i^2 p_{x-y}(i)\]

\item \textbf{Diferencia de Entropías}

\[f_{11} = -\sum_{i=0}^{N_g-1}p_{x-y}(i)\log(p_{x-y}(i))\]

\item \textbf{Medidas de Información de Correlación 1}

\[f_{12} = \frac{HXY-HXY1}{\max(HX,HY)}\]

Donde:

\[HXY = -\sum_{i=1}^{N_g}\sum_{j=1}^{N_g}p(i,j)\log \big(p(i,j)\big)\]
\[HXY1 = -\sum_{i=1}^{N_g}\sum_{j=1}^{N_g}p(i,j)\log \big(p_x(i) p_y(j)\big)\]
\[HXY2 = -\sum_{i=1}^{N_g}\sum_{j=1}^{N_g}p_x(i) p_y(j)\log \big(p_x(i) p_y(j)\big)\]

\item \textbf{Medidas de Información de Correlación 2}

\[f_{13} = \big(1- \exp(-2 |HXY2 - HXY|)\big)^{1/2}\]

\item \textbf{Coeficiente de Correlación Máxima}

\[f_{14} = \sqrt{\lambda_2 }\]

Donde $\lambda_2$ es el segundo valor propio de la matriz Q definida como:

\[Q(i,j) = \sum_k\frac{p(i,k)p(j,k)}{p_x(i)p_y(i)}\]
\end{enumerate}


\paragraph*{Local Binary Patterns}\mbox{} \\
\indent Los Local Binary Patterns (LBP) \cite{wiki:LocalBinaryPatterns} son un tipo de característica usado para la clasificación de texturas. Fueron descritos por primera vez en 1994 \cite{ojala}.

Debido a su poder de discriminación y su simplicidad de cálculo, se ha convertido en un método popular que se usa en varios tipos de aplicaciones \cite{de2011transformaciones}.

Este operador de textura etiqueta los píxeles de una imagen comparando los valores de intensidad de los píxeles de una vecindad de 3x3 con el del píxel central.

Cuando el valor del píxel vecino es mayor que el del píxel central, se escribe «1». En caso contrario, se escribe «0». El resultado es un número binario de 8 dígitos, que suele convertirse a decimal por comodidad o para mayor facilidad de cálculo. Este número recibe también el nombre de «patrón».

Luego se realiza un histograma que contendrá la frecuencia con la que se ha producido cada patrón. Dicho histograma podrá utilizarse como descriptor de textura.

Posteriormente se ha extendido el uso de diferentes tamaños, no sólo a ocho puntos, sino a muestreos circulares donde la bilinealidad se consigue con la interpolación de los valores de los píxeles, lo que permite utilizar cualquier radio y por lo tanto cualquier número de píxeles vecinos.

Para reducir la longitud del vector de características se utilizan los patrones uniformes. Un local binary pattern es uniforme si el patrón contiene un máximo de dos transiciones a nivel de bit, de «0» a «1» o viceversa.

\figuraConPosicion{1}{imgs/lbp.png}{Ejemplo del funcionamiento de los Local Binary Patterns \cite{de2011transformaciones}}{lbp}{}{H}

Por ejemplo, los patrones 00000000 (0 transiciones), 01110000 (2 transiciones) y 11001111 (2 transiciones) son uniformes mientras que los patrones 11001001 (4 transiciones) y 01010010 (6 transiciones) no lo son. El número de transiciones se guarda en un valor llamado medida de uniformidad \textit{U}.

En el cálculo de los LBP, se utiliza una etiqueta para cada uno de los patrones uniformes, mientras que todos los patrones no uniformes son agrupados en una sola etiqueta. Por ejemplo, cuando se usa una vecindad (8,$R$) (donde $R$ es el radio), hay un total de 256 patrones, 58 de los cuales son uniformes, lo cual produce un total de 59 etiquetas diferentes. Todo esto dará como resultado un histograma con 59 intervalos.
\newpage

\figura{1}{imgs/vecindades_lbp.png}{Vecindades de distinto tamaño en los LBP \cite{de2011transformaciones}}{vecindades_lbp}{}





\subsubsection{Características geométricas}
Además de los descriptores de regiones que ya hemos visto, pensamos en la posibilidad de realizar algunos cálculos de características geométricas sobre las regiones segmentadas, con la mirada puesta en poder clasificar mediante ellas a los defectos en distintos tipos.

Las características geométricas que hemos calculado son \cite{analyzeij}:

\begin{itemize}

\item \textbf{Área:} es la superficie de la región de interés medida en píxeles cuadrados.

\item \textbf{Perímetro:} es la longitud del límite exterior de la región.

\item \textbf{Circularidad:} descriptor de forma dado por la fórmula $\frac{4\pi \times \acute{a}rea}{per\acute{i}metro^2}$. Un valor de 1 indica que se trata de un círculo perfecto. Según se acerca a cero, indica que la forma es cada vez más alargada.

\item \textbf{Redondez:} descriptor de forma dado por la fórmula $\frac{4\pi\times \acute{a}rea}{\pi \times semieje\_mayor^2}$. Es el inverso del cociente entre el semieje mayor y el menor de la mejor elipse que puede ser dibujada en la región.

\item \textbf{Semieje mayor:} semieje mayor de la mayor elipse que puede ser dibujada en la región.

\item \textbf{Semieje menor:} semieje menor de la mayor elipse que puede ser dibujada en la región.

\item \textbf{Ángulo:} es el ángulo entre el semieje mayor y una línea paralela al eje $x$.

\item \textbf{Distancia Feret:} también llamada diámetro de Feret. Es la máxima distancia entre dos puntos cualesquiera de la región.

\end{itemize}




\subsection{Reconocimiento e interpretación de imágenes}\label{reconocimiento}
Una vez obtenidas las características de la imagen, el siguiente paso será reconocer dichos datos e interpretarlos. Para ello será necesario entrenar un clasificador. Una vez entrenado, podrá predecir dónde estarán los defectos que buscamos.

Un clasificador \cite{wiki:Clasificadores} es un elemento que, tomando un conjunto de características como entrada, proporciona a la salida una etiqueta de clase. En nuestro caso, el atributo de clase sería el «defecto», que podría tomar dos valores: «verdadero» o «falso». En el caso de que sea una regresión, en vez de una clasificación de clases nominales, los valores que devolverá el clasificador serán 0 y 1.

Utilizaremos el clasificador por su capacidad de aprender a partir de imágenes de ejemplo y de generalizar este conocimiento para que se pueda aplicar a nuevas y diferentes imágenes.

Para construir el clasificador utilizaremos un conjunto de imágenes etiquetadas. Para etiquetarlas, se creará una máscara de cada imagen en la que se marcarán a mano los defectos. Estas máscaras permitirán al clasificador saber qué partes de la imagen son defectos y cuales no.  Es lo que se denomina \emph{ground truth} en la literatura de procesamiento digital de imágenes.

Los clasificadores que hemos utilizado no pueden trabajar directamente con imágenes, sino con vectores de características, que serán las que se calculen a partir de las imágenes de ejemplo. Estos vectores de características se guardarán en ficheros \textit{ARFF}, que tendrán una serie de atributos definidos en la cabecera, cada uno de ellos correspondiente a una característica. El fichero contendrá un conjunto de instancias, que son cada serie de valores que toman los atributos. Los ficheros \textit{ARFF} son el formato propio de \textit{Weka}, y en su estructura se pueden diferenciar las siguientes secciones:

\begin{itemize}
\item \textbf{@relation:} Los ficheros \textit{ARFF} deben empezar con esta declaración en su primera línea (no puede haber líneas en blanco antes). Será una cadena de caracteres.

\item \textbf{@attribute:} En esta sección hay que poner una línea por cada atributo que vaya a tener el conjunto de datos. Para cada atributo habrá que indicar su nombre y el tipo de dato. El tipo puede ser \emph{numeric}, \emph{string}, etc.

\item \textbf{@data:} En esta sección se incluyen los datos. Cada columna se separa por comas y todas las
filas deberán tener el mismo número de columnas, que coincidirá con el número de atributos declarados.
\end{itemize}

Al entrenar al clasificador obtendremos un modelo, el cual usaremos cuando queramos detectar los defectos de una nueva imagen.

Un clasificador puede ser, según los tipos de aprendizaje:

\begin{description}
\item[Supervisado:] cuando se utilizan ejemplos previamente etiquetados.
\item[No supervisado:] cuando se utilizan patrones de entradas para los no se especifican los valores
de sus salidas.
\end{description}

Por lo dicho anteriormente, hemos utilizado clasificadores supervisados. Nos hemos basado en el estudio con varios clasificadores que hicieron los alumnos que desarrollaron la versión de la aplicación que este proyecto esta mejorando y ampliando, así que hemos usado el que ellos consideraron mejor: \textit{Bagging}.

Como algoritmo base hemos usado \textit{REPTree}, que, por sus características, también lo hemos podido usar para la regresión.

\subsubsection{REPTree}
El algoritmo de \textit{REPTree} \cite{ian_h._witten_data_2005} permite construir tanto un árbol de clasificación como un árbol de regresión, usando las medidas de \textit{ganancia de información} y de \textit{reducción de la varianza}. Puede podar los árboles generados usando la \textit{poda de error reducido}. Además, está optimizado para la rápida ejecución. Sólo ordena valores para atributos numéricos una única vez.

La ganancia de información es una reducción de la entropía esperada (es decir, de la impureza de un conjunto de ejemplos), causada por la partición de los ejemplos de acuerdo al atributo \cite{juan_jose_rodriguez_diez_apuntes_2012}. La reducción de la varianza es una técnica que se usa para aumentar la precisión, disminuyendo el error que se produce cuando se entrena con diferentes conjuntos de datos.

Considera los valores desconocidos partiendo instancias en pedazos, como hace \textit{C4.5}. Se puede especificar el mínimo número de instancias por hoja, profundidad máxima del árbol (útil cuando se utiliza la técnica de \textit{boosting}), la mínima proporción de la varianza de los datos del conjunto de entrenamiento que se consideran para partir (sólo para clases numéricas), y el número de pliegues por poda \cite{ian_h._witten_data_2005}.

\subsubsection{Bagging}
\textit{Bagging} (de \textit{Bootstrap Aggregating})\cite{juan_jose_rodriguez_diez_apuntes_2012} es un algoritmo que busca combinar predicciones de un cierto tipo de modelo (en nuestro caso, \textit{REPTree}) mediante votación o media, teniendo los modelos el mismo peso.

Lo que se hace es, teniendo un conjunto de datos, generar a partir de él un cierto número de conjuntos de datos mediante muestreo con reemplazamiento, para entrenar con ellos un cierto número de modelos. Después, se combinan las predicciones de todos los clasificadores entrenados para obtener la predicción final.

La combinación de estas predicciones se puede hacer con una media de las salidas (regresión) o con una votación de mayoría (clasificación).

Con esto conseguimos que métodos inestables, como árboles de decisión, mejoren su rendimiento.



\section{Aplicación de la visión artificial a la detección y segmentación de defectos}\label{aplicacion}


\subsection{Fundamentos teóricos de la versión inicial}
En este apartado se describe la primera versión que realizamos de la detección de defectos. Esta primera versión es una versión mejorada de la versión final de los alumnos del año pasado.

Se ha continuado la misma idea que utilizaron los desarrolladores de la versión inicial, basada en el artículo «\emph{Automated Detection of Welding Defects without Segmentation}» de Domingo Mery \cite{DomingoMery}. Los cambios realizadas en esta versión inicial con respecto a la versión de la que se partía han sido:
\begin{itemize}
\item Se ha realizado un nuevo diseño artquitectónico.
\item Se ha cambiado parte del código de la interfaz.
\item Se ha cambiado el proceso de entrenamiento.
\item Ahora se utiliza un enfoque multihilo.
\end{itemize}

\subsubsection{Preproceso}
Para la realización del análisis de la imagen y su posterior detección de defectos, hay que partir una imagen original en escala de grises. A dicha imagen se la pasa un filtro \textit{«Saliency Map»} (ver \ref{saliency}). Con todo esto, se obtiene la imagen original y la imagen Saliency que son analizadas paralelamente.

\figuraConPosicion{1}{imgs/extraccion.png}{Esquema del proceso de extracción de características \cite{DomingoMery}}{extraccion}{}{H}

\subsubsection{Extracción de características y etiquetado de instancias}
Antes de detectar cualquier tipo de defecto, es necesario entrenar un clasificador para que pueda predecir donde están los defectos buscados. Para ello se analizan un conjunto de imágenes de las que se crean máscaras en las que se pintan los defectos. Estas máscaras sirven para que el clasificador sepa qué partes de la imagen son defectos y cuáles no. En la imagen \ver{mascara} vemos un ejemplo de cómo son estas máscaras (también conocidas como \textit{Ground Touch}).

\figuraConPosicion{1}{imgs/mascara.png}{Ejemplo de máscara junto a imagen original}{mascara}{}{H}

Para analizar cada imagen se utiliza una ventana que recorre toda la región de interés analizando sus características. En el artículo original de Mery \cite{DomingoMery} sólo aparecía el número total de características, lo cual dificultó la identificación del número que había que calcular para cada tipo, por eso nosotros las hemos desglosado según su clase:

\begin{itemize}
\item \textbf{Características estándar:} 4 características.
\item \textbf{Características de Haralick:} Se calculan 14 características, obteniendo 5 vectores de medias
y 5 de rangos. $14 \times 10 = 140$ características.
\item \textbf{Características LBP:} Se obtiene un histograma con 59 intervalos, es decir, 59 características.
\end{itemize}

El número final de características habrá que multiplicarlo por dos, ya que todas las características se calculan tanto para la imagen original como para la imagen con \emph{saliency map} aplicado. Por lo tanto, el total sería:

\begin{center}
( 4 estándar + 140 haralick + 59 lbp )$\times$2 = 406 CARACTERÍSTICAS
\end{center}

Estas características serán las que se guarden en los ficheros \textit{ARFF} que se utilicen para entrenar
al clasificador, junto con la clase, que tendrá valor «true» si la instancia tiene defecto, o «false» si
no. En caso de que se use regresión lineal serán 0 y 1. El defecto se identificará gracias a las máscaras mencionadas anteriormente.

La ventana irá analizando cada vez una región de la imagen, recorriendo todos y cada uno de sus píxeles, de los que extraerá las características que posteriormente se analizarán. Cada vez que la ventana se mueva y analice una nueva región, se creará una nueva instancia, que se corresponderá con una línea del fichero \textit{ARFF} en el que se guardan las características. Nosotros hemos hecho que el tamaño de la ventana sea configurable, aunque por defecto se utilizará una ventana de 24$\times$24 píxeles, como en el artículo de Mery \cite{DomingoMery}. Utilizamos dos tipos de estrategias de desplazamiento:

\begin{itemize}
\item Ventana deslizante.
\item Ventana aleatoria.
\end{itemize}

\paragraph*{Ventana deslizante}\mbox{} \\
\indent La ventana comienza desde la esquina superior izquierda de la imagen y se va moviendo cada cierto porcentaje del tamaño de la ventana. Cuando llega al extremo derecho baja ese mismo porcentaje y comienza de nuevo desde el lado izquierdo. A diferencia del artículo, donde se usa siempre un salto de 4 píxeles, nosotros hemos hecho que se pueda seleccionar el tamaño del salto.

\figuraConPosicion{1}{imgs/ventana_deslizante.png}{Esquema de la ventana deslizante \cite{DomingoMery}}{ventana_deslizante}{}{H}

\paragraph*{Ventana aleatoria}\mbox{} \\
\indent Se obtienen 300 muestras de ventanas por imagen, seleccionándolas aleatoriamente entre aquellas que tienen defecto y las que no.

\subsubsection{Estrategias de etiquetado}
Como hemos visto, se hace necesario determinar cuándo una ventana tiene defecto o no, ya que se necesita etiquetar las muestras de cada ventana como «defecto» o «no defecto». Los alumnos que desarrollaron la versión previa utilizaron una aproximación muy simple: consideraban que una ventana era defectuosa cuando, al poner esa ventana sobre la máscara coloreada manualmente, al menos un píxel de la misma estaba coloreado. Comprobamos que esto provoca falsos positivos, con lo que se pierde exactitud. Por lo tanto, decidimos implementar otras posibilidades:

\paragraph*{Píxel central}\mbox{} \\
\indent En este caso, se considera una ventana defectuosa cuando el píxel central de la misma es defectuoso. Es un poco más preciso que la versión del año pasado, pero sigue sin ser demasiado buena.

\paragraph*{Porcentaje de la ventana}\mbox{} \\
\indent En esta aproximación, se considera defectuosa una ventana cuando al menos un cierto porcentaje de la ventana contiene píxeles coloreados, es decir, defectuosos. Hemos obtenido resultados muy buenos con porcentajes que van desde el 50\% al 75\%.

\paragraph*{Píxel central más región de vecinos}\mbox{} \\
\indent En este caso, no sólo consideramos el píxel central, si no que creamos una región cuadrada de $3 \times 3$ píxeles a su alrededor. Consideramos entonces una ventana como defectuosa cuando un cierto porcentaje de esta región de vecinos está coloreada. Obtenemos unos resultados muy parecidos a los de la anterior aproximación con porcentajes parecidos.

Se realizó un experimento con todas estas opciones para determinar cuáles eran las que mejor funcionaba. En \ver{experimentos}, se pueden ver algunos de los resultados que se obtuvieron, a la vista de los cuales decidimos quedarnos con las opciones de porcentaje de la ventana y píxel centrar más región de vecinos.

\figuraConPosicion{0.5}{imgs/experimentos.png}{Experimento estrategias de etiquetado
\newline
Píxel central \newline
Porcentaje de ventana \newline
Región de vecinos}{experimentos}{}{H}


\subsubsection{Detección de defectos}
El proceso es prácticamente igual al entrenamiento del clasificador. Se genera una imagen \emph{Saliency Map} a partir de la imagen original, al igual que antes, y a partir de las dos imágenes (la original y la \emph{Saliency}) se analizan mediante una ventana deslizante.

Cada vez que la ventana se desplaza se genera una instancia que contiene todas las características analizadas. Esta instancia es utilizada por el modelo que predice si la ventana está situada sobre un defecto.

Si la ventana contiene un defecto, es marcada con un cuadrado verde y a cada uno de sus píxeles se le suma una unidad (este valor de los píxeles desde ahora será llamado factor).

Estos píxeles son almacenados en una matriz global del mismo tamaño que la imagen a analizar, por lo que según se desplaza la ventana deslizante se van actualizando todos los valores de los píxeles que contienen defectos.

Una vez finalizado el proceso de detección, se binariza la matriz resultante según su factor. Para este proceso en el artículo original \cite{DomingoMery} se utiliza un umbral de 24.

Este factor es descrito como el número de veces que ha sido marcado un píxel como defecto. Si un píxel tiene un factor menor que 24, dicho pixel se considera como no defecto. En cambio, si un píxel tiene un factor de 24 o más será considerado como defecto.

Dicho esto, en el proyecto utilizamos un factor variable. Por defecto el factor es de 8, pero tras la ejecución se puede variar para observar los cambios en la imagen en tiempo real.

Como ya se ha descrito, el proceso de binarización recae en el factor de un píxel. Si el factor es 8 o mayor, al pixel se le asigna un 1 (defecto), si es menor que 8 se le asigna un 0 (no defecto). Al tener una matriz de ceros y unos es sencillo obtener una imagen binaria con la región de defectos.

\figuraConPosicion{0.7}{imgs/dibujado_defectos.png}{Proceso de dibujado de defectos en nuestro proyecto. \newline
a)Imagen con defectos marcados \newline
b)Imagen binarizada con el área de defectos \newline
c)Imagen con filtro de detección de bordes\newline
d)Resultado final}{procesoBorde}{}{h}

Una vez obtenida la imagen binarizada con la región de defectos se le aplica un filtro de detección de bordes para marcar una línea que rodee el defecto mostrando su ubicación. En este caso se ha utilizado el filtro por defecto de detección de bordes de ImageJ.

Observamos que el proceso de detección de bordes invierte los colores, la línea queda en blanco y el fondo en negro, así que invertimos los colores utilizando uno de los filtros de ImageJ.

Hecho esto, solo queda poner el fondo transparente y superponer la imagen del borde sobre la imagen original.

\subsubsection{Multihilo}
Como ya hemos dicho varias veces, uno de los objetivos del proyecto es mejorar el rendimiento de la versión previa. Una de las principales modificaciones para conseguir esto ha sido la inclusión de una estrategia multihilo, aprovechando las oportunidades que nos brindan los procesadores actuales.

Lo que hemos hecho ha sido dividir la imagen en tantas partes como procesadores disponibles tenga la máquina que está ejecutando el programa. Por ejemplo, si tenemos 2 procesadores, la imagen se dividirá en 2, en su dimensión vertical. Hay que tener en cuenta un pequeño margen, necesario para que el cálculo de algunas características sea correcto. Por ello, en el ejemplo anterior, las 2 imágenes no serían exactamente de la mitad de altura que la original, si no que serían un poco más grandes. Hay un pequeño solapamiento.

Una vez que tenemos dividida la imagen, se ejecuta el proceso de detección o de entrenamiento sobre cada uno de los trozos de imagen. Con esto obtenemos un rendimiento mucho mayor que la versión del año pasado.


\subsection{Fundamentos teóricos de la versión final}
Con las modificaciones descritas en el apartado anterior, ya observamos que tanto el rendimiento como la precisión habían aumentado con respecto al año pasado. Aún así, teniendo en mente la posibilidad de calcular otra serie de características (como las geométricas) sobre los defectos detectados para, en un futuro, poder clasificarlos en sus diferentes tipos, se decidió intentar mejorar aún más esta precisión.

Se implementaron dos nuevas aproximaciones a la hora de detectar defectos (por lo tanto, el proceso de entrenamiento cambia). En ambas aproximaciones nos valemos de una imagen segmentada con los filtros de umbrales locales que ya hemos visto. Lo que cambia entre ellas es cómo consideramos estos filtros.

\subsubsection{Primera innovación: detección normal con posterior intersección con umbrales locales}
En esta primera aproximación, primero se realiza la detección de defectos como en la primera versión del proyecto. La diferencia está en el dibujado definitivo de los defectos.

Con la detección de defectos normal, obtenemos una matriz del mismo tamaño que la imagen, a partir de la cual se dibujan los defectos. En esta matriz se guarda, en cada posición (es decir, en cada píxel), el número de ventanas que los han cubierto y que se han marcado como defecto. Si este número supera cierto umbral, habrá una nueva matriz, en la que habrá un uno en todos aquellos píxeles en los que la probabilidad de ser defecto es mayor. A partir de esta matriz, se crea el dibujado definitivo.

En nuestra aproximación, introducimos un paso intermedio antes de generar la matriz definitiva. En vez de poner directamente un uno o un cero, lo que hacemos es segmentar la imagen mediante el filtro de umbrales locales. A continuación, los píxeles que superan el umbral seleccionado son comparados con la imagen de umbrales locales. Si en esta imagen aparecen como blancos, se consideran defecto. Si no, se descarta.

\subsubsection{Segunda innovación: píxeles blancos en umbrales locales}
Esta aproximación difiere bastante más respecto a las opciones ya vistas.

En este caso, lo primero que se hace es generar la imagen de umbrales locales. A partir de esta imagen, generamos una lista con las coordenadas de los píxeles que se han marcado como blancos.

Después, binarizamos la imagen de umbrales locales para aplicar un análisis de partículas sobre ella, con el objetivo de obtener todas las regiones que destacan, a las que consideramos regiones candidatas de albergar un defecto. Este análisis de partículas se hace con las clases de \emph{ImageJ}, que va a aplicar ciertas medidas a los objetos que vaya encontrando en una imagen, buscando sus bordes \cite{particleij}.

Cuando tenemos la lista y las regiones, vamos sacando coordenadas de la lista y vamos determinando la región a la que pertenecen. Dependiendo del tamaño de la región y el tamaño de la ventana, se considera la coordenada o se desecha. En caso de que se considere, centramos una ventana sobre ella y aplicamos el cálculo de características y clasificación ya vistos anteriormente.

\figuraConPosicion{0.85}{imgs/segunda_opcion.png}{Proceso de detección en la segunda opción \newline
a)Imagen original \newline
b)Imagen segmentada con los umbrales locales \newline
c)Binarización de los umbrales locales\newline
d)Identificación de las regiones candidatas}{segunda_opcion}{}{h}

El dibujado de los defectos se realiza con la misma matriz que en la primera versión del proyecto.

La parte de analizar las regiones para determinar si hay que considerar la coordenada o no fue añadida después de una primera versión de esta opción, en la que se consideraba toda la lista de píxeles blancos. Se decidió cambiar porque si considerábamos toda la lista obteníamos muchísimos falsos positivos, con lo que la precisión no era buena.

Para aumentar el rendimiento, antes de empezar a calcular características, se divide la lista de píxeles blancos en tantas partes como procesadores disponibles haya. Después, habrá tantos hilos como partes, en los que en cada uno de ellos se iterará sobre cada una de las partes, de forma paralela.

\subsubsection{Cálculo de características geométricas}
Una vez que se ha realizado el proceso de detección y dibujado de defectos, podemos calcular los descriptores geométricos de estos defectos. Para ello, se usa la imagen binarizada con la región de defectos para aplicar sobre ella un proceso de segmentación, a través del cual vamos a poder obtener una serie de regiones, sobre las cuales se puede aplicar el cálculo de las características geométricas ya mencionadas. Con estos resultados creamos una tabla que se irá refrescando si el usuario selecciona otro umbral de detección.

Estas características permitirán, en un futuro, aplicar un proceso de clasificación sobre las regiones detectadas como defecto en distintos tipos (burbuja, poros...).

\subsubsection{Precision \& Recall}
En reconocimiento de patrones y recuperación de información, \textit{precision} (precisión, también llamada valor predictivo positivo) es la fracción de instancias recuperadas que son relevantes, mientras que \textit{recall} (exhaustividad, también llamada sensitividad) es la fracción de instancias relevantes que son recuperadas \cite{wiki:precisionandrecall}.

Para tareas de clasificación, los términos verdaderos positivos, verdaderos negativos, falsos positivos y falsos negativos comparan los resultados del clasificador con juicios externos de confianza. Los términos positivo y negativo se refieren a la predicción del clasificador (también conocida como expectación), y los términos verdadero y falso hacen referencia a si la predicción se corresponde con el ya mencionado juicio externo (también conocido como observación). En la imagen \ver{pr} se puede ver mejor cómo se ilustran estas situaciones.

\figuraConPosicion{0.5}{imgs/precisionrecall.png}{Términos precision and recall \cite{wiki:precisionandrecall}}{pr}{}{H}

Con esta idea en mente, ya se pueden entender las fórmulas que definen a \textit{precision}:

\[Precision=\frac{tp}{tp+fp}\]

Y a \textit{recall}:

\[Recall=\frac{tp}{tp+fn}\]

Con estas medidas podemos ver cómo ha sido de preciso el proceso de detección de defectos, mediante la comparación de cuántos píxeles han sido clasificados realmente como defecto y cuántos son realmente defectuosos. Esta información se saca de las máscaras que ya vimos antes.
%%%%%%%%%%%%%%%%%%%%%%%%%%%%%%%%%%%%%%%%%%%%%%%%%%%%%%%%%%%%%%%%%%
%%%%%%%%%%%%%%%%%%%%%%%%%%%%%%%%%%%%%%%%%%%%%%%%%%%%%%%%%%%%%%%%%%
\chapter{Técnicas y herramientas}
%%%%%%%%%%%%%%%%%%%%%%%%%%%%%%%%%%%%%%%%%%%%%%%%%%%%%%%%%%%%%%%%%%
%%%%%%%%%%%%%%%%%%%%%%%%%%%%%%%%%%%%%%%%%%%%%%%%%%%%%%%%%%%%%%%%%%

En este apartado se comentan las técnicas y herramientas utilizadas durante el desarrollo del proyecto.



\section{Técnicas}
En esta sección aparecen recogidas las técnicas utilizadas para la realización del proyecto.


\subsection{Metodología Scrum}
\scrum{} \citeotras{scrum} es una metodología para la gestión y desarrollo de proyectos software basada en un proceso iterativo e incremental. Cada iteración termina con una pieza de software ejecutable que incorpora una nueva funcionalidad o mejora las ya existentes. Estas iteraciones suelen durar de dos a cuatro semanas.

Enumerando los elementos clave de \scrum{} según \textit{Control Chaos} \citeotras{controlchaos}:
\begin{itemize}
 \item \scrum{} es un proceso ágil para gestionar y controlar el trabajo de desarrollo.
 \item \scrum{} es un envoltorio para prácticas de ingeniería existentes.
 \item \scrum{} es una aproximación basada en equipos para desarrollar sistemas y productos iterativa e incrementalmente cuando los requisitos cambian rápidamente.
 \item \scrum{} es un proceso que controla el caos de necesidades e intereses en conflicto.
 \item \scrum{} es una forma de mejorar las comunicaciones y maximizar la cooperación.
 \item \scrum{} es una forma de detectar y eliminar cualquier cosa que se interponga en el desarrollo y distribución de productos.
 \item \scrum{} es una forma de maximizar la productividad.
 \item \scrum{} es escalable desde un único proyecto a organizaciones completas. Ha controlado y organizado el desarrollo e implementación de muchos   productos y proyectos interrelacionados con más de mil desarrolladores e implementadores.
 \item \scrum{} es una forma de que todo el mundo se sienta bien con su trabajo, sus aportaciones, y que ellos han hecho lo mejor que pueden hacer.
\end{itemize}

Los principales beneficios que aporta \scrum{} son:
\begin{itemize}
 \item Entrega mensual o trimestral de resultados lo cual aporta las siguientes ventajas:
  \begin{itemize}
   \item Gestión regular de las expectativas del cliente y basada en resultados tangibles: El cliente establece sus prioridades y cuando espera tenerlo acabado.
   \item Resultados anticipados: el cliente puede empezar a utilizar los resultados mas importantes antes de que esté totalmente finalizado el proyecto.
   \item Flexibilidad de adaptación respecto a las necesidades del cliente, cambios en el mercado, etc.
   \item Gestión sistemático del retorno de inversión (\textit{ROI}): el cliente maximiza el \textit{ROI} del proyecto, de este modo cuando el beneficio pendiente de obtener es menor que el coste del desarrollo el cliente puede finalizar el proyecto.
   \item Mitigación sistemática de riesgos del proyecto: la cantidad de riesgo a la que se enfrenta el equipo está limitada a los requisitos que se puede desarrollar en una iteración.
  \end{itemize}
 \item Productividad y calidad: de manera regular el equipo de desarrollo va mejorando y simplificando su manera de trabajar.
 \item Alineamiento entre cliente y el equipo de desarrollo: todos los participantes del proyecto conocen cuál es el objetivo a conseguir. El producto se enriquece con las aportaciones de todos.
 \item Equipo motivado: las personas están más motivadas cuando pueden usar su creatividad para resolver problemas y cuando pueden decidir organizar su trabajo.
\end{itemize}

En el diagrama \ver{DiagramaFuncionamientoScrum} \citeotras{scrum_process_image} se muestra el funcionamiento y actividades de la metodología \scrum{}.

%Proceso de Scrum
\figura{1}{imgs/ScrumProcess.jpg}{Diagrama de la metodología Scrum}{DiagramaFuncionamientoScrum}{}

Algunos conceptos básicos para entender \scrum{} son:
\begin{itemize}
 \item \productbacklog{}: conjunto de historias de usuario que representan los requisitos funcionales y no funcionales. Se trata de una lista priorizada en función de lo que el cliente da mayor importancia.
 \item \sprintbacklog{}: conjunto de tareas extraídas del \productbacklog{} y que serán realizadas durante un \sprint{}.
 \item \burndownchart{}: gráfico de tareas pendientes por hacer. Representan el esfuerzo y ofrece información sobre la evolución del proyecto.
\end{itemize}

El objetivo del diagrama adjunto \ver{DiagramaFasesScrum} \citeotras{scrum_stages_image} es el de agrupar y sintetizar todos los elementos de la metodología \scrum{}.

%Fases de Scrum
\figuraSinMarco{1}{imgs/ScrumDiagram.png}{Diagrama de etapas en Scrum}{DiagramaFasesScrum}{}

\subsubsection*{Roles}
\scrum{} \citeotras{scrum_wiki} define una serie de roles y que se dividen en dos grupos: gallinas y cerdos.
\begin{itemize}
 \item Roles <<cerdo>>: son aquellos que están comprometidos a construir el software de manera regular y frecuente.
  \begin{itemize}
   \item \textit{Product Owner}: representa la voz del cliente. Debe asegurarse de que el equipo trabaja de forma adecuada desde la perspectiva de negocio. Escribe las historias de usuario, las prioriza y las coloca en el \productbacklog{}.
   \item \textit{Scrum Master}: su principal trabajo es eliminar obstáculos que puedan hacer que el equipo no alcance el objetivo al final del sprint, no es el líder del equipo, pero sirve de pantalla y protección.
   \item \textit{Team}: es el equipo de desarrollo y su responsabilidad es la de generar y entregar el producto (diseñadores, programadores\dots).
  \end{itemize}
 \item Roles <<gallina>>: en realidad no son parte del proceso \scrum{}, pero deben tenerse en cuenta. Estos roles deben participar en el proceso.
  \begin{itemize}
   \item \textit{Stakeholders}: agrupa a la gente que hace posible el proyecto y para quienes el proyecto producirá el beneficio que justifica el coste. Dentro de este grupo estarían los clientes, \textit{stakeholders} \dots
   \item \textit{Managers}: es la gente que establece el ambiente para el desarrollo del producto.
  \end{itemize}
\end{itemize}

\subsubsection*{Reuniones}
\scrum{} define una serie de reuniones para el correcto funcionamiento del equipo. Éstas se encuentran bien definidas en cuanto a contenido y a tiempo empleado.
\begin{itemize}
 \item \textit{Daily Scrum}: Cada día de un \sprint{} se realiza una reunión sobre el estado del proyecto.
  \begin{itemize}
   \item La duración es fija (15 minutos) independientemente del tamaño del equipo.
   \item La reunión debe comenzar puntualmente a la hora. A menudo hay castigos para quien lo incumple.
   \item Todos pueden estar presentes pero solo pueden hablar los ``cerdos''.
   \item La reunión se realiza de pie, facilitando no alargar la reunión.
   \item El lugar y la hora deben ser fijos todos los días.
   \item Preguntas que debe responder cada miembro del equipo:
    \begin{itemize}
     \item ¿Qué has hecho desde ayer?
     \item ¿Qué vas a hacer hoy?
     \item ¿Qué obstáculos te has encontrado?
    \end{itemize}
  \end{itemize}
 \item \textit{Sprint Planning Meeting}: Al inicio de cada \sprint{} se debe llevar a cabo una.
  \begin{itemize}
   \item Crear y planificar, el equipo completo, el \sprintbacklog{}. Se obtiene extrayendo tareas del \productbacklog{}.
   \item Límite de ocho horas.
  \end{itemize}
 \item \textit{Sprint Review Meeting}: Al finalizar cada \sprint{}.
  \begin{itemize}
   \item Revisar el trabajo que fue planificado y no ha sido completado.
   \item Presentar el trabajo completado a los interesados. El trabajo incompleto no puede ser mostrado.
   \item Límite de cuatro horas.
  \end{itemize}
 \item \textit{Sprint Retrospective}: Al finalizar cada \sprint{}.
  \begin{itemize}
   \item Se analiza el \sprint{} y todos los miembros del equipo analizan qué mejoras podrían aplicarse.
   \item Su objetivo es la mejora continua.
   \item Límite de cuatro horas.
  \end{itemize}
\end{itemize}


\subsection{Java}
\java{} es un lenguaje de programación que data de finales de los años 70. Ha tenido una gran implantación debido a la sencillez respecto a otros lenguajes orientados a objetos como \cpp{}.

\java{} ofrece un API (Application Program Interface) que ofrece a los programadores una serie de librerías y facilidades para el desarrollo de aplicaciones \java{}.

El API se encuentra divido en paquetes, que son la estructura de organización lógica. En su interior se encuentran una gran cantidad de clases que cubren un amplio abanico de funcionalidades del desarrollo software en general.

La documentación del API se encuentra disponible en la web y su consulta resulta imprescindible para cualquier tipo de desarrollo en \java{}.

\subsubsection*{Máquina virtual}
La máquina virtual es un programa capaz de interpretar y ejecutar instrucciones expresadas en un código especial (el \java{} \textit{bytecode}). Este código se obtiene al compilar el fuente original con el compilador de \java{}.

Este código es un lenguaje máquina de bajo nivel que es interpretado por la máquina virtual para realizar las operaciones. Ésto hace que el rendimiento de los programas escritos en \java{} sea inferior ya que aparece una nueva pieza intermedia que es la máquina virtual.

La ventaja de ser un lenguaje interpretado es que cualquier programa escrito en \java{} puede ejecutarse en cualquier \textit{hardware} o sistema operativo, la única condición necesaria es que exista una máquina virtual disponible.

\subsection{UML}
El Lenguaje Unificado de Modelado (UML) [18] es el lenguaje de modelado de sistemas de software más conocido y utilizado en la actualidad. Se trata de un lenguaje gráfico, llamado «lenguaje de modelado», que se utiliza para visualizar, especificar, construir y documentar un sistema, describiendo sus métodos o procesos. Es el lenguaje en el que está descrito el modelo.

UML permite modelar la estructura, comportamiento y arquitectura de las aplicaciones. Además, la programación orientada a objetos, que ha sido la elegida para este proyecto, es un complemento perfecto de UML. Por estas razones se ha elegido UML, en su versión 2.0.

UML cuenta con varios tipos de diagramas, los cuales muestran diferentes aspectos de las entidades
representadas. Para la realización del proyecto se han utilizado los siguientes tipos:

\begin{itemize}
\item \textbf{Diagrama de Casos de Uso:} muestra los casos de uso, actores y sus interrelaciones.
\item \textbf{Diagrama de Paquetes:} muestra como los elementos de modelado están organizados en
paquetes, ademas de las dependencias entre esos paquetes.
\item \textbf{Diagrama de Clases:} representa una colección de elementos de modelado estáticos, tales
como clases y tipos, sus contenidos y sus relaciones.
\item \textbf{Diagrama de Secuencias:} modela la lógica secuencial, ordenando en el tiempo los diferentes
mensajes entre entidades.
\end{itemize}

El análisis, diseño e implementación del sistema se ha realizado empleando esta técnica, gracias a lo aprendido en las diferentes asignaturas de la carrera.


\subsection{Weka}
\weka{} es una plataforma de software para aprendizaje automático y minería de datos, diseñada por la Universidad de Waikato. Está escrito en \java{}. Se trata de software libre distribuido bajo licencia \gnu{}.

Las principales ventajas que nos ofrece son:
\begin{enumerate}
 \item Se encuentra escrito en \java{}, lenguaje que se va a usar en el proyecto.
 \item Una consecuencia de lo anterior es que \weka{} es portable, puede funcionar en cualquier sistema operativo.
 \item Se dispone del código fuente para ver cómo hace las cosas y cómo funciona.
 \item Buena documentación (\textit{JavaDoc}) para ayudar a la programación.
 \item Completa API capaz de representar de una manera sencilla la abstracción de las instancias.
 \item Es capaz de obtener los datos de diversos orígenes, tanto de texto como \arff{} y \csv{}, como de bases de datos.
 \item Su uso está muy generalizado en el mundo de la minería de datos.
 \item Ya utilizado con anterioridad, por lo que su manejo no resulta nuevo.
\end{enumerate}
No todo son ventajas y a continuación se detallan los inconvenientes valorados:
\begin{enumerate}
 \item \weka{} carga todas las instancias en memoria por lo que se limita el número de instancias que es capaz de manejar.
\end{enumerate}

\weka{} soporta varias tareas estándar de minería de datos, especialmente, preprocesamiento de
datos, clustering, clasificación, regresión, visualización, y selección. Todas las técnicas de \weka{} se fundamentan en la asunción de que los datos están disponibles en un fichero plano (flat file) o una relación, en la que cada registro de datos está descrito por un número fijo de atributos (normalmente numéricos o nominales, aunque también se soportan otros tipos).


\subsection{ImageJ}
ImageJ [3] es un programa de procesamiento de imagen digital de dominio público programado en \java{} desarrollado en el National Institutes of Health.

La licencia de ImageJ es la siguiente:

\begin{quotation}
ImageJ is a work of the United States Government. It is in the public domain and
open source. There is no copyright. You are free to do anything you want with this
source but I like to get credit for my work and I would like you to offer your changes
to me so I can possibly add them to the, “official” version.
\end{quotation}

Lo que quiere decir básicamente que somos libres de hacer lo que queramos con ImageJ pero que si realizamos algún cambio deberíamos ofrecérselos al creador para que los añada a la versión «oficial».

ImageJ fue diseñado con una arquitectura abierta que proporciona extensibilidad vía plugins \java{} y macros (macroinstrucciones) grabables. Se pueden desarrollar plugins de escaneo personalizado, análisis y procesamiento usando el editor incluido en ImageJ y un compilador \java{}. Los plug-ins escritos por usuarios hacen posible resolver muchos problemas de procesado y análisis de imágenes, desde de imágenes en vivo de las células en tres dimensiones, procesado de imágenes radiológicas, comparaciones de múltiples datos de sistema de imagen hasta sistemas automáticos de hematología.

Aunque ImageJ es extensible mediante plugins y macros, nosotros lo hemos elegido con la finalidad de utilizarlo como librería.

ImageJ puede mostrar, editar, analizar, procesar, guardar, e imprimir imágenes de 8 bits (256 colores), 16 bits (miles de colores) y 32 bits (millones de colores). Puede leer varios formatos de imagen incluyendo TIFF, PNG, GIF, JPEG, BMP, DICOM, FITS, así como formatos RAW (formato).

ImageJ aguanta pilas o lotes, una serie de imágenes que comparten una sola ventana, y es multiproceso, de forma que las operaciones que requieren mucho tiempo se pueden realizar en paralelo en hardware multi-CPU.

ImageJ puede calcular el área y las estadísticas de valor de píxel de selecciones definidas por el usuario y la intensidad de objetos umbral (thresholded objects). Puede medir distancias y ángulos. Se puede crear histogramas de densidad y gráficos de línea de perfil.

Es compatible con las funciones estándar de procesamiento de imágenes tales como operaciones lógicas y aritméticas entre imágenes, manipulación de contraste, convolución, Análisis de Fourier, nitidez, suavizado, detección de bordes y filtrado de mediana. Hace transformaciones geométricas como ampliar, rotación y flips. El programa es compatible con cualquier número de imágenes al mismo tiempo, limitado solamente por la memoria disponible.

Preferimos usar ImageJ frente a otras opciones, como OpenGL, principalmente por su buena documentación (\textit{JavaDoc}), cosa que facilita mucho la programación.

\subsection{Diagramas de Gantt}
El diagrama de Gantt [63], gráfica de Gantt o carta Gantt es una popular herramienta gráfica cuyo objetivo es mostrar el tiempo de dedicación previsto para diferentes tareas o actividades a lo largo de un tiempo total determinado. A pesar de que, en principio, el diagrama de Gantt no indica las relaciones existentes entre actividades, la posición de cada tarea a lo largo del tiempo hace que se puedan identificar dichas relaciones e interdependencias. Fue Henry Laurence Gantt quien, entre 1910 y 1915, desarrolló y popularizó este tipo de diagrama en Occidente.

Por esta razón, para la planificación del desarrollo de proyectos complejos (superiores a 25 actividades) se requiere además el uso de técnicas basadas en redes de precedencia como CPM o los grafos PERT. Estas redes relacionan las actividades de manera que se puede visualizar el camino crítico del proyecto y permiten reflejar una escala de tiempos para facilitar la asignación de recursos y la determinación del presupuesto. El diagrama de Gantt, sin embargo, resulta útil para la relación entre tiempo y carga de trabajo.

En gestión de proyectos, el diagrama de Gantt muestra el origen y el final de las diferentes unidades mínimas de trabajo y los grupos de tareas (llamados summary elements en la imagen) o las dependencias entre unidades mínimas de trabajo (no mostradas en la imagen).

Desde su introducción los diagramas de Gantt se han convertido en una herramienta básica en la gestión de proyectos de todo tipo, con la finalidad de representar las diferentes fases, tareas y actividades programadas como parte de un proyecto o para mostrar una línea de tiempo en las diferentes actividades haciendo el método más eficiente.

Básicamente el diagrama está compuesto por un eje vertical donde se establecen las actividades que constituyen el trabajo que se va a ejecutar, y un eje horizontal que muestra en un calendario la duración de cada una de ellas.



\section{Herramientas}
En esta sección aparecen aparecen cada una de las herramientas utilizadas para la realización del proyecto.


\subsection{Eclipse}
Como entorno de desarrollo de la biblioteca se utilizará Eclipse. La decisión fue tomada por haber trabajado anteriormente con esta herramienta y conocer sus ventajas e inconvenientes. La disponibilidad de \textit{plugins} disponibles facilita el desarrollo, integrando el control de versiones, \textit{suites} de pruebas \dots

Eclipse es un producto realizado por la \textit{Eclipse Foundation}, que es una comunidad de codigo abierto que tiene como objetivo desarrollar una plataforma para el desarrollo software.

Se encuentra escrito en \java{} bajo una licencia propia, la \epl{} (Eclipse Public License \citeotras{epl}).

Aunque en su origen se creó para \java{} existen versiones de todo tipo para otros lenguajes como pueden ser \clang{} o adaptaciones comerciales para productos o lenguajes concretos.

Página web de la herramienta: \url{http://www.eclipse.org/}.


\subsection{JUnit}
\textit{JUnit} es un conjunto de bibliotecas o \textit{framework} que son utilizadas para realizar las pruebas unitarias de aplicaciones \java{}. Dispone de una buena reputación dentro de la literatura sobre programación de pruebas.

El propio \textit{framework} permite visualizar los resultados en texto, como gráficos o como tarea de \ant{}.

Se utilizará el \textit{plugin} de eclipse por la facilidad que aporta para la ejecución de las pruebas y su total integración con el entorno de desarrollo.

Se ha utilizado por haber sido utilizado durante la carrera y ser el lanzador más conocido por el autor.

Página web de la herramienta: \url{http://www.junit.org/}.


\subsection{JDepend}
Se trata de una herramienta de métricas que permite conocer información de utilidad de un proyecto software.

Se utilizará el \textit{plugin} para Eclipse que permite visualizar los valores de las métricas y la gráfica comparativa de cada paquete.

Página web de la herramienta: \url{http://www.clarkware.com/software/JDepend.html}.


\subsection{Source Monitor}
Se trata de una herramienta de análisis de código capaz de analizar proyectos escritos en diversos lenguajes.

Su utilización ha sido motivada porque permitirá comparar las métricas obtenidas con los umbrales establecidos por la Universidad de Burgos.

Página web de la herramienta: \url{http://www.campwoodsw.com/sourcemonitor.html}.


\subsection{Astah}
Es una herramienta de modelado \uml{} creado por la compañía \textit{ChangeVision}.

Al estar pensado para \java{}, permite la importación y exportación de código fuente y la generación de gráficos automáticos.

En el proyecto se utilizará la versión \textit{Community} porque es gratuita y cubre las necesidades de modelado del proyecto.

Página web de la herramienta: \url{http://astah.net/editions/community}.


\subsection{GitHub}
\textit{GitHub} es una plataforma para el desarrollo colaborativo de software que utiliza el control de versiones \textit{Git}.

Es una herramienta completa y robusta que permite la creación de grupos, ramas, etiquetas y todo tipo de artefactos necesarios para la organización de un proyecto de programación.

Las principales ventajas que nos ofrece son:
\begin{itemize}
\item Nuestro código queda alojado en la nube, permitiendo acceder a él desde cualquier lugar.
\item Nos permite trabajar con la metodología de \textit{Rama por tarea}, en la que se crea una rama por cada tarea a realizar, permitiendo trabajar en dos tareas simultáneamente, fusionando después los cambios.
\item Tiene cliente propio multipltaforma, lo que nos permite gestionar el repositorio de una forma muy sencilla, pudiendo validar los cambios que hagamos en el código y subiendo estos cambios al servidor.
\end{itemize}

Página web de la herramienta: \url{https://github.com/}.


\subsection{PivotalTracker}
\textit{PivotalTracker} será utilizada como herramienta de gestión y control de tareas y errores.

Está especializada en proyectos ágiles por lo que da soporte a todos los conceptos de la metodología \scrum{} utilizada en el proyecto.

Se utilizará la versión de \textit{hosting} que permite acceder desde cualquier equipo al servidor desde un navegador. Además, permite sincornizarse con \textit{GitHub}, con lo que podemos ver cómo se van realizando las tareas tanto en la propia herramienta como en el código. En el Apéndice A, aparece un breve manual de usuario que permite visualizar la planificación y las tareas.

Página web de la herramienta: \url{https://www.pivotaltracker.com/}.


\subsection{MiK\TeX{}}
Mik\TeX{} es una distribución \TeX{}/\LaTeX{} libre de código abierto para Windows.

Una de sus características es la capacidad que tiene para instalar paquetes automáticamente sin necesidad de intervención del usuario. Al contrario que otras distribuciones, su instalación es extremadamente sencilla.

Ha sido utilizada por recomendación de César quien suministró un tutorial sobre su instalación.

Página web de la herramienta: \url{http://miktex.org/}.


\subsection{\TeX{}Maker}
\TeX{}Maker es un editor de \LaTeX{} multiplataforma similar a \textit{Kile}.

Aunque existen multitud de editores para \LaTeX{} se ha escogido este por haber sido recomendado por César, dado que nunca antes se ha trabajado con este lenguaje de documentación se aceptó la sugerencia. Además, aportó un tutorial de como instalar y configurar el editor en el sistema operativo Windows.

Una de sus características es la posibilidad de trazabilidad de código desde PDF. Configurando el editor y un visor de PDF, el programa es capaz de detectar la línea a la que se corresponde un determinado comando \LaTeX{}. Esto es especialmente ventajoso cuando, como en este caso, no se conoce el lenguaje.

En el enlace \url{http://en.wikipedia.org/wiki/Comparison_of_TeX_editors} aparece una comparativa entre diversos editores disponibles.

Página web de la herramienta: \url{http://www.xm1math.net/texmaker/}.

\subsection{WindowBuilder}
WindowBuilder es un \textit{plugin} de Eclipse que permite diseñar de una forma fácil y rápida interfaces gráficas basadas en \textit{Swing}.

Hemos elegido este editor de interfaces gráficas debido a su facilidad para crearlas, ya que incluye un editor \textit{WYSIWYG}, con el que podemos arrastrar los elementos a la ventana que estamos creando y moverlos hasta dejarlos en la posición deseada.

La otra parte interesante de este \textit{plugin} es que genera el código automáticamente, con lo que nos ahorra mucho trabajo. Además, el código que genera está bien agrupado, con lo que después es muy fácil refactorizarlo.

Página web del \textit{plugin}: \url{http://www.eclipse.org/windowbuilder/}.

\subsection{Auto Local Threshold}
Auto Local Threshold es un \textit{plugin} de ImageJ que implementa la segmentación de imágenes mediante los filtros de umbrales locales.

Por defecto, ImageJ no tiene esta funcionalidad, así que tuvimos que descargarnos e instalar este \textit{plugin}, que permite realizar estas opciones de forma muy sencilla.

Para poder usarlo en el proyecto, tuvimos que cambiar alguna cosa de la clase del \textit{plugin} para poder usarlo con nuestro código.

Página web del \textit{plugin}: \url{http://fiji.sc/wiki/index.php/Auto_Local_Threshold}.

\subsection{Apache Commons IO}
Apache Commons IO es una librería de utilidades para asistir al desarrollo de funcionalidad relacionada con entrada/salida.

Hemos decidido utlizarla debido a que simplifica mucho el realizar algunas operaciones con ficheros, como es la fusión de uno o más ficheros de texto, o la exportación de un cierto texto a un fichero externo.

La página web de la herramienta es: \url{http://commons.apache.org/proper/commons-io/}.



%%%%%%%%%%%%%%%%%%%%%%%%%%%%%%%%%%%%%%%%%%%%%%%%%%%%%%%%%%%%%%%%%%
%%%%%%%%%%%%%%%%%%%%%%%%%%%%%%%%%%%%%%%%%%%%%%%%%%%%%%%%%%%%%%%%%%
\chapter{Aspectos relevantes del desarrollo del proyecto}
%%%%%%%%%%%%%%%%%%%%%%%%%%%%%%%%%%%%%%%%%%%%%%%%%%%%%%%%%%%%%%%%%%
%%%%%%%%%%%%%%%%%%%%%%%%%%%%%%%%%%%%%%%%%%%%%%%%%%%%%%%%%%%%%%%%%%

En este apartado se detallan los aspectos más relevantes que se han encontrado en el proceso de desarrollo del proyecto.

\section{Problemas y retos}
El tema principal del proyecto era totalmente desconocido. Para la obtención de los defectos en las radiografías se han utilizado técnicas novedosas que no se enseñan en ninguna asignatura de la carrera. Hay que añadir que algunas de esas técnicas se usan actualmente en proyectos de investigación en universidades y centros de investigación con mucha más experiencia en este campo que la Universidad de Burgos.

Se busca resolver un problema real, con imágenes reales (cedidas por el Grupo Antolín) obtenidas desde los puntos de vista y condiciones que obtiene una máquina real, piezas reales y con formas muy complejas. Es un problema mucho más complicado que el que se aborda en los artículos relacionados donde la imagen suele ser mucho mas uniforme y sencilla. En la imagen \ver{radiografia_articulos} se puede ver una comparativa de algunas de las imágenes usadas en otros artículos. En la figura \ver{radiografia_antolin} se puede ver un ejemplo de una de las imágenes que usamos en nuestro proyecto.

Por lo anterior, este ha sido un proyecto de gran incertidumbre y riesgo que ha hecho que:

\begin{enumerate}
\item Como se preveía que los requisitos del proyecto iban a cambiar mucho durante el desarrollo del mismo, ya que surgen nuevas ideas, nuevas aproximaciones para resolver el problema, se decidió usar la metodología Scrum, que está pensada para este tipo de entornos, en los que los requisitos cambian y es difícil hacer una planificación.

\item Haya sido necesario adquirir muchos nuevos conocimientos.
\end{enumerate}

\figura{1}{imgs/radiografia_articulos.png}{Radiografías usadas en otros artículos[5] [36] [38]}{radiografia_articulos}{}

\figura{1}{imgs/radiografia_antolin.png}{Ejemplo de radiografía usada en nuestro proyecto}{radiografia_antolin}{}


\section{Mejoras respecto a la versión previa}
Uno de los objetivos del proyecto era mejorar a la versión previa presentada el curso pasado. Esto supuso algunos problemas, ya que arreglar ciertos aspectos no ha sido tan sencillo.

Las mejoras introducidas han sido:

\begin{enumerate}

\item \textbf{Mejoras sobre el código:} se ha mejorado el código en general, procurando que sea más claro y evitando ciertos defectos de código, en la medida de lo posible. Se han refactorizado, por ejemplo, algunas clases en las que había métodos exageradamente largos, procurando dividirlos en métodos más pequeños, lo que mejora mucho la legibilidad.

\item \textbf{Mejoras en la estructura:} se ha cambiado completamente la estructura de la aplicación, introduciendo un diseño en tres capas y algunos patrones de diseño que permiten mejorar el mantenimiento y la extensibilidad de la aplicación.

\item \textbf{Mejoras de rendimiento:} se ha intentado mejorar el rendimiento general de la aplicación. Básicamente, esto lo hemos conseguido mediante el proceso paralelo de los distintos trozos de la imagen mediante multihilo, aunque también se han cambiado algunos cálculos para que sean más eficientes.

\item \textbf{Mejoras en la interfaz:} la interfaz se cambió casi completamente, buscando una mejor intuitividad para el usuario. Por ejemplo, los botones son ahora más claros y se desactivan cuando no se pueden usar, cosa que antes no pasaba y podía llegar a causar problemas.

\item \textbf{Mejoras en la documentación y ayuda:} se ha mejorado la calidad de la API, extendiéndola a todos los elementos del código y arreglando fallos del lenguaje. Además, se ha añadido un módulo de ayuda en línea para permitir al usuario consultar cualquier duda de una forma rápida y sencilla [AÚN NO ESTÁ HECHO].

\item \textbf{Mejoras en la precisión:} se ha buscado mejorar la detección de defectos implementando nuevas aproximaciones y mejorando la que ya existía.

\item \textbf{Ampliación de funcionalidad:} no sólo se han añadido nuevas aproximaciones para la detección de defectos, sino que también se ha ampliado la funcionalidad de la aplicación en otras formas, como por ejemplo, el cálculo de características geométricas, la interactividad con los defectos detectados, las medidas de \emph{precision \& recall},...

\end{enumerate}

En definitiva, se ha intentado que, a partir de la buena base que representaba el proyecto del año pasado, se pueda ampliar esta idea de una forma mucho más sencilla. Se ha buscado, por tanto, crear una aplicación mucho más fácil de ampliar, ya que este proyecto es candidato a recibir una innumerable cantidad de mejoras prometedoras en un futuro.


\section{Conocimientos adquiridos}
Durante el desarrollo del proyecto se han ido aprendiendo y perfeccionando distintas disciplinas, las cuales aparecen detalladas a continuación.

\subsection{Minería de datos}
Al inicio del proyecto el conocimiento sobre la minería de datos estaba limitado a los conocimientos adquiridos en la asignatura de Minería de Datos de 5º curso de Ingeniería Informática.

Por este motivo cuando se expuso el proyecto se entendió como un reto y una manera de poder aprender sobre esta interesante rama de la informática en la que entran en juego grandes volúmenes de datos.

De este modo y a base de leer artículos, se han asimilado y refinado multitud de conceptos y técnicas.

\subsection{Weka}
En la asignatura de Minería de Datos, de la que ya hemos hablado en el apartado anterior, se utilizó \weka{} para realizar las prácticas. Es por ello que ya teníamos algunos conocimientos sobre esta herramienta.

Este proyecto nos ha permitido usar \weka{} de una forma que no habíamos considerado hasta ahora, y es incluir alguno de sus métodos dentro de nuestra propia aplicación, aprovechando las posibilidades que nos brinda la herramienta.

\subsection{Metodología Scrum}
Las metodologías ágiles han adquirido un gran éxito dentro del desarrollo de software. Por este motivo el proyecto de final de carrera se presentaba como una buena base sobre la que aplicar una de estas metodologías y aprender de ella.

Se eligió \scrum{} por el hecho de que está pensado para entornos en los que cambian los requisitos y es difícil hacer una buena planificación.

Además, \scrum{} se ha explicado en la asignatura de Planificación y Gestión de Proyectos de 4º curso de Ingenería Informática. De este modo la realización del proyecto bajo esta nueva metodología ha aportado una experiencia adicional a los desarrollos clásicos en cascada que son utilizados en multitud de empresas.

\subsection{Programación multihilo}
Para poder mejorar el rendimiento de la aplicación, se hizo necesaria la programación multihilo. Ya poseíamos algunos conocimientos de algunas asignaturas de la carrera, pero nunca lo habíamos usado en Java. Por ello, ha sido necesario leer documentación al respecto y lidiar con algunos problemas que pueden presentar este tipo de aplicaciones.

\subsection{Documentación en \LaTeX{}}
Al principio, la temida documentación se iba a desarrollar con uno de los clásicos compositores
de texto, pero pronto nuestros tutores nos recomendaron utilizar encarecidamente la herramienta
\LaTeX{} [30]. Este hecho, que a priori parecía un reto sencillo, se convirtió en un proceso de cierta
complejidad y con una curva de aprendizaje larga e intensa.

En estos momentos damos gracias a nuestros tutores por empeñarse en convencernos a utilizar
\LaTeX{}, ya no solo por el resultado estético que se obtiene, sino por darnos un nuevo reto a superar
que añade valor a la realización de este proyecto y la documentación, además del valor para el
futuro laboral.

Hemos utilizado una plantilla creada por el alumno Álvar Arnáiz González en el proyecto «Biblioteca
de algoritmos de selección de instancias y aplicación orientada a su docencia» [74]. Futuros
alumnos podrán disfrutarla y mejorarla. Tarde o temprano, el hecho de que cada año nuevo
alumnos utilicen y mejoren esta plantilla, hará que la Universidad de Burgos tenga una plantilla
estándar para la realización de cualquier memoria escrita en \LaTeX{}.

%%%%%%%%%%%%%%%%%%%%%%%%%%%%%%%%%%%%%%%%%%%%%%%%%%%%%%%%%%%%%%%%%%
%%%%%%%%%%%%%%%%%%%%%%%%%%%%%%%%%%%%%%%%%%%%%%%%%%%%%%%%%%%%%%%%%%
\chapter{Trabajos relacionados}
%%%%%%%%%%%%%%%%%%%%%%%%%%%%%%%%%%%%%%%%%%%%%%%%%%%%%%%%%%%%%%%%%%
%%%%%%%%%%%%%%%%%%%%%%%%%%%%%%%%%%%%%%%%%%%%%%%%%%%%%%%%%%%%%%%%%%

En este apartado se hablará de algunos trabajos relacionados con los temas que se tratan en el proyecto.

\section{Artículos Estudiados}
A la hora de comenzar el proyecto, necesitamos leer el artículo principal sobre el que estaba basado el proyecto anterior. Además, durante el desarrollo del mismo, fue necesario leer otros artículos para buscar nueva información, como por ejemplo, cómo clasificar defectos a través de características geométricas. A continuación se muestra un listado de todos los artículos, junto con una tabla comparativa \vertabla{tablaComparativa} de aquellos artículos que nos han parecido más interesantes. Para ver un resumen más detallado de estos artículos, se puede ir a la sección \ref{estadoArte} de la memoria. 

\begin{itemize}
		\item\textit{Automated detection of welding defects without segmentation} \cite{DomingoMery}
		\item \textit{An automatic system of classification of weld defects in radiographic images} \cite{vilar2009automatic}
		\item \textit{Recognition of welding defects in radiographic images by using support vector machine classifier} \cite{wang2010recognition}
		\item \textit{Image thresholding based on the EM algorithm and the generalized Gaussian distribution} \cite{bazi2007image}
		\item \textit{Weld defect classification using EM algorithm for Gaussian mixture model} \cite{tridi}
		\item \textit{Multiclass defect detection and classification in weld radiographic images
using geometric and texture features}\cite{Valavanis20107606}
	\end{itemize}


%Tabla comparativa de artículos
\tablaSinColores{Tabla comparativa de artículos}{| p{3.5cm} | p{2cm} | p{0.75cm} | p{2.5cm} | p{1.75cm} | p{1.75cm} |}{6}{tablaComparativa}{
\multicolumn{1}{c}{Título} & \multicolumn{1}{c}{Autores} & \multicolumn{1}{c}{Año} & \multicolumn{1}{c}{Preprocesamiento} & \multicolumn{1}{c}{Características} & \multicolumn{1}{c}{Clasificador} \\
 }
 {
  An automatic system of classification of weld defects in radiographic images & Rafael Vilar et al. & 2009 & Filtro adaptativo de Wiener de 7 $\times$ 7. \newline Filtro gausiano de paso bajo de 3 $\times$ 3. \newline Método de Otsu. & Área, centroide, eje mayor, eje menor, excentricidad, etc. & Red neuronal artificial (ANN) \\ \hline
  Weld defect classification using EM algorithm for Gaussian mixture model & M.Tridi et al. & 2005 & Segmentación de la imagen & Geométricas: área, longitud, anchura, elongación, perímetro, etc. & k-medias \\ \hline
  Recognition of welding Defects in radiographic images by using support vector machine & X.Wang et al. & 2010 & Binarización adaptativa basada en wavelet \newline Ecualización adaptativa del histograma & Características de Haralick, de Gabor, de matriz de co-ocurrencia y morfológicas & Máquina de vector de soporte (SVM) \\ \hline
  Automated detection of welding defects without segmentation & Domingo Mery   		& 2011 & Saliency Map & Características estándar, de Haralick y LBP & Máquina de vector de soporte (SVM)  \\ \hline
  Multiclass defect detection and classification in weld radiographic images using geometric and texture features & Ioannis Valavanis et al. & 2010 & Umbrales locales de Savuola. \newline Método de segmentación basado en grafos. & De texturas (2º momento angular, contraste, correlación, suma de cuadrados, etc) y geométricas (posición, ratio de aspecto, área, longitud, redondez...). & SVM y red neuronal artificial \\
 }
 
 \newpage
\section{Revisión del estado del arte}\label{estadoArte}
En este apartado se examina brevemente el estado del arte relevante para el tema de este proyecto. Es importante realizar una revisión bibliográfica ya que nos permitirá saber si el problema que nos planteamos está ya resuelto, así como conocer lo que otros investigadores han aportado en nuestra línea de trabajo y cómo han planteado y realizado sus investigaciones. Además, es necesario conocer con detalle las técnicas experimentales que otros han usado en problemas parecidos al nuestro, para seguirlas o para modificarlas.

A continuación, se incluyen resúmenes de aquellos artículos que nos han parecido más interesantes para ayudarnos en el desarrollo del proyecto.


\subsection{An automatic system of classification of weld defects in radiographic images - Rafael Vilar et al.}
En este artículo \cite{vilar2009automatic} se estudia la forma de detectar defectos de soldadura en radiografías. Se utilizan imágenes de 8 bits con una resolución de 2900$\times$1950 píxeles. Consta de las siguientes fases:


\subsubsection{Preprocesado de las imágenes}
Para reducir el ruido se utilizan dos técnicas:
	\begin{itemize}
	\item Filtro adaptativo de Wiener de 7$\times$7 \cite{wiener1949extrapolation}.
	\item Filtro gausiano de paso bajo de 3$\times$3.
	\end{itemize}
También se utilizan técnicas para mejorar el contraste.
Finalmente, la imagen se divide en bandas de 640$\times$480 píxeles. 


\subsubsection{Segmentación de las regiones de soldadura}
El objetivo de esta fase es aislar la región de soldadura del resto de elementos. El proceso se desarrolla en tres fases. 
En la primera se busca un umbral óptimo que permite binarizar la imagen, separando los píxeles de los objetos de los píxeles del fondo. Para ello se utiliza el método de Otsu \cite{otsu1979threshold}.
En la segunda se etiquetan los componentes conectados de la imagen binarizada. Se utiliza el procedimiento propuesto por Haralick y Shapiro \cite{haralick1992computer}, que devuelve una matriz con el mismo tamaño que la imagen. Los píxeles etiquetados como <<0>> son el fondo, los píxeles etiquetados como <<1>> representan un objeto, los píxeles etiquetados como <<2>> representan un segundo objeto y así sucesivamente.
Para concluir, en la tercera fase, como un criterio para seleccionar entre los objetos etiquetados, el área máxima es establecida. De esta manera, se identifica la región de soldadura de entre todos los objetos de la imagen.


\subsubsection{Segmentación de heterogeneidades}
Se toma como entrada la imagen producida por la fase anterior. 
La salida obtenida es una imagen que contiene únicamente defectos potenciales.
Primero, se binariza la imagen utilizando el método de Otsu para obtener el umbral óptimo. Después, se traza el borde exterior de los objetos. Una vez que se ha hecho esto, se deduce que los defectos son objetos situados dentro de una región de soldadura.


\subsubsection{Extracción de características}
La salida de esta fase es una descripción de cada defecto candidato de la imagen. Las características extraídas son: área, centroide (coordenadas $X$ e $Y$), eje mayor, eje menor, excentricidad, orientación, número de Euler \cite{dunham1999euler}, diámetro equivalente, solidez, extensión y posición. Se genera un vector de entrada (12 componentes) para cada defecto candidato y expertos humanos en defectos de soldadura producen un vector objetivo asociado.


\subsubsection{Análisis de componentes principales}
En esta fase se reduce el tamaño de los vectores de características de entrada. Para ello se utiliza la técnica PCA \cite{PCA}.


\subsubsection{Predicción utilizando una red neuronal multicapa}
Se utiliza una red neuronal multicapa para clasificar los defectos.
Se implementan clasificadores de patrones no lineales de tipo supervisado utilizando ANN.

Con los datos de entrenamiento, el error es pequeño, pero cuando se introducen nuevos datos a la red el error es grande. La red ha memorizado los ejemplos de entrenamiento pero no ha aprendido a generalizar en nuevas situaciones. Para mejorar la generalización se utilizan tres técnicas:
	\begin{enumerate}
	\item Regularización.
	\item Regularización de Bayes.
	\item \emph{Early stopping} o \emph{bootstrap}.
	\end{enumerate}

Para evaluar el rendimiento de la red, se realiza un análisis de regresión entre la respuesta de la red y los objetivos correspondientes. El coeficiente de correlación obtenido entre las salidas y los objetivos es una medida de cómo de bien es explicada la variación de la salida por los objetivos. Si este número es igual a uno, entonces hay correlación perfecta entre los objetivos y las salidas. Para determinar el coeficiente de correlación se emplea una regresión lineal usando el método de mínimos cuadrados.


\subsection{Weld defect classification using EM algorithm for Gaussian mixture model - M.Tridi et al.}
En este articulo \cite{tridi}, se proponen dos algoritmos de clasificación de los defectos de soldadura  \textit{(El algoritmo Fuzzy-C-Means Iterativo: FCMI y el algoritmo Expectation maximization: EM)}.
El primer algoritmo se basa en el concepto de distancia y lógica difusa, y el segundo está basado en conceptos estadísticos. 


\subsubsection{Algoritmo Fuzzy-C-Means Iterativo} 
El algoritmo \textit{Fuzzy-C-Means Iterativo} \cite{kandel1999introduction} utiliza el concepto de lógica difusa y distancia para la clasificación. Está dada por el siguiente algoritmo:
	\begin{enumerate}
	\item Los centros de los clusters son inicializados en un conjunto de ejemplos
	\item Cálculo de la distancia euclidiana entre cada muestra y cada centro de cluster
	\item Cálculo de la función de pertenencia (\emph{Fuzzification}) 
	\end{enumerate}


\subsubsection{The EM algorithm}
El algoritmo \textit{Expectation Maximization (EM)} \cite{zhang2003algorithms}, es una extensa clase de algoritmos iterativos usada para estimación de máxima verosimilitud o máxima probabilidad a posteriori en problemas en los que faltan datos.


\subsubsection{Aplicación en la clasificación de defectos de soldadura}
En esta aplicación se ha tomado una base de datos formada por 72 radiografías con defectos.
Para poder clasificar un patrón (imagen segmentada), es esencial caracterizarlas por un vector de características. La elección de este vector está basada en el conocimiento obtenido por un experto en radiografías. Se pueden encontrar varios tipos de características, como por ejemplo: momentos Zernik, momentos Legendre, momentos Geométricos, coeficientes de Fourier etc.
Las características usadas en esta aplicación son parámetros o características geométricas. Este tipo de características consisten en caracterizar un objeto acorde al vector cuyos elementos son característicos, como por el perímetro, superficie, dirección principal de la inercia inercia y elongación.

Se advierte el hecho de que los centros de los clusters representan eficientemente las cuatro clases (Y1 para roturas, Y2 para falta de penetración, Y3 para inclusión de gas e Y4 para inclusión de óxido), y son diferentes entre sí. 


\subsubsection{Conclusión}
Se describe un nuevo enfoque para clasificar el defecto de soldadura para las imágenes de radiografía usando el algoritmo EM. El algoritmo EM es muy sensible a la elección de los valores iniciales de los parámetros. En este caso, se ha utilizado el algoritmo de $k$-medias para la inicialización. La principal contribución es una comparación entre los algoritmos EM y FCMI. Los resultados experimentales indican que este algoritmo ha dado mejores resultados que el algoritmo FCMI.



\subsection{Recognition of Welding Defects in Radiographic Images by Using Support Vector Machine - X.Wang et al.}
En este artículo \cite{wang2010recognition} se describe un método para detectar defectos en imágenes de rayos X basado en Support Vector Machine (SVM).
El método está compuesto por tres fases:


\subsubsection{Preprocesado de las imágenes}
 Las imágenes que se van a analizar tienen bajo contraste, mucho ruido y fondo no uniforme. Para mejorar estas condiciones se utilizan dos métodos: \textit{binarización adaptativa basada en wavelet y ecualización adaptativa del histograma}. 
Después, se segmenta la imagen utilizando \textit{umbralización multi-nivel basada en entropía máxima borrosa}.
Se segmentan los defectos hipotéticos (algunos son falsas alarmas).


\subsubsection{Extracción de características}
Este apartado se centra en la medición de las propiedades de las regiones. Se extraen dos tipos de características:
	\begin{enumerate}
	\item Características de textura
	Se extraen la matriz de co-ocurrencia y los filtros de Gabor \cite{daugman1988complete}. Para medir estas características, se utilizan 4 de las 14 medidas propuestas por Haralick \cite{haralick1973textural}:
		\begin{itemize}
		\item Shannon Entropy
		\item Contrast
		\item Angular Second Moment
		\item Inverse Difference Moment
		\end{itemize}
	Se extraen 64 características de Gabor y 16 a partir de la matriz de co-ocurrencia.

	\item Características morfológicas: área, longitud, anchura, elongación, orientación, ratio entre la anchura y el área (RWA), compacidad.
	
	Estas características, sumadas a las de textura, nos dan un total de 87 características
	\end{enumerate}


\subsubsection{Clasificación de patrones}
Se dividen las imágenes en regiones específicas de acuerdo a las características extraídas, clasificándolas en dos grupos (<<defecto>> o <<no defecto>>).
Se seleccionan 16 características combinando las 12 mejores obtenidas mediante la aplicación de un algoritmo basado en SVM y las 12 mejores obtenidas con un análisis ROC.
Se entrena la SVM con los vectores formados por esas características.
Después, se preprocesa la imagen de prueba para extraer las 16 características, y entonces se aplica la SVM entrenado para que decida entre <<defecto>> y <<no defecto>>.


\subsection{Multiclass defect detection and classification in weld radiographic images using geometric and texture features - Ioannis Valavanis}

\subsubsection{Local thresholding}
Los autores usan un filtro de umbrales locales, llamado \emph{Sauvola} \cite{Sauvola00adaptivedocument}, como primer paso para la detección de defectos. Los autores dicen que se comporta mejor que otros métodos, como \textit{Otsu} o \textit{Niblack}. Después de usar este método, se hace necesario utilizar unas operaciones morfológicas para eliminar una serie de puntos aislados (ruido). Con esto, se obtienen una serie de regiones candidatas a albergar defectos, marcadas en blanco.

\subsubsection{Segmentación}
El segundo paso para la detección de defectos es usar una operación de segmentación. Se utiliza un método basado en grafos que es capaz de capturar regiones perceptualmente distintas, aunque su interior se caracterice por una alta variabilidad, considerando características globales de la imagen.

\subsubsection{Clasificación}
Para realizar la clasificación de los defectos se hace necesario calcular una serie de descriptores de regiones, divididos en descriptores de texturas y descriptores geométricos. Los descriptores de texturas son los mismos que hemos usado en nuestro proyecto, ya que este artículo también está basado en el artículo de Domingo Mery. Entre los geométricos se encuentran el área, la redondez, semieje mayor y menor de la mayor elipse,...

Como clasificadores, los autores usan \emph{support vector machine} y una red neuronal artificial, realizando una comparativa entre ambos.
%%%%%%%%%%%%%%%%%%%%%%%%%%%%%%%%%%%%%%%%%%%%%%%%%%%%%%%%%%%%%%%%%%
%%%%%%%%%%%%%%%%%%%%%%%%%%%%%%%%%%%%%%%%%%%%%%%%%%%%%%%%%%%%%%%%%%
\chapter{Conclusiones y líneas de trabajo futuras}
%%%%%%%%%%%%%%%%%%%%%%%%%%%%%%%%%%%%%%%%%%%%%%%%%%%%%%%%%%%%%%%%%%
%%%%%%%%%%%%%%%%%%%%%%%%%%%%%%%%%%%%%%%%%%%%%%%%%%%%%%%%%%%%%%%%%%

En este capítulo se van a exponer las conclusiones obtenidas tras el desarrollo del proyecto y las posibles líneas de trabajo futuras.


\section{Aspectos que han complicado la realización del proyecto}
Los retos más importantes que han surgido durante el proyecto y han complicado la realización del mismo han sido los siguientes:
\begin{itemize}
 \item \LaTeX{}: La curva de aprendizaje de este lenguaje es bastante dura y esto ha sido especialmente visible en las primeras fases del proyecto.
 \item Desconocimiento del \textit{background} teórico: Los conocimientos previos sobre visión artificial eran muy escasos, por lo que ha sido complicado adaptarse a tantos conocimientos nuevos. 
\item Programación multihilo: Nos ha dado numerosos quebraderos de cabeza implementar los hilos y aplicarlos al análisis de imágenes paralelamente.
\item Comprensión del proyecto del año pasado: Entender el proyecto del año pasado ha sido difícil. Por una parte, por lo que ya hemos dicho sobre el desconocimiento del \textit{background} teórico. Por otra parte, porque el código era bastante complicado de entender, no sólo por su complejidad, sino también por la falta de un diseño robusto que facilite el mantenimiento y porque la documentación no siempre era todo lo buena que esperábamos.
\item Dificultades para planificar: No siempre era sencillo planificar un \sprint{}, ya que hemos tenido una carga de trabajo muy importante a lo largo de todo el curso.
 
\end{itemize}

\newpage
\section{Conclusiones}
Las conclusiones extraídas tras el desarrollo del proyecto son detalladas a continuación:
\begin{itemize}
\item Se ha mejorado la precisión de la herramienta del año pasado mediante la mejora de algunos aspectos (como la determinación de cuándo una ventana es defectuosa o no) o la inclusión de nuevas características, como los filtros de umbrales adaptativos.
\item Se ha mejorado el rendimiento de la aplicación, incluyendo la programación multihilo y el cambio de algunos cálculos para que fueran más rápidos.
 \item Se ha mejorado la \gui{} (\textit{Intefaz Gráfica de Usuario}), haciéndola más intuitiva y funcional.
 \item Se ha mejorado el diseño de la aplicación, buscando que sea más fácil de ampliar y mantener.
 \item Se ha mejorado la documentación del código.
 \item Se ha mejorado la calidad del código.
 \item Se han incluido nuevas funcionalidades, como el cálculo de características geométricas y la tabla de resultados, que permite interactuar con los defectos dibujados.
 \item Durante todo el proceso se han reforzado conceptos y técnicas tratadas durante la carrera.
 \item Se ha perfeccionado el conocimiento sobre \java{}, como por ejemplo, mediante la programación multihilo.
 \item Se ha aprendido un nuevo modo de realizar documentos técnicos con el uso de \LaTeX{}. Aunque en un principio supuso una carga a la documentación, a medida que avanzó el desarrollo, favoreció el interés por la escritura debido a los retos que se presentaron durante su ejecución.
 \item Se ha podido aplicar y, así, obtener un conocimiento más profundo, una metodología de desarrollo ágil como es \textit{Scrum}, tan en auge en la actualidad.
 \item Por último, destacar el perfeccionamiento de todas las tareas del desarrollo software: planificación, análisis, diseño, implementación, pruebas y documentación
\end{itemize}

Por todo ello, consideramos cumplidos los objetivos del proyecto.


\newpage
\section{Líneas de trabajo futuras}
Este proyecto representa un primer prototipo de una herramienta que debe evolucionar y mejorar con los años. Además, el análisis de la bibliografía y de los conceptos teóricos supone un esfuerzo que puede facilitar el trabajo de los alumnos que continúen el desarrollo. 

Durante la fase de diseño se ha tenido muy en cuenta la realización de una aplicación que permita posteriores ampliaciones y mejoras. Se ha buscado crear una base sobre la que se pueda seguir como referencia a la hora de realizar trabajos parecidos o ampliar el mismo. Ha habido algunas ideas que no han podido ser incorporadas debido a la falta de tiempo, y que podrían ser implementadas en un futuro para ampliar el proyecto. A continuación se proponen algunas:

\begin{itemize}
\item Clasificar los defectos en tipos: actualmente, la aplicación únicamente detecta los defectos, pero no los clasifica. Sería interesante conseguir que, una vez detectado el defecto, se informara al usuario del tipo al que pertenece. La inclusión del cálculo de características geométricas debería facilitar esto.
\item Cálculo de nuevas características: Se podrían añadir más características, como los \emph{Filter Banks}, para intentar mejorar la detección aún más.
\item Inclusión de una nueva forma de detectar defectos, descrita en [INSERTAR ARTÍCULO DE CAEPIA], artículo presentado por nuestros tutores, José Francisco Díez Y César I. García, al CAEPIA'13 (Conferencia de la Asociación Española para la Inteligencia Artificial), en la que primero se utiliza el filtro de umbrales locales ya visto en esta memoria para detectar regiones candidatas a albergar defecto, sobre las cuales se aplican después los cálculos de características, sin utilizar ventanas. Esta aproximación ha demostrado ser muy rápida, pero en ocasiones no se comporta bien. Por ello, se podría incluir en el proyecto y, de forma inteligente, determinar cuál es el mejor método a utilizar.
\item Adaptación de la aplicación para que pueda ser ejecutada en un supercomputador.
\item Otras aplicaciones: Se podría intentar utilizar las técnicas utilizadas en este proyecto para resolver otros problemas parecidos, como por ejemplo el descrito en el artículo \emph{Automated fish bone detection using X-ray imaging} \cite{mery2011automated}, también de Domingo Mery. En este trabajo se utiliza la misma metodología para detectar espinas de pescado.
\end{itemize}



% APÉNDICES %%%%%%%%%%%%%%%%%%%%%%%%%%%%%%%%%%%%%%%%%%%%%%%%%%%%%%%%%%%%%%
\backmatter
\appendix

\portadasAuxiliares{Anexo I - Plan del proyecto software}
%%%%%%%%%%%%%%%%%%%%%%%%%%%%%%%%%%%%%%%%%%%%%%%%%%%%%%%%%%%%%%%%%%
%%%%%%%%%%%%%%%%%%%%%%%%%%%%%%%%%%%%%%%%%%%%%%%%%%%%%%%%%%%%%%%%%%
\chapter{Plan del proyecto software}
%%%%%%%%%%%%%%%%%%%%%%%%%%%%%%%%%%%%%%%%%%%%%%%%%%%%%%%%%%%%%%%%%%
%%%%%%%%%%%%%%%%%%%%%%%%%%%%%%%%%%%%%%%%%%%%%%%%%%%%%%%%%%%%%%%%%%

%Introducción
\section{Introducción}
En este anexo se detalla el estudio desde el punto de vista temporal y de la viabilidad del proyecto software.

La planificación es una de las tareas más importantes en el desarrollo de un proyecto software y servirá para determinar objetivos, evaluar la viabilidad del proyecto, priorizar actividades\dots

En la primera parte del anexo se detallará la planificación temporal del proyecto teniendo en cuenta la metodología ágil que se va a utilizar: \scrum{}. En esta fase se determinarán los elementos que forman el \productbacklog{} y la prioridad de cada uno de ellos.

Debido a la metodología empleada, no se utilizará el clásico diagrama de \textit{GANTT}. En lugar de esto, se definirá el \productbacklog{} y para el seguimiento se utilizará una herramienta de gestión especializada en metodologías ágiles, \textit{PivotalTracker}, que permite el seguimiento diario de las tareas por parte del equipo de desarrollo.

En la segunda parte se calcularán los costes, analizando la rentabilidad del proyecto y justificando su desarrollo desde diversos puntos de vista: viabilidad técnica, legal, económica\dots
\newpage



%Planifiación temporal del proyecto
\section{Planificación temporal del proyecto} 
Como se explicó en la memoria, para el desarrollo del proyecto de final de carrera se va, a utilizar una metodología ágil llamada \scrum{}. Esta metodología establece una serie de prácticas que serán llevadas a cabo con algunas limitaciones debido al reducido tamaño del equipo de desarrollo.

Para poder estimar de manera general el tiempo total que va a llevar el desarrollo del proyecto se va a realizar una estimación a partir de los casos de uso definidos en el Anexo II.

\newcommand{\pesoscasosdeuso}{\operatorname{Peso De Casos De Uso}}
\newcommand{\pesosactores}{\operatorname{Peso De Actores}}

\subsection{Estimación temporal a partir de casos de uso}
Antes de mostrar la tabla de estimación temporal, conviene repasar una serie de fórmulas que se utilizan para calcular algunos valores de la tabla:
\begin{itemize}
 \item Puntos de casos de uso no ajustados:
  \[ UUCP = \pesosactores + \pesoscasosdeuso \]
 \item Peso de los casos de uso:
  \[ \pesoscasosdeuso = \sum_{i = 0}^{i} Factor_{i} \]
 \item Peso de los factores técnicos:
  \[ TFC = 0,6 + 0,01 \cdot \sum_{i = 0}^{i} Factor_{i} \cdot Peso_{i} \]
 \item Factores de entorno:
  \[ EF = 1,4 + (-0,03) \cdot \sum_{i = 0}^{i} Factor_{i} \cdot Peso_{i} \]
\end{itemize}

A continuación \vertabla{tablaEstimacionTemporalCasosDeUso}, aparece detallada la duración estimada del desarrollo del proyecto a partir de los casos de uso identificados. Se encuentra dividida en:
\begin{itemize}
 \item Puntos de casos de uso no ajustados: sirven para conocer la envergadura del proyecto tomando como referencia los casos de uso.
 \item Factores técnicos: cuantifica la dificultad del proyecto en función de sus características internas.
 \item Factores del entorno: sirven para valorar lo familiarizado que se encuentra el equipo de desarrollo con proyectos de este tipo.
\end{itemize}

% Estimación temporal a partir de casos de uso.
\tablaSmallSinColores{Estimación temporal a partir de casos de uso}{p{8cm} c c c}{tablaEstimacionTemporalCasosDeUso}{
  \multicolumn{1}{c}{Factor} & \multicolumn{1}{c}{Peso} & \multicolumn{1}{c}{$ F_{i} $} & \multicolumn{1}{c}{Total} \\
 }
 {
  Actores simples                                              & 0  & 1   & 0    \\
  Actores medios                                               & 0  & 2   & 0    \\
  Actores complejos                                            & 1  & 3   & 3    \\
  \rowcolor[gray]{.8} Total peso actores                       &    &     & 3    \\
  Casos de uso simples                                         & 5  & 5   & 25   \\
  Casos de uso medios                                          & 1  & 10  & 10   \\
  Casos de uso complejos                                       & 2  & 15  & 30   \\
  \rowcolor[gray]{.8} Total peso casos de uso                  &    &     & 65   \\
  \rowcolor[gray]{.8} UUCP (Puntos de caso de uso no ajustados)&    &     & 68   \\
  Sistema distribuido                                          & 1  & 2   & 2    \\
  Tiempos de respuesta críticos                                & 1  & 1   & 1    \\
  En línea                                                     & 1  & 1   & 1    \\
  Procesos internos complejos                                  & 5  & 1   & 5    \\
  El código debe ser reutilizable                              & 3  & 1   & 3    \\
  Fácil de instalar                                            & 1  & 0,5 & 0,5  \\
  Fácil de utilizar                                            & 5  & 0,5 & 2,5  \\
  Portable                                                     & 1  & 2   & 2    \\
  Fácil de modificar                                           & 4  & 1   & 4    \\
  Concurrencia                                                 & 1  & 1   & 1    \\
  Incluye características de seguridad                         & 1  & 1   & 1    \\
  Acceso a software creado por otras compañías                 & 1  & 1   & 1    \\
  Incluye facilidades de aprendizaje para usuario              & 5  & 1   & 5    \\
  \rowcolor[gray]{.8} TF (Factores técnicos)                   &    &     & 0,89 \\
  Familiarizado con \scrum{}                                   & 1  & 0,5 & 0,5  \\
  Experiencia en este tipo de aplicaciones                     & 2  & 1   & 2    \\
  Experiencia en Orientación a Objetos                         & 5  & 0,5 & 2,5  \\
  Capacidad de liderazgo del analista                          & 5  & 1   & 5    \\
  Motivación                                                   & 5  & 1   & 5    \\
  Requisitos estables                                          & 3  & 2   & 6    \\
  Trabajadores a tiempo parcial                                & 5  & -1  & -5   \\
  Lenguaje de programación difícil de utilizar                 & 1  & -1  & -1   \\
  \rowcolor[gray]{.8} EF (Factores de entorno)                 &    &     & 0,95 \\
 }

Los puntos de casos de uso se calculan según la siguiente fórmula:
\[ UCP = UUCP \cdot TF \cdot EF \]

Para calcular los puntos de casos de uso hay que sustituir, en la fórmula anterior, con los valores de la tabla, es decir:
\[ UCP = 68 \cdot 0,89 \cdot 0,95 = 57,494 \]

Para obtener la duración del proyecto estimado según los casos de uso hay que multiplicar el valor de \textit{UCP} por un factor que depende del número de factores de entorno (\textit{EF}) a los cuales se les haya dado peso 0. En este caso, como no se ha dado ningún valor cero se multiplica por 20.

Por lo expuesto anteriormente: $ Nº horas = 20 \cdot 57,494 $, es decir, 1150 horas/hombre. [CAMBIAR]

Conviene destacar que los resultados obtenidos son meramente orientativos, ya que se va a utilizar una metodología ágil, por lo que la planificación va a ser a nivel de \sprint{} (de 15 a 30 días). Al comenzar cada uno de ellos, se definirán una serie de tareas que deberán ser completadas en dicho \sprint{}.

El proyecto comenzará en noviembre de 2012 y se desea finalizar en junio de 2013, es decir, se trabajará durante 7 meses. Se deben realizar los cálculos de horas para comprobar si es viable el desarrollo en las fechas previstas:
\newcommand{\tiempo}{\operatorname{Tiempo}}
\newcommand{\horas}{\operatorname{horas}}
\newcommand{\persona}{\operatorname{persona}}
\newcommand{\jornada}{\operatorname{jornada}}
\newcommand{\jornadas}{\operatorname{jornadas}}
\newcommand{\mes}{\operatorname{mes}}
\newcommand{\meses}{\operatorname{meses}}
\newcommand{\semana}{\operatorname{semana}}
\newcommand{\hora}{\operatorname{hora}}

\[
 \tiempo = \dfrac{1150 \horas}{1 \persona} \cdot \dfrac{1 \jornada}{7 \horas} \cdot \dfrac{1 \mes}{20 \jornadas} = 8,21 \meses
\]

El análisis de casos de uso junto con el horario marcado hace que la fecha de finalización deba trasladarse hasta julio para poder cumplir los plazos. De este modo la fecha de finalización del proyecto se retrasa hasta julio de 2010, manteniendo la jornada laboral de 35 horas semanales.

[TODO ESTO HAY QUE CAMBIARLO, PERO NOS SIRVE DE PLANTILLA]
\newpage


%Aplicando una metodología ágil: SCRUM
\section{Aplicando una metodología ágil: \textit{SCRUM}}
En esta sección se repasa de una manera rápida los principales aspectos del desarrollo de proyectos con \scrum{}. Al mismo tiempo se analizan los distintos roles que va a asumir cada uno de los participantes en el proyecto de final de carrera.

Conviene aclarar que los términos que define la metodología \scrum {} se encuentran en inglés, han sido mantenidos en el idioma original para no confundir con las traducciones.

\scrum{} \citeotras{scrum} es una metodología para la gestión y desarrollo de proyectos software basada en un proceso iterativo e incremental. Cada iteración termina con una pieza de software ejecutable que incorpora una nueva funcionalidad o mejora las ya existentes. Estas iteraciones suelen durar de dos a cuatro semanas.

\scrum{} busca priorizar los trabajos que mayor valor aportan al negocio evitando, en la medida de lo posible, complejos manuales de documentación que no tengan utilidad en el proceso.

Los requerimientos y prioridades se revisan y ajustan durante el proyecto en intervalos cortos y regulares. De esta forma es sencillo adaptarse a los cambios solicitados por el cliente y responder de una forma rápida a los mismos.

\subsection{Actores}
\scrum{} define un conjunto de roles o actores detallados a continuación:
\begin{itemize}
 \item \textit{Scrum Master} (o facilitador): coordina el desarrollo del proyecto y trabaja de manera similar al director de proyectos. Una de sus tareas es eliminar los obstáculos que puedan dificultar al equipo de desarrollo o \textit{team} la consecución de sus objetivos. En el proyecto, este rol lo han tomado los autores del mismo: Adrián González y Joaquín Bravo.
 \item \textit{Scrum Team} (o equipo): incluye a los desarrolladores y son los encargados de realizar las tareas que se definen en cada \sprint{}. En el proyecto, este rol lo han tomado los autores del mismo.
 \item \textit{Product Owner} (o cliente): representa la voz del cliente y aporta la visión del negocio. Representa el destinatario final del proyecto a desarrollar y el que, en última instancia, realiza las pruebas de aceptación. Es también el encargado de mantener al día el listado de las tareas o \productbacklog{} y sus prioridades. En el proyecto este rol sería tomado por la Universidad de Burgos, en concreto, personalizado en los tutores del proyecto César I. García Osorio y José Francisco Díez Pastor.
\end{itemize}

\subsection{Ciclo de desarrollo}
Al inicio del proyecto se definen una serie de requisitos que serán los objetivos a cumplir. Todos ellos quedan reflejados en el \productbacklog{}. Cada uno de estos objetivos serán subdivididos en tareas pequeñas y atómicas al inicio de cada \sprint{}.

El ciclo de desarrollo, como se muestra \ver{scrumProcess}, es iterativo a nivel de \sprint{}, al inicio del mismo se extraen una serie de tareas de los elementos del \productbacklog{} que conforman el \sprintbacklog{}. Estas tareas deberán ser completadas durante el ciclo.

%Ciclo de desarrollo con Scrum.
\figura{1}{imgs/ScrumProcess.png}{Metodología Scrum}{ScrumProcess}{}

\scrum{} define que debe realizarse una breve reunión diaria \textit{Daily Meeting} en la que, cada miembro del equipo de desarrollo, explica lo que ha hecho desde la última sincronización, que va a hacer a partir de ese momento y las dificultades encontradas o que espera encontrar. En este caso, debido a que el equipo de desarrollo está formado por un único desarrollador, se va a omitir esta reunión y se sustituye por una pequeña reflexión interior en la que analizar las tareas a desarrollar durante el día.

Los \sprints{} se definirán, como establece \scrum{}, con una duración de dos a cuatro semanas que podrán alargarse o contraerse para coincidir con los tutores para las reuniones periódicas que establece:
\begin{itemize}
 \item \textit{Planning meeting}: reunión inicial de cada \sprint{} en la que se extraen los \textit{item backlog}. Se numeran las tareas extraídas del \productbacklog{} para desarrollar durante el \sprint{}. Esta reunión tiene una gran importancia ya que, durante el desarrollo del \sprint{}, no se podrá modificar ni añadir nuevas tareas.
 \item \textit{Review meeting}: reunión final (4 horas máximo) de cada \sprint{} donde se detallan los objetivos cumplidos durante el mismo. Se muestra al usuario el producto, en caso de que sea posible. En función de los resultados mostrados y de los cambios que haya habido en el contexto del proyecto, el cliente realiza las adaptaciones necesarias de manera objetiva, ya desde la primera iteración, replanificando el proyecto.
 \item \textit{Retrospective meeting}: en esta reunión (4 horas máximo), el equipo analiza cómo ha sido su manera de trabajar y cuáles son los problemas que podrían impedirle progresar adecuadamente, mejorando de manera continua su productividad. El \textit{Scrum Master}, o \textit{Facilitador}, se encargará de ir eliminando los obstáculos identificados.
\end{itemize}

En este caso, todas las reuniones requieren de la presencia de los tutores y del equipo de desarrollo ya que entre ambos, agrupan los roles que se ven implicados en dichas reuniones.
\newpage



%Product backlog
\section{Product Backlog del proyecto}
En esta sección, aparece redactado el \productbacklog{} \vertabla{productBacklog} o lista de objetivos, con la prioridad de cada una de los \productitem{} y el \textit{sprint} al que pertenecen. Hemos decidido incluir también los \textit{bugs} que se han ido localizando[INSERTAR CITA].

\scrum{} no define un método o herramienta para llevar el día a día del trabajo realizado, por lo que el seguimiento se puede realizar con una simple hoja de cálculo, o con herramientas de gestión especializadas en este tipo de desarrollos.

En este caso, se ha utilizado \textit{PivotalTracker} (\url{https://www.pivotaltracker.com/}) para el seguimiento. Es un software potente al que se accede mediante un navegador y sirve de control de tareas y defectos. Su licencia ha sido gratuita por tratarse de un proyecto público.

%Tabla que contiene el Product Backlog
\tabla{Product Backlog}{c p{7.5cm} p{2cm} c c}{4}{productBacklog}{
  \multicolumn{1}{c}{ID} & \multicolumn{1}{c}{Backlog Item} & \multicolumn{1}{c}{Propietario} & \multicolumn{1}{c}{Prioridad} & \multicolumn{1}{c}{Sprint} \\
 }
 {
  S-01001 & Se debe poder cargar una imagen y sacar información de sus píxeles & César, José & Alta & Sprint 1 \\
  S-01002 & Se debe poder analizar una imagen y extraer sus características & César, José & Alta & Sprint 2  \\
  S-01003 & La aplicación debe ser capaz de calcular una imagen Saliency Map & César, José & Alta & Sprint 3 \\
  S-01004 & La aplicación debe poder calcular las características de Haralick & César, José & Alta & Sprint 3 \\
  S-01005 & La aplicación debe ser capaz de generar ficheros ARFF & César, José & Alta & Sprint 3 \\
  S-01006 & La aplicación debe ser capaz de calcular los eigenvalues mediante weka.Matrix o COLT & César, José & Baja & Sprint 4 \\
  S-01007 & La aplicación debe ser capaz de dividir la imagen según los hilos disponibles, con solapamiento entre las mismas  & César, José & Alta & Sprint 4  \\
  S-01008 & La aplicación debe ser capaz de realizar la convolución de la imagen completa, para extraer la media & César, José & Media & Sprint 4  \\
  S-01009 & La aplicación debe ser capaz de albergar varios tipos de ventana & César, José & Alta & Sprint 4  \\
  S-01010 & La aplicación debe ser capaz de entrenar un clasificador y clasificar & César, José & Alta & Sprint 4  \\
  S-01011 & La aplicación debe ser capaz de mostrar un GUI con las principales opciones & César, José & Alta & Sprint 4  \\
  S-01012 & La aplicación debe mostrar el progreso en una Progress Bar & César, José & Baja & Sprint 5  \\
  S-01013 & La aplicación debe mostrar un aviso si no se selecciona una región de la imagen a analizar & César, José & Baja & Sprint 5  \\
  S-01014 & La ventana de resultados se debe limpiar después de cada análisis & César, José & Media & Sprint 5  \\
  S-01015 & La aplicación debe escribir un ARFF por cada hilo, uniéndolos al terminar el proceso & César, José & Alta & Sprint 5  \\
  S-01016 & La aplicación debe mostrar eventos de funcionamiento en la interfaz & César, José & Media & Sprint 5  \\
  S-01017 & La aplicación debe guardar y cargar en un fichero las opciones & César, José & Media & Sprint 5  \\
  S-01018 & La aplicación debe ser capaz de guardar un log de errores & César, José & Baja & Sprint 5  \\
  S-01019 & La aplicación debe usar el filtro saliency para calcular características para entrenar y clasificar & César, José & Alta & Sprint 5  \\
  S-01020 & La aplicación debe usar la convolución de la imagen completa, en el caso de que no se elija región, o toda la región más un margen & César, José & Alta & Sprint 5  \\
  S-01021 & La aplicación debe poder dibujar los defectos encontrados & César, José & Alta & Sprint 5  \\
  S-01022 & La aplicación debe ser capaz de usar varios tipos de métodos para determinar si una ventana es defectuosa o no & César, José & Alta & Sprint 5  \\
  S-01023 & La aplicación debe mostrar un slider para seleccionar el umbral de detección & César, José & Alta & Sprint 5  \\
  S-01024 & La aplicación debe ser capaz de escribir texto formateado en HTML en el panel de log & César, José & Baja & Sprint 6  \\
  S-01025 & La aplicación debe ser capaz de exportar el log en formato HTML & César, José & Baja & Sprint 6  \\
  S-01026 & La aplicación debe tener un menú con varias opciones & César, José & Media & Sprint 6  \\
  S-01027 & La aplicación debe usar REPTree para regresión lineal & César, José & Media & Sprint 6  \\
  S-01028 & La aplicación debe ser capaz de calcular filtros de umbrales locales y comprobar si las regiones candidatas están dentro de la máscara & César, José & Alta & Sprint 6  \\
  S-01029 & La aplicación debe dirigir la búsqueda de defectos usando filtros de umbrales locales & César, José & Alta & Sprint 6  \\
  B-01001 & BUG: la primera y segunda derivadas deben estar bien calculadas  & César, José & Alta & Sprint 6  \\
  S-01030 & La aplicación debe dar la oportunidad al usuario de elegir el proceso de detección de defectos & César, José & Media & Sprint 7  \\
  S-01031 & La aplicación debe calcular los umbrales locales con un pequeño margen & César, José & Media & Sprint 7  \\
  S-01032 & La aplicación debe ser capaz de generar un conjunto de datos a partir de la ventana deslizante & César, José & Alta & Sprint 7  \\
  S-01033 & La aplicación debe poder remuestrear la lista de píxeles blancos, de tal manera que no considere todos & César, José & Alta & Sprint 7  \\
  S-01034 & La aplicación debe ser capaz de calcular características geométricas sobre las regiones segmentadas & César, José & Alta & Sprint 7  \\
  S-01035 & Se debe poder mostrar una tabla resumen de los defectos & César, José & Alta & Sprint 7  \\
  S-01036 & La aplicación debe permitir al usuario qué tipo de clasificación se va a usar: clases nominales o regresión & César, José & Media & Sprint 8  \\
  S-01037 & La aplicación debe añadir un borde a la selección en aquellas zonas que no coincidan con los bordes de la imagen & César, José & Media & Sprint 8  \\
  S-01038 & La aplicación debe permitir al usuario qué tipo de ventana defectuosa quiere usar  & César, José & Media & Sprint 8  \\
  S-01039 & Se debe añadir un módulo de ayuda en línea (JavaHelp), accesible desde el menú y con teclas rápidas (F1)  & César, José & Media & Sprint 8  \\
  S-01040 & La aplicación debe poder permitir seleccionar una fila de la tabla de resultados e iluminar el defecto correspondiente  & César, José & Alta & Sprint 8  \\
  S-01041 & La aplicación debe dar la opción de elegir si se calculan todas las características o sólo mejores  & César, José & Media & Sprint 8  \\
  B-01002 & BUG: la aplicación debe impedir la reselección mientras se analiza  & César, José & Alta & Sprint 8  \\
 }
\newpage


%Planificación por Sprint
\section{Planificación por Sprint}
En esta sección aparece el listado de los diversos \sprints{} del proyecto. La información de cada uno está dividida del siguiente modo:
\begin{itemize}
 \item \textit{Planning meeting}: acta de la reunión inicial de cada \sprint{} en la que se detallan los objetivos a cumplir durante el transcurso del mismo.
 \item \textit{Sprint planning}: se numeran las tareas extraídas del \productbacklog{} para desarrollar durante el \sprint{}.
 \item \textit{Burndown chart}: gráfico que muestra la evolución a lo largo del \sprint{} en comparación a la línea óptima.
 \item \textit{Retrospective meeting}: acta de la reunión retrospectiva realizada al final del \sprint{} y donde se analizan los problemas detectados durante su ejecución y las desviaciones (en caso de haberlas).
\end{itemize}

Debido a la falta de conocimiento inicial y a la incertidumbre sobre el desarrollo, se inicia el proyecto con \sprints{} de corta duración, aproximadamente dos semanas. Se ha intentado mantener siempre esta duración, aunque a veces nos ha podido el optimismo y hemos metido demasiado trabajo o no se han tenido en cuenta ciertos problemas de tiempo.


\portadasAuxiliares{Anexo II - Especificación de requisitos}
%%%%%%%%%%%%%%%%%%%%%%%%%%%%%%%%%%%%%%%%%%%%%%%%%%%%%%%%%%%%%%%%%%
%%%%%%%%%%%%%%%%%%%%%%%%%%%%%%%%%%%%%%%%%%%%%%%%%%%%%%%%%%%%%%%%%%
\chapter{Especificación de requisitos}
%%%%%%%%%%%%%%%%%%%%%%%%%%%%%%%%%%%%%%%%%%%%%%%%%%%%%%%%%%%%%%%%%%
%%%%%%%%%%%%%%%%%%%%%%%%%%%%%%%%%%%%%%%%%%%%%%%%%%%%%%%%%%%%%%%%%%

%Introducción
\section{Introducción}
Este anexo tiene como objetivo analizar y documentar las necesidades funcionales que deberán ser soportadas por el sistema a desarrollar. Para  ello, hay que identificar los requisitos que debe satisfacer el nuevo sistema, el estudio de los problemas de las unidades afectadas y sus necesidades actuales.

El objetivo de esta fase es describir lo que el sistema deberá ser capaz de hacer pero no cómo debe hacerlo.

Los tutores del proyecto toma un papel fundamental en este punto del proyecto, ya que asume el rol de cliente, es el encargado de especificar la funcionalidad que la aplicación debe implementar y establecer los plazos para lograrla.

Otra de las tareas a llevar a cabo en el presente anexo es la de priorizar cada una de las funcionalidades a implementar. Teniendo claras cuales son las tareas más críticas se podrá prestar un especial interés en ellas y asignar una cantidad mayor de recursos.

Hay que destacar la importancia de este documento, ya que sirve como punto de partida para el resto del trabajo. Por eso hay que hacer un esfuerzo extraordinario a la hora de definir y declarar correctamente todos los artefactos de ingeniería de software, para no tener que volver sobre nuestros pasos en etapas posteriores del trabajo. En primer lugar y de forma general, se recogen los objetivos que se quieren alcanzar con este trabajo. Estos objetivos son de muy alto nivel y estarán expresados por el cliente en lenguaje natural, siendo tarea del ingeniero su desglose y formalización. A continuación se detalla una lista de los usuarios que han participado en la toma de requisitos así como su relación con el sistema a construir.

Además, se incluye un catálogo con los requisitos que deberá cumplir el sistema a construir, junto con una breve explicación de cada uno, su tipo y su importancia. De estos requisitos se puede confeccionar la lista de los usuarios que interactúan con él, llamados actores. Se debe conocer el perfil de los usuarios para los que está destinada la aplicación, ya que esto condicionará en el futuro muchos aspectos del diseño, como por ejemplo las interfaces o la ayuda.
\newpage



%Objetivos del proyecto
\section{Objetivos del proyecto}\label{obj}
El objetivo principal de este proyecto es el de mejorar la aplicación en diversos aspectos, entre los que se encuentran:

\begin{itemize}
\item Mejora del rendimiento, mediante inclusión de programación multihilo y sustitución de algunos cálculos por otros más eficientes.
\item Mejora de la precisión en la detección de defectos, mediante la implementación de nuevos enfoques.
\item Mejora de la interfaz gráfica, haciéndola más intuitiva y funcional.
\item Mejora del diseño arquitectónico, enfocándolo hacia el mantenimiento y ampliación y solucionando fallos.
\item Mejora del código.
\item Implementación de nuevas funcionalidades, como el cálculo de características geométricas.
\end{itemize}

El proyecto, como vemos, está basado en uno anterior, pero se han tenido que cambiar un gran número de aspectos. Tenemos una parte muy orientada al mantenimiento, como es la mejora de todos los aspectos que ya existen, y otra de adición de nuevas funcionalidades.

Las nuevas funcionalidades incluyen la detección de defectos usando nuevos enfoques, es decir, mediante el uso de los filtros de umbrales locales y nuevas formas de determinar cuándo una ventana es defecto, así como una detección de defectos más interactiva, permitiendo al usuario interactuar con los defectos dibujados. También se incluye la implementación del cálculo de características geométricas y la inclusión de estos cálculos en una tabla interactiva. Estos cálculos permitirán, de una forma sencilla, incluir una clasificación de los defectos en tipos.
\newpage



%Usuarios participantes
\section{Usuarios participantes}
Durante la fase de análisis han participado diversos usuarios, cada uno desempeñando uno o varios papeles. De este modo han sido establecidos los requisitos necesarios para el desarrollo del proyecto software. Cabe destacar la participación de los tutores de este proyecto.
\begin{itemize}
 \item José Francisco Díez Pastor y César Ignacio García Osorio, tutores del proyecto, han asumido diversos roles durante el proceso:
 \begin{itemize}
  \item Como \textit{cliente}, han participado en la fase de análisis describiendo las funcionalidades y el comportamiento del sistema a desarrollar. Asimismo, han organizado el calendario y marcado los plazos de entrega.
  \item Como técnicos, ofreciendo sus conocimientos sobre la minería de datos. Además, César ha aportado sus conocimientos de LaTeX y experiencia en la realización de proyectos previos, y José su experiencia con la biblioteca Weka y con la programación en Java y sus conocimientos en las técnicas de procesamiento digital de imagen.
 \end{itemize}
 \item Por último, Adrián González Duarte y Joaquín Bravo Panadero han asumido el rol de \textit{analista}. Este rol agrupa las siguientes responsabilidades: analizar y describir el problema planteado por el cliente y realizar el diseño con la solución propuesta.
\end{itemize}
\newpage

\section{Descripción del sistema actual}
En este apartado se pretende hacer referencia a la aplicación en la que hemos integrado las funcionalidades descritas anteriormente.

El proyecto en el que está basada nuestra herramienta es \textbf{X-Ray Detector: Detección de defectos en piezas metálicas mediante análisis de radiografías}, por Alan Blanco Álamo y Víctor Barbero García, junio de 2012. Este proyecto incluía la funcionalidad de entrenamiento, detección y dibujado de defectos que tiene nuestros proyecto. En un principio, se pensó en simplemente refactorizar su código e ir incluyendo sobre él las nuevas modificaciones, pero al poco tiempo se vio que esto no era del todo sencillo, ya que el diseño y el propio código no eran de la calidad esperada (diseño sin ningún tipo de patrón que facilite la inclusión de nuevas funcionalidades, defectos de código o \textit{bad smells}...), por lo que decidimos rediseñar la aplicación desde el principio prácticamente.

Se ha aprovechado todo lo que se ha podido del proyecto anterior, como son los cálculos de características, si bien se han cambiado un poco para permitir, por ejemplo, que puedan ser usadas a modo de librería. También se ha mejorado alguno de estos cálculos, bien usando funciones e ImageJ o bien usando otras librerías, como EJML.

Una vez rediseñada la aplicación, se ha procedido a incluir las nuevas funcionalidades o mejoras poco a poco sobre este nuevo diseño.

\newpage

%Factores de riesgo
\section{Factores de riesgo}
En este apartado se analizan las dificultades a las que se va a tener que enfrentar el desarrollo de este proyecto software. Identificar los riesgos servirá para estar alerta sobre los posibles problemas que puedan surgir y los retrasos que éstos desencadenen.

Los factores de riesgo que han sido identificados son:
\begin{itemize}
 \item Falta de conocimiento: la primera dificultad encontrada es el desconocimiento de los fundamentos teóricos sobre temas como análiss de imágenes, etc. 
 \item Complejidad de la documentación existente: la documentación disponible se basa en los artículos sobre los algoritmos a implementar y en algún que otro artículo de investigación, por lo que la labor de comprensión de información se presenta difícil.
 \item Complejidad de la aplicación del proyecto del año pasado: se hace difícil comprender el proyecto del año pasado, sobre todo en cuanto al código.
 \item Desconocimiento sobre herramientas: nunca habíamos trabajado con una herramienta como ImageJ, o con librerías complejas como EJML, lo que hace complicado su uso.
 \item Desconocimiento sobre la metodología de desarrollo: si bien es cierto que teníamos una buena base teórica al respecto, nunca habíamos podido poner en práctica los conocimientos sobre las metodologías ágiles, como Scrum, lo que hace algo complicado crear un \textit{backlog}, usar un \textit{tracker}, etc.
 \item Nuevo sistema de composición de texto: para la documentación se va a utilizar \LaTeX{} lo que provocará una carga extra al desarrollo de la memoria del proyecto. Se utilizará para lograr una apariencia más sólida y profesional.
\end{itemize}

Se hace patente que para la creación de la aplicación se requerirá de una gran labor de investigación y análisis que posibilite superar todos los riesgos identificados y que concluyan con la superación de los objetivos marcados en los plazos establecidos.
\newpage



%Catálogo de requisitos del sistema
\section{Catálogo de requisitos del sistema}
El objetivo de este apartado es definir de forma clara, precisa, completa y verificable todas las funcionalidades y restricciones del sistema a construir.

\subsection{Objetivos del proyecto}
Los objetivos salen, en parte, de los ya descritos en el apartado \ref{obj} y, por otra parte, de los objetivos de la aplicación del año pasado. A continuación, se especifican los objetivos en la correspondiente plantilla de objetivos:

\tablaSinCabecera{Objetivos}{p{3.5cm} p{10cm}}{2}{tablaObjetivos}{
  \rowcolor[gray]{.8}\textbf{OBJ - 01}       & \textbf{Abrir imagen} \\\hline
  Descripción                      & La aplicación debe ser capaz de abrir y mostrar una imagen en la aplicación.\\\hline
 \multirow{2}*{Autores} 		& Adrián González Duarte \\ 
                                & Joaquín Bravo Panadero\\\hline
  Importancia                      & Vital \\\hline
  Urgencia                      & Alta \\\hline
  Comentarios                      & Es uno de los objetivos principales de la aplicación y, por tanto, un objetivo general \\\\

  \rowcolor[gray]{.8}\textbf{OBJ - 02}       & \textbf{Entrenar Clasificador} \\\hline
  Descripción                      & La aplicación debe ser capaz de entrenar un clasificador para realizar posteriormente el proceso de detección de defectos.\\\hline
  \multirow{2}*{Autores} 		& Adrián González Duarte \\ 
                                & Joaquín Bravo Panadero\\\hline
  Importancia                      & Vital \\\hline
  Urgencia                      & Alta \\\hline
  Comentarios                      & Es uno de los objetivos principales de la aplicación y, por tanto, un objetivo general \\\\
  
  \rowcolor[gray]{.8}\textbf{OBJ - 03}       & \textbf{Analizar Imagen} \\\hline
  Descripción                      & La aplicación debe ser capaz de analizar imágenes y detectar defectos si los tuviera. \\\hline
  \multirow{2}*{Autores} 		& Adrián González Duarte \\ 
                                & Joaquín Bravo Panadero\\\hline
  Importancia                      & Vital \\\hline
  Urgencia                      & Alta \\\hline
  Comentarios                      & Es uno de los objetivos principales de la aplicación y, por tanto, un objetivo general \\\\
  
  \rowcolor[gray]{.8}\textbf{OBJ - 04}       & \textbf{Exportar Resultados} \\\hline
  Descripción                      & La aplicación debe ser capaz de exportar los resultados de las operaciones que ha ido realizando la aplicación a un log. \\\hline
 \multirow{2}*{Autores} 		& Adrián González Duarte \\ 
                                & Joaquín Bravo Panadero\\\hline
  Importancia                      & Alta \\\hline
  Urgencia                      & Media \\\hline
  Comentarios                      & El formato del informe a exportar será en HTML. \\\\
  
  \rowcolor[gray]{.8}\textbf{OBJ - 05}       & \textbf{Mostrar Características y Seleccionar Defecto} \\\hline
  Descripción                      & La aplicación debe ser capaz de mostrar las características de un defecto e identificarlo seleccionándolo en una tabla de resultados o directamente sobre la imagen analizada. \\\hline
 \multirow{2}*{Autores} 		& Adrián González Duarte \\ 
                                & Joaquín Bravo Panadero\\\hline
  Importancia                      & Alta \\\hline
  Urgencia                      & Media \\\hline
  Comentarios                      & Nos referimos a las características geométricas. \\\\
  }

\subsection{Descripción de los actores}
A continuación, se describen los actores, es decir, los usuarios externos que interactúan con el sistema.

\tablaSinCabecera{Actores}{p{3.5cm} p{10cm}}{2}{tablaActores}{
  \rowcolor[gray]{.8}\textbf{ACT - 01}       & \textbf{Usuario} \\\hline
  Versión                      & 1.0\\\hline
  \multirow{2}*{Autores} 		& Adrián González Duarte \\ 
                                & Joaquín Bravo Panadero\\\hline
  Descripción                      & Este actor representa al usuario que quiere utilizar la aplicación. \\\hline
  Comentarios                      & Persona física que interactúa con el sistema y realiza las operaciones pertinentes de entrenamiento de clasificadores y análisis de imágenes. \\\\
  }


\subsection{Funciones del producto}
Una de las características más importantes de la aplicación que pretende desarrollarse es la facilidad de uso. Dado que su empleo está orientado a convertirse en una herramienta empresarial que ayude a los operarios a identificar piezas defectuosas, debe ser fácil de usar y tener la versatilidad para poder entrenar nuevos conjuntos de imágenes de cualquier nueva pieza que pueda desarrollar la empresa.

La aplicación cuenta con un panel de Log que muestra al usuario datos sobre el progreso del proceso que se está ejecutando.

Para cargar y guardar los conjuntos de datos se usará el formato nativo de \weka{} (\arff{}).


\subsubsection{Características a calcular}
La aplicación deberá calcular las siguientes características, tanto para la imagen normal como para la imagen con Saliency Map aplicado.
\begin{itemize}
 \item Características estándar: 
 \begin{enumerate}
 \item Media
 \item Desviación estándar
 \item Primera derivada
 \item Segunda derivada
 \end{enumerate}
 \item Características de Haralick:
 \begin{enumerate}
 \item Segundo Momento Angular
 \item Contraste
 \item Correlación
 \item Suma de cuadrados
 \item Momento Diferencial Inveros
 \item Suma Promedio
 \item Suma de Entropías
 \item Suma de Varianzas
 \item Entropía
 \item Diferencia de Varianzas
 \item Diferencia de Entropías
 \item Medidas de Información de Correlación 1
 \item Medidas de Información de Correlación 2
 \item Coeficiente de Correlación
 \end{enumerate}
 \item Local Binary Patterns
\end{itemize}

Además, debe ser capaz de calcular las siguientes características geométricas sobre los defectos encontrados:
\begin{itemize}
\item Área
\item Perímetro
\item Circularidad
\item Redondez
\item Semieje mayor
\item Semieje menor
\item Ángulo
\item Distancia Feret
\end{itemize}


\subsubsection{Entrenar el clasificador}
Esta parte deberá posibilitar las siguientes funciones:
\begin{itemize}
 \item Se permitirá al usuario abrir un directorio que contenga radiografías etiquetadas y listas para entrenar.
  \item Se informará al usuario en caso de que el directorio no sea apropiado, ya sea porque el número de imágenes originales y máscaras no coinciden, sus nombres no son los mismos, etc.
 \item Una vez cargado el directorio, se podrá crear un fichero \arff{} que contendrá el cálculo de todas las características para las imágenes de dicho directorio. Este fichero podrá ser guardado donde elija el usuario.
 \item El usuario podrá decidir si quiere que la extracción de características se haga mediante regiones aleatorias o con una ventana deslizante que recorra toda la imagen.
 \item Durante el cálculo de características, se informará al usuario del progreso y de los pasos que se van ejecutando mediante un Log, que podrá ser guardado en el disco duro. Asimismo, una barra de progreso le indicará el porcentaje aproximado de proceso ejecutado.
 \item La ejecución del cálculo de características podrá ser detenida en cualquier momento a petición del usuario.
 \item El entrenamiento se puede realizar también con un \arff{} generado previamente.
\end{itemize}


\subsubsection{Usar el clasificador para detectar defectos}
En este apartado se incluyen las siguientes funciones:
\begin{itemize}
 \item El usuario podrá abrir una imagen para detectar sus defectos.
 \item Se podrá iniciar la detección de defectos, teniendo que cargar un modelo entrenado.
 \item Se podrá seleccionar un área de la imagen para detectar los defectos contenidos en él. Si no se selecciona un área, se debe avisar al usuario de que el proceso puede llevar mucho tiempo.
 \item La detección de defectos podrá ser detenida en cualquier momento a petición del usuario.
 \item Se mostrará la imagen con los defectos marcados en un panel y la información de las características geométricas en una tabla.
 \item Se podrá seleccionar un defecto en la imagen para identificar la fila de la tabla de características geométricas que contiene sus datos. La inversa, seleccionar una fila y que se ilumine el defecto, también debe ser posible.
\end{itemize}

\subsection{Requisitos de usuario}
El usuario debe ser capaz de utilizar la aplicación mediante la interfaz gráfica cargando imágenes, detectando defectos y visualizando los resultados.

La ayuda será un punto a tener en cuenta y en todo momento el usuario podrá solicitarla para orientarse y resolver dudas que le puedan surgir durante el manejo de la misma.


\subsection{Requisitos de sistema}
Al tratarse de una aplicación de escritorio será necesario contar con un hardware que tenga unos requisitos mínimos.

La aplicación se va a desarrollar en \java{} por lo que no debe ser muy exigente, es decir, cualquier máquina capaz de hacer correr la máquina virtual de \java{} debe ser capaz de ejecutar la aplicación.

No obstante, dependiendo del tamaño del fichero \arff{} que se vaya a cargar, o del número de características que se quieran calcular, será recomendable una mayor capacidad hardware para reducir el tiempo de ejecución.

Los requisitos mínimos serán:
\begin{itemize}
 \item Equivalente a 2.8 GHz.
 \item 2 GB de memoria RAM.
 \item Resolución de pantalla igual o superior a 1200 x 720
\end{itemize}

Mientras que los requisitos recomendados serán:
\begin{itemize}
 \item Equivalente a 2.8 GHz(doble núcleo).
 \item 4 GB de memoria RAM.
 \item Resolución de pantalla igual o superior a 1280 x 960
\end{itemize}

Aunque estos requisitos dependerán en gran medida de el número de instancias o características con el que se trabaje.

En cuanto al sistema operativo, deberá ser posible la ejecución en todos los que exista una versión de máquina virtual de \java{}. Uno de los beneficios de que el proyecto se implemente en un lenguaje interpretado es que maximiza las posibilidades de uso en distintos entornos.

La versión de \java{} sobre la que se desarrollará el proyecto es la 1.7 (o 7 directamente).
\newpage



%Especificación de los requisitos
\section{Especificación de los requisitos}
Esta sección del anexo recoge los casos de uso que representan la funcionalidad que debe cubrir la aplicación a desarrollar. Cada caso de uso es una descripción de una secuencia de acciones que se ejecutan en un sistema para producir la salida esperada por un actor.

El primer lugar quedan definidos los requisitos inherentes a la tecnología utilizada y después los requisitos de información. Con este primer análisis se detallan a continuación los requisitos funcionales, apoyados en los casos de uso, y por último los requisitos no funcionales.


\subsection{Especificación de requisitos inherentes a la tecnología utilizada}
Son los requisitos que vienen <<heredados>> por la tecnología a utilizar.

Se ha elegido el lenguaje de programación \java{} para el desarrollo del proyecto, concretamente la versión 7. El requisito necesario para poder ejecutar la aplicación será que el terminal tenga instalada una máquina virtual de \java{}.


\subsection{Especificación de requisitos de información}
El objetivo de este apartado es determinar la información que se debe almacenar para cumplir los objetivos anteriormente descritos y que el programa funcione.
Con objeto de identificar cada uno de los requisitos de información se les ha asignado un código único y un nombre descriptivo. A continuación \vertabla{tablaRequisitosInformacion} se detallan de manera tabulada cada uno de los requisitos en plantillas.

\tablaSinCabecera{Requisitos de información}{p{3.5cm} p{10cm}}{2}{tablaRequisitosInformacion}{
  \rowcolor[gray]{.8}\textbf{RI - 01}       & \textbf{Información cálculo de características} \\\hline
  Descripción                      & Almacena la información resultante del cálculo de características de una radiografía mediante ficheros \arff{}\\\hline
  Datos específicos                & Características calculadas a partir de una radiografía \\\hline
  Importancia                      & Alta \\\hline
  Comentarios                      & Esta información se utilizará para entrenar un clasificador \\\\

  \rowcolor[gray]{.8}\textbf{RI - 02}       & \textbf{Información de configuración de la ventana de análisis} \\\hline
  Descripción                      & Almacena la información necesaria para analizar la imagen mediante ventanas\\\hline
  \multirow{4}*{Datos específicos} & Tamaño de la ventana \\\cline{2-2} 
                                   & Salto entre ventanas \\\cline{2-2}
                                   & Tipo de ventana (aleatoria, deslizante) \\\cline{2-2}	
                                   & Cuándo una ventana es defectuosa (entrenamiento) \\\hline
  Importancia                      & Alta \\\hline
  Comentarios                      & Se almacena todo en un mismo fichero de configuración \\\\
  
  \rowcolor[gray]{.8}\textbf{RI - 03}       & \textbf{Información para el entrenamiento del clasificador o para el dibujado de características} \\\hline
  Descripción                      & Almacena la información necesaria para entrenar un clasificador o para dibujar características \\\hline
 Datos específicos				   & Fichero \arff{} generado al calcular las características de una radiografía  \\\hline
  Importancia                      & Alta \\\hline
  Comentarios                      & Es obligatorio para poder entrenar al clasificador\\\\
  
  \rowcolor[gray]{.8}\textbf{RI - 04}       & \textbf{Información para la detección de defectos} \\\hline
  Descripción                      & Almacena la información necesaria detectar los defectos de una radiografía \\\hline
 Datos específicos				   & Modelo entrenado que utilizará la aplicación para detectar los defectos de una radiografía  \\\hline
  Importancia                      & Alta \\\hline
  Comentarios                      & Es obligatorio para poder detectar los defectos \\\\
  }


\subsection{Requisitos funcionales}
La definición de los requisitos funcionales servirá de ayuda para el diseño de la herramienta.

En la lista aparecen todos los requisitos identificados en la aplicación:
\begin{itemize}
 \item Cargar imágenes: el usuario podrá cargar imágenes para detectar defectos.
 \item Detención del proceso: se podrá detener el proceso en cualquier momento.
 \item Cargar instancias: la aplicación debe estar capacitada para abrir conjuntos de instancias en el formato \arff{} soportado por \weka{}.
 \item Entrenar clasificador: se podrá entrenar un clasificador para detectar los defectos de nuevas radiografías, bien sea con un \arff{} ya creado o creando uno nuevo.
 \item Detectar defectos: se podrá realizar la detección de defectos de cualquier imagen que el usuario cargue, pudiendo seleccionar la región exacta de la imagen que se quiera analizar.
 \item Visualización de resultados: se podrán visualizar los defectos detectados, así como una tabla con las características geométricas.
 \item Interacción con los resultados: se podrá seleccionar un defecto y se iluminará la fila de la tabla correspondiente a ese defecto. Se podrá seleccionar también una fila de la tabla y se iluminará el defecto.
 \item Guardar Log: el usuario podrá visualizar el progreso en un log y guardarlo en el disco duro.
 \item Visualización de la ayuda: las interfaces deberán ofrecer la ayuda al usuario que necesite para poder utilizarlas.
 \item Guardar imagen analizada: se permite guardar los resultados del análisis en un fichero.
 \item Guardar imagen binarizada: se permite guardar una binarización de los defectos encontrados.
 \item Calcular <<Precision \& Recall>>: se realiza el cálculo de estas medidas de precisión, a través de una máscara.
\end{itemize}

Para la definición formal de los casos de uso se hace uso de los diagramas de casos de uso. Para completar la información de los diagramas se utilizan las plantillas donde quedan reflejados los aspectos funcionales.

En el primer diagrama \ver{DiagramaCasosDeUsoGeneralGUI} se muestra el diagrama de casos de uso principal.

%Diagrama general de casos de uso GUI.
\figura{0.9}{imgs/casosdeusogeneral.png}{Diagrama de casos de uso principal}{DiagramaCasosDeUsoGeneralGUI}{}




\newpage
A continuación, aparecen los casos de uso listados y cada una de las plantillas:
\begin{itemize}
 \item RF-01 Abrir imagen \vertabla{tablaRF1}
 \item RF-02 Entrenar clasificador \vertabla{tablaRF2}
 \item RF-03 Entrenar clasificador con ARFF existente \vertabla{tablaRF3}
 \item RF-04 Entrenar clasificador generando nuevo ARFF \vertabla{tablaRF4}
 \item RF-05 Analizar imagen \vertabla{tablaRF5}
 \item RF-06 Calcular características \vertabla{tablaRF6}
 \item RF-07 Seleccionar defecto \vertabla{tablaRF7}
 \item RF-08 Exportar Log \vertabla{tablaRF8}
 \item RF-09 Abrir ayuda \vertabla{tablaRF9}
 \item RF-10 Cambiar opciones \vertabla{tablaRF10}
 \item RF-11 Guardar imagen analizada \vertabla{tablaRF11}
 \item RF-12 Guardar imagen binarizada \vertabla{tablaRF12}
 \item RF-13 Calcular <<Precision \& Recall>> \vertabla{tablaRF13}
\end{itemize}

\tablaSmallSinColores{Caso de uso: RF-01 Abrir imagen}{p{3cm} p{.75cm} p{9.5cm}}{tablaRF1}{
  \multicolumn{3}{l}{\textbf{RF-01  Abrir imagen}} \\
 }
 {
  Versión								 & 1.0\\\hline
  \multirow{2}{3.5cm}{Autores} 		& \multicolumn{2}{p{10cm}}{Adrián González Duarte} \\
                                	& \multicolumn{2}{p{10cm}}{Joaquín Bravo Panadero}\\\hline
  Objetivos asociados					 & \multicolumn{2}{p{10cm}}{OBJ-01}\\\hline
  Descripción                            & \multicolumn{2}{p{10cm}}{Permite abrir una imagen en la aplicación y mostrarla al usuario para posteriormente ser analizada en busca de posibles defectos} \\\hline
  Precondiciones                         & \multicolumn{2}{p{10cm}}{Debe existir la imagen en el sistema} \\\hline
  \multirow{4}{3.5cm}{Secuencia normal}  & Paso & Acción \\\cline{2-3}
                                         & 1    & Seleccionar la opción de abrir imagen \\\cline{2-3}
                                         & 2    & El sistema pedirá al usuario que especifique que imagen quiere cargar en la aplicación a partir de un explorador de archivos \\\cline{2-3}
                                         & 3	& Se carga la imagen en la aplicación \\\hline
  Postcondiciones                        & \multicolumn{2}{l}{Imagen importada en la aplicación} \\\hline
  \multirow{2}{3.5cm}{Excepciones}       & Paso & Acción \\\cline{2-3}
                                         & 1    &  Si la operación se cancela el caso de uso finaliza\\\hline
  Rendimiento                            &      & \\\hline
  Frecuencia                             & \alta{2} \\\hline
  Importancia                            & \alta{2} \\\hline
  Urgencia                               & \alta{2} \\\hline
  Comentarios                            & \multicolumn{2}{l}{Va a permitir realizar más acciones, como analizar} \\
}

\tablaSmallSinColores{Caso de uso: RF-02 Entrenar clasificador}{p{3cm} p{.75cm} p{9.5cm}}{tablaRF2}{
  \multicolumn{3}{l}{\textbf{RF-02 Entrenar clasificador}} \\
 }
 {
  Versión								 & 1.0\\\hline
  \multirow{2}{3.5cm}{Autores} 		& \multicolumn{2}{p{10cm}}{Adrián González Duarte} \\
                                	& \multicolumn{2}{p{10cm}}{Joaquín Bravo Panadero}\\\hline
  Objetivos asociados					 & \multicolumn{2}{p{10cm}}{OBJ-02}\\\hline
  Descripción                            & \multicolumn{2}{p{10cm}}{Permite a la aplicación entrenar un clasificador a partir de un conjunto de imágenes o de un archivo ARFF existente} \\\hline
  Precondiciones                         & \multicolumn{2}{l}{Ninguna} \\\hline
  \multirow{4}{3.5cm}{Secuencia normal}  & Paso & Acción \\\cline{2-3}
                                         & 1    & El usuario selecciona la opción de entrenar clasificador  \\\cline{2-3}
                                         & 2    & El sistema pedirá al usuario cómo se va a entrenar el clasificador, mediante un ARFF existente o mediante un conjunto de imágenes \\
                                         & 3 	& El sistema entrena un clasificador\\\hline
  Postcondiciones                        & \multicolumn{2}{l}{El sistema entrena un clasificador} \\\hline
  \multirow{3}{3.5cm}{Excepciones}       & Paso & Acción \\\cline{2-3}
                                         & 2     & Si la operación se cancela en el proceso de selección del tipo de entrenamiento el caso de uso finaliza\\\cline{2-3}
                                         & 3     & Si el proceso se para a lo largo de su ejecución el caso de uso finaliza\\\hline
  Rendimiento                            &      & \\\hline
  Frecuencia                             & \media{2} \\\hline
  Importancia                            & \alta{2} \\\hline
  Urgencia                               & \alta{2} \\\hline
  Comentarios                            & \multicolumn{2}{p{10cm}}{Es una característica imprescindible, ya que sin un clasificador la aplicación no puede detectar defectos en una imagen posteriormente} \\
}

\tablaSmallSinColores{Caso de uso: RF-03 Entrenar clasificador con ARFF existente}{p{3cm} p{.75cm} p{9.5cm}}{tablaRF3}{
  \multicolumn{3}{l}{\textbf{RF-03 Entrenar clasificador con ARFF existente}} \\
 }
 {
 Versión								 & 1.0\\\hline
  \multirow{2}{3.5cm}{Autores} 		& \multicolumn{2}{p{10cm}}{Adrián González Duarte} \\
                                	& \multicolumn{2}{p{10cm}}{Joaquín Bravo Panadero}\\\hline
  Objetivos asociados					 & \multicolumn{2}{p{10cm}}{OBJ-02}\\\hline
  Descripción                            & \multicolumn{2}{p{10cm}}{Permite a la aplicación entrenar un clasificador a partir de un archivo ARFF existente} \\\hline
  Precondiciones                         & \multicolumn{2}{p{10cm}}{Debe existir un conjunto de imágenes en el sistema junto con su conjunto de máscaras} \\\hline
  \multirow{4}{3.5cm}{Secuencia normal}  & Paso & Acción \\\cline{2-3}
                                         & 1    & Seleccionar la opción de entrenar clasificador. Y el usuario especifica que quiere entrenar un clasificador a partir de un archivo ARFF existente \\\cline{2-3}
                                         & 2    & El sistema pedirá al usuario que especifique la ruta del archivo ARFF con el que se desea realizar el proceso de entrenamiento \\\cline{2-3} 							 							 & 3    & El sistema entrena un clasificador \\\hline
  Postcondiciones                        & \multicolumn{2}{p{10cm}}{Clasificador entrenado} \\\hline
  \multirow{3}{3.5cm}{Excepciones}       & Paso & Acción \\\cline{2-3}
                                         & 2    & Si la operación se cancela antes de seleccionar que se desea utilizar un ARFF existente el caso de uso finaliza \\\cline{2-3}
                                         & 3    & Si el proceso se para a lo largo de su ejecución el caso de uso finaliza \\\hline
  Rendimiento                            &      & \\\hline
  Frecuencia                             & \media{2} \\\hline
  Importancia                            & \alta{2} \\\hline
  Urgencia                               & \alta{2} \\\hline
  Comentarios                            & \multicolumn{2}{p{10cm}}{Es una característica imprescindible, ya que sin un clasificador la aplicación no puede detectar defectos en una imagen posteriormente} \\
}

\tablaSmallSinColores{Caso de uso: RF-04 Entrenar clasificador generando nuevo ARFF}{p{3cm} p{.75cm} p{9.5cm}}{tablaRF4}{
  \multicolumn{3}{l}{\textbf{RF-04 Entrenar clasificador generando nuevo ARFF}} \\
 }
 {
 Versión								 & 1.0\\\hline
  \multirow{2}{3.5cm}{Autores} 		& \multicolumn{2}{p{10cm}}{Adrián González Duarte} \\
                                	& \multicolumn{2}{p{10cm}}{Joaquín Bravo Panadero}\\\hline
  Objetivos asociados					 & \multicolumn{2}{p{10cm}}{OBJ-02}\\\hline
  Descripción                            & \multicolumn{2}{p{10cm}}{Permite a la aplicación entrenar un clasificador a partir de un conjunto de imágenes} \\\hline
  Precondiciones                         & \multicolumn{2}{p{10cm}}{Debe existir un conjunto de imágenes en el sistema junto con su conjunto de máscaras} \\\hline
  \multirow{5}{3.5cm}{Secuencia normal}  & Paso & Acción \\\cline{2-3}
                                         & 1    & Seleccionar la opción de entrenar clasificador. El usuario especificará que desea entrenar un clasificador a partir de un conjunto de imágenes \\\cline{2-3}
                                         & 2    & El sistema pedirá al usuario que especifique la carpeta que contiene las imágenes con las que se va a entrenar el clasificador \\\cline{2-3} 																 & 3    & El sistema inicia un proceso de extracción de características sobre cada una de las imágenes generando un archivo ARFF nuevo. Se inicia el caso de uso RF-06 Calcular características \\\cline{2-3} 
                                         & 4    & El sistema entrena un clasificador \\\hline 		
  Postcondiciones                        & \multicolumn{2}{l}{Clasificador entrenado} \\\hline
  \multirow{3}{3.5cm}{Excepciones}       & Paso & Acción \\\cline{2-3}
                                         & 2    & Si la operación se cancela antes de seleccionar que se desea generar un ARFF nuevo el caso de uso finaliza \\\cline{2-3}
                                         & 2    & Si el proceso se para a lo largo de su ejecución el caso de uso finaliza\\\hline
  Rendimiento                            &      & \\\hline
  Frecuencia                             & \media{2} \\\hline
  Importancia                            & \alta{2} \\\hline
  Urgencia                               & \alta{2} \\\hline
  Comentarios                            & \multicolumn{2}{p{10cm}}{Es una característica imprescindible, ya que sin un clasificador la aplicación no puede detectar defectos en una imagen posteriormente} \\
}

\tablaSmallSinColores{Caso de uso: RF-05 Analizar imagen}{p{3cm} p{.75cm} p{9.5cm}}{tablaRF5}{
  \multicolumn{3}{l}{\textbf{RF-05 Analizar imagen}} \\
 }
 {
 Versión								 & 1.0\\\hline
  \multirow{2}{3.5cm}{Autores} 		& \multicolumn{2}{p{10cm}}{Adrián González Duarte} \\
                                	& \multicolumn{2}{p{10cm}}{Joaquín Bravo Panadero}\\\hline
  Objetivos asociados					 & \multicolumn{2}{p{10cm}}{OBJ-03}\\\hline
  Descripción                            & \multicolumn{2}{p{10cm}}{Permite a la aplicación analizar una imagen en busca de posibles defectos} \\\hline
  Precondiciones                         & \multicolumn{2}{p{10cm}}{Debe existir un clasificador entrenado así como una imagen importada en la aplicación} \\\hline
  \multirow{2}{3.5cm}{Secuencia normal}  & Paso & Acción \\\cline{2-3}
                                         & 1    & Seleccionar la opción de entrenar analizar imagen y elección de clasificador \\\cline{2-3}
                                         & 2    & El sistema inicia el proceso de análisis de la imagen en busca de defectos. Se inicia en caso de uso RF-06 Calcular características \\\cline{2-3}
                                         & 3    & La aplicación muestra los resultados del análisis \\\hline
  Postcondiciones                        & \multicolumn{2}{p{10cm}}{Imagen analizada: defectos dibujados y características geométricas listadas} \\\hline
  \multirow{2}{3.5cm}{Excepciones}       & Paso & Acción \\\cline{2-3}
                                         & 1    & Si se cancelar la elección de clasificador, el casi de uso finaliza\\\cline{2-3}
                                         & 2	& Si el proceso se para a lo largo de su ejecución el caso de uso finaliza\\\hline
  Rendimiento                            &      & \\\hline
  Frecuencia                             & \alta{2} \\\hline
  Importancia                            & \alta{2} \\\hline
  Urgencia                               & \alta{2} \\\hline
  Comentarios                            & \multicolumn{2}{p{10cm}}{Es una característica imprescindible, ya que es el objetivo principal del proyecto} \\
}

\tablaSmallSinColores{Caso de uso: RF-06 Calcular características}{p{3cm} p{.75cm} p{9.5cm}}{tablaRF6}{
  \multicolumn{3}{l}{\textbf{RF-06 Calcular características}} \\
 }
 {
  Versión								 & 1.0\\\hline
  \multirow{2}{3.5cm}{Autores} 		& \multicolumn{2}{p{10cm}}{Adrián González Duarte} \\
                                	& \multicolumn{2}{p{10cm}}{Joaquín Bravo Panadero}\\\hline
  Objetivos asociados					 & \multicolumn{2}{p{10cm}}{OBJ-02, OBJ-03}\\\hline
  Descripción                            & \multicolumn{2}{p{10cm}}{Calcular las características de uma imagen} \\\hline
  Precondiciones                         & \multicolumn{2}{p{10cm}}{Debe haberse iniciado el caso de uso RF-04 o el RF-05} \\\hline
  \multirow{2}{3.5cm}{Secuencia normal}  & Paso & Acción \\\cline{2-3}
										 & 1 	& El sistema reciba una región a analizar \\\cline{2-3}                                         
                                         & 2    & El sistema calcula las características (todas o las mejores) sobre la región \\\cline{2-3}
                                         & 3    & Si se había iniciado el caso de uso RF-04, se generará una nueva línea en el ARFF. Si era el RF-05, se pasarán las características al clasificador, que dirá si corresponden con un defecto o no\\\hline
  Postcondiciones                        & \multicolumn{2}{l}{Características calculadas} \\\hline
  \multirow{2}{3.5cm}{Excepciones}       & Paso & Acción \\\cline{2-3}
                                         & 2    & Si hay algún problema al calcular, el sistema mostrará un error y el caso de uso finaliza \\\hline
  Rendimiento                            &      & \\\hline
  Frecuencia                             & \alta{2} \\\hline
  Importancia                            & \alta{2} \\\hline
  Urgencia                               & \alta{2} \\\hline
  Comentarios                            & \multicolumn{2}{p{10cm}}{Es una característica básica, pues forma parte tanto del entrenamiento de clasificadores como de la detección de defectos} \\
}

\tablaSmallSinColores{Caso de uso: RF-07 Seleccionar defecto}{p{3cm} p{.75cm} p{9.5cm}}{tablaRF7}{
  \multicolumn{3}{l}{\textbf{RF-07 Seleccionar defecto}} \\
 }
 {
 Versión								 & 1.0\\\hline
  \multirow{2}{3.5cm}{Autores} 		& \multicolumn{2}{p{10cm}}{Adrián González Duarte} \\
                                	& \multicolumn{2}{p{10cm}}{Joaquín Bravo Panadero}\\\hline
  Objetivos asociados					 & \multicolumn{2}{p{10cm}}{OBJ-05}\\\hline
  Descripción                            & \multicolumn{2}{p{10cm}}{Muestra al usuario los defectos detectados en una imagen tras ser analizada con sus características. Permite seleccionar un defecto dibujado para ver sus características.} \\\hline
  Precondiciones                         & \multicolumn{2}{p{10cm}}{Debe haber finalizado el caso de uso RF-05} \\\hline
  \multirow{2}{3.5cm}{Secuencia normal}  & Paso & Acción \\\cline{2-3}
                                         & 1    & Se ejecuta el caso de uso RF-05 \\\cline{2-3} 
                                         & 2    & La aplicación muestra los resultados del análisis indicando los defectos que ha encontrado junto con sus características en una tabla de resultados \\\cline{2-3} 
                                         & 3	& El usuario selecciona bien el defecto o bien unos resultados en la tabla y tanto el defecto como sus características asociadas se seleccionan para ser mejor identificados \\\hline
  Postcondiciones                        & \multicolumn{2}{p{10cm}}{Defectos detectados y seleccionados y características calculadas} \\\hline
  \multirow{2}{3.5cm}{Excepciones}       & Paso & Acción \\\cline{2-3}
                                         &      &  \\\hline
  Rendimiento                            &      & \\\hline
  Frecuencia                             & \alta{2} \\\hline
  Importancia                            & \alta{2} \\\hline
  Urgencia                               & \media{2} \\\hline
  Comentarios                            & \multicolumn{2}{p{10cm}}{En este caso de uso, nos estamos refiriendo a las características geométricas. Proporciona la interactividad con los resultados.} \\
}

\tablaSmallSinColores{Caso de uso: RF-08 Exportar Log}{p{3cm} p{.75cm} p{9.5cm}}{tablaRF8}{
  \multicolumn{3}{l}{\textbf{RF-08 Exportar Log}} \\
 }
 {
  Versión								 & 1.0\\\hline
  \multirow{2}{3.5cm}{Autores} 		& \multicolumn{2}{p{10cm}}{Adrián González Duarte} \\
                                	& \multicolumn{2}{p{10cm}}{Joaquín Bravo Panadero}\\\hline
  Objetivos asociados					 & \multicolumn{2}{p{10cm}}{OBJ-04}\\\hline
  Descripción                            & \multicolumn{2}{p{10cm}}{Permite al usuario exportar un log con los resultados de los procesos ejecutados a lo largo de toda la aplicación} \\\hline
  Precondiciones                         & \multicolumn{2}{l}{Ninguna} \\\hline
  \multirow{2}{3.5cm}{Secuencia normal}  & Paso & Acción \\\cline{2-3}
                                         & 1    & El usuario pulsa el botón <<Exportar Log>> \\\cline{2-3} 
                                         & 2    & El sistema genera un log en formato HTML que mostrará todos los procesos llevados a cabo por la aplicación junto con sus resultados \\\hline
  Postcondiciones                        & \multicolumn{2}{l}{Log exportado} \\\hline
  \multirow{2}{3.5cm}{Excepciones}       & Paso & Acción \\\cline{2-3}
                                         &      &  \\\hline
  Rendimiento                            &      & \\\hline
  Frecuencia                             & \baja{2} \\\hline
  Importancia                            & \media{2} \\\hline
  Urgencia                               & \baja{2} \\\hline
  Comentarios                            & \multicolumn{2}{p{10cm}}{} \\
}

\tablaSmallSinColores{Caso de uso: RF-09 Abrir ayuda}{p{3cm} p{.75cm} p{9.5cm}}{tablaRF9}{
  \multicolumn{3}{l}{\textbf{RF-09 Abrir ayuda}} \\
 }
 {
 Versión								 & 1.0\\\hline
  \multirow{2}{3.5cm}{Autores} 		& \multicolumn{2}{p{10cm}}{Adrián González Duarte} \\
                                	& \multicolumn{2}{p{10cm}}{Joaquín Bravo Panadero}\\\hline
  Objetivos asociados					 & \multicolumn{2}{p{10cm}}{}\\\hline
  Descripción                            & \multicolumn{2}{p{10cm}}{Permite visualizar la ayuda de la aplicación} \\\hline
  Precondiciones                         & \multicolumn{2}{l}{Ninguna} \\\hline
  \multirow{2}{3.5cm}{Secuencia normal}  & Paso & Acción \\\cline{2-3}
                                         & 1    & El usuario abre la ayuda, bien mediante una tecla especial o bien mediante una opción en un menú \\\cline{2-3}
                                         & 2    & Se mostrará la ayuda \\\hline
  Postcondiciones                        & \multicolumn{2}{p{10cm}}{Se mostrará una nueva ventana en la que se encuentra la ayuda} \\\hline
  \multirow{2}{3.5cm}{Excepciones}       & Paso & Acción \\\cline{2-3}
                                         &      &  \\\hline
  Rendimiento                            &      & \\\hline
  Frecuencia                             & \media{2} \\\hline
  Importancia                            & \media{2} \\\hline
  Urgencia                               & \baja{2} \\\hline
  Comentarios                            & \multicolumn{2}{l}{} \\
}

\tablaSmallSinColores{Caso de uso: RF-10 Cambiar opciones}{p{3cm} p{.75cm} p{9.5cm}}{tablaRF10}{
  \multicolumn{3}{l}{\textbf{RF-10 Cambiar opciones}} \\
 }
 {
 Versión								 & 1.0\\\hline
  \multirow{2}{3.5cm}{Autores} 		& \multicolumn{2}{p{10cm}}{Adrián González Duarte} \\
                                	& \multicolumn{2}{p{10cm}}{Joaquín Bravo Panadero}\\\hline
  Objetivos asociados					 & \multicolumn{2}{p{10cm}}{OBJ-02, OBJ-03}\\\hline
  Descripción                            & \multicolumn{2}{p{10cm}}{Permite cambiar algunas opciones de la aplicación que afectarán a los procesos de entrenamiento y detección} \\\hline
  Precondiciones                         & \multicolumn{2}{l}{Ninguna} \\\hline
  \multirow{2}{3.5cm}{Secuencia normal}  & Paso & Acción \\\cline{2-3}
                                         & 1    & El usuario selecciona <<opciones avanzadas>> \\\cline{2-3} 
                                         & 2    & El usuario determina cuáles de las opciones quiere cambiar y con qué valores \\\cline{2-3}
                                         & 3	& El sistema guarda los cambios en un fichero de propiedades\\\hline
  Postcondiciones                        & \multicolumn{2}{p{10cm}}{El fichero de propiedades contiene los nuevos cambios} \\\hline
   \multirow{2}{3.5cm}{Excepciones}       & Paso & Acción \\\cline{2-3}
                                         & 3    & Si se cancela el proceso, los cambios no se guardan\\\hline
  Rendimiento                            &      & \\\hline
  Frecuencia                             & \media{2} \\\hline
  Importancia                            & \alta{2} \\\hline
  Urgencia                               & \media{2} \\\hline
  Comentarios                            & \multicolumn{2}{p{10cm}}{Las opciones que se pueden cambiar están relacionadas con el tamaño de las ventanas y de su salto, el tipo de heurística de ventana defectuosa, el tipo de ventana para entrenar, el tipo de detección...} \\
}

\tablaSmallSinColores{Caso de uso: RF-11 Guardar imagen analizada}{p{3cm} p{.75cm} p{9.5cm}}{tablaRF11}{
  \multicolumn{3}{l}{\textbf{RF-11 Guardar imagen analizada}} \\
 }
 {
 Versión								 & 1.0\\\hline
  \multirow{2}{3.5cm}{Autores} 		& \multicolumn{2}{p{10cm}}{Adrián González Duarte} \\
                                	& \multicolumn{2}{p{10cm}}{Joaquín Bravo Panadero}\\\hline
  Objetivos asociados					 & \multicolumn{2}{p{10cm}}{OBJ-03}\\\hline
  Descripción                            & \multicolumn{2}{p{10cm}}{Permite exportar la imagen analizada junto con los defectos dibujados a un fichero} \\\hline
  Precondiciones                         & \multicolumn{2}{p{10cm}}{Debe haber finalizado el caso de uso RF-05} \\\hline
  \multirow{2}{3.5cm}{Secuencia normal}  & Paso & Acción \\\cline{2-3}
                                         & 1    & El usuario selecciona <<guardar imagen>> \\\cline{2-3} 
                                         & 2    & El usuario determina el fichero en el que guardar la imagen \\\cline{2-3}
                                         & 3	& El sistema guarda la imagen en el fichero especificado\\\hline
  Postcondiciones                        & \multicolumn{2}{p{10cm}}{Se ha creado un fichero con la imagen} \\\hline
   \multirow{2}{3.5cm}{Excepciones}       & Paso & Acción \\\cline{2-3}
                                         & 2    & Si se cancela el proceso, la imagen no se guarda\\\hline
  Rendimiento                            &      & \\\hline
  Frecuencia                             & \baja{2} \\\hline
  Importancia                            & \baja{2} \\\hline
  Urgencia                               & \baja{2} \\\hline
  Comentarios                            & \multicolumn{2}{p{10cm}}{La imagen se guardará en un fichero con formato JPG} \\
}

\tablaSmallSinColores{Caso de uso: RF-12 Guardar imagen binarizada}{p{3cm} p{.75cm} p{9.5cm}}{tablaRF12}{
  \multicolumn{3}{l}{\textbf{RF-12 Guardar imagen binarizada}} \\
 }
 {
 Versión								 & 1.0\\\hline
  \multirow{2}{3.5cm}{Autores} 		& \multicolumn{2}{p{10cm}}{Adrián González Duarte} \\
                                	& \multicolumn{2}{p{10cm}}{Joaquín Bravo Panadero}\\\hline
  Objetivos asociados					 & \multicolumn{2}{p{10cm}}{OBJ-03}\\\hline
  Descripción                            & \multicolumn{2}{p{10cm}}{Permite exportar los defectos detectados en una imagen binarizada a un fichero} \\\hline
  Precondiciones                         & \multicolumn{2}{p{10cm}}{Debe haber finalizado el caso de uso RF-05} \\\hline
  \multirow{2}{3.5cm}{Secuencia normal}  & Paso & Acción \\\cline{2-3}
                                         & 1    & El usuario selecciona <<guardar defectos binarizados>> \\\cline{2-3} 
                                         & 2    & El usuario determina el fichero en el que guardar la imagen \\\cline{2-3}
                                         & 3	& El sistema guarda la imagen en el fichero especificado\\\hline
  Postcondiciones                        & \multicolumn{2}{p{10cm}}{Se ha creado un fichero con la imagen} \\\hline
   \multirow{2}{3.5cm}{Excepciones}       & Paso & Acción \\\cline{2-3}
                                         & 2    & Si se cancela el proceso, la imagen no se guarda\\\hline
  Rendimiento                            &      & \\\hline
  Frecuencia                             & \baja{2} \\\hline
  Importancia                            & \baja{2} \\\hline
  Urgencia                               & \baja{2} \\\hline
  Comentarios                            & \multicolumn{2}{p{10cm}}{La imagen se guardará en un fichero con formato JPG} \\
}

\tablaSmallSinColores{Caso de uso: RF-13 Calcular <<Precision \& Recall>>}{p{3cm} p{.75cm} p{9.5cm}}{tablaRF13}{
  \multicolumn{3}{l}{\textbf{RF-13 Calcular <<Precision \& Recall>>}} \\
 }
 {
 Versión								 & 1.0\\\hline
  \multirow{2}{3.5cm}{Autores} 		& \multicolumn{2}{p{10cm}}{Adrián González Duarte} \\
                                	& \multicolumn{2}{p{10cm}}{Joaquín Bravo Panadero}\\\hline
  Objetivos asociados					 & \multicolumn{2}{p{10cm}}{OBJ-03}\\\hline
  Descripción                            & \multicolumn{2}{p{10cm}}{Permite realizar los cálculos de precision \& recall sobre los defectos localizados} \\\hline
  Precondiciones                         & \multicolumn{2}{p{10cm}}{Debe haber finalizado el caso de uso RF-05} \\\hline
  \multirow{2}{3.5cm}{Secuencia normal}  & Paso & Acción \\\cline{2-3}
                                         & 1    & El usuario selecciona <<calcular precision \& recall>> \\\cline{2-3} 
                                         & 2    & El usuario determina cuál es la máscara asociada a la imagen \\\cline{2-3}
                                         & 3	& El sistema realiza los cálculos y los muestra\\\hline
  Postcondiciones                        & \multicolumn{2}{p{10cm}}{Se muestra un diálogo con los datos} \\\hline
   \multirow{2}{3.5cm}{Excepciones}       & Paso & Acción \\\cline{2-3}
                                         & 2    & Si se cancela el proceso, el caso de uso finaliza\\\hline
  Rendimiento                            &      & \\\hline
  Frecuencia                             & \baja{2} \\\hline
  Importancia                            & \media{2} \\\hline
  Urgencia                               & \media{2} \\\hline
  Comentarios                            & \multicolumn{2}{p{10cm}}{Debe existir una máscara sobre la cual se van a realizar las comparaciones} \\
}


\newpage
\subsection{Requisitos no funcionales}
Una vez analizados los requisitos de información y los funcionales falta un tipo de requisitos que, normalmente, suelen ser de carácter técnico y se engloban dentro de requisitos no funcionales.

A continuación aparecen listados los requisitos no funcionales a tener en cuenta para el diseño de la aplicación:

\begin{itemize}
 \item \textbf{Extensible:} debe estar pensado para que se pueda ampliar añadiendo nuevas características a calcular, utilizando la misma interfaz.
 \item \textbf{Facilidad de uso de la interfaz:} la aplicación debe tener una interfaz que sea fácil de manejar y de entender por cualquier tipo de usuario. Debe ser intuitiva.
 \item \textbf{Ayuda:} la aplicación debe tener explicaciones de ayuda para los usuarios en las partes que puedan ser más difíciles de entender o manejar, o como aclaración de algún concepto.
 \item \textbf{Documentado:} si se pretende que un software sea ampliado por terceros, o continuado en un futuro, éste debe estar debidamente documentado.
\end{itemize}
\newpage

\subsection{Matrices}
¿Sería buena idea meter las matrices? ?¿O es información redundante? En caso de meterlas, serían:

\begin{itemize}
\item Matriz de actores - requisitos
\item Matriz de objetivos - requisitos
\item Matriz de requisitos de información – requisitos funcionales
\end{itemize}

%Interfaz de usuario
\section{Interfaz de usuario}
En este apartado se muestran una serie de ventanas correspondientes a la interfaz de usuario con la que la aplicación es capaz de cumplir los requisitos funcionales que se han establecido anteriormente.

Debido a que estamos aún en la parte de análisis del proyecto, estas interfaces tan solo son bocetos de las interfaces que tendrá la aplicación definitiva.

En la imagen \ver{prototipoVentanaPpal} se muestra el prototipo de la ventana principal, desde la que se pueden realizar todas las funcionalidades.

\figura{0.75}{imgs/prototipoVentanaPpal.png}{Prototipo de la ventana principal}{prototipoVentanaPpal}{}

En la imagen \ver{prototipoOpciones} se muestra el prototipo de la ventana de opciones avanzadas, desde la que podemos cambiar un cierto número de opciones.

\figura{0.75}{imgs/prototipoOpciones.png}{Prototipo de la ventana de opciones avanzadas}{prototipoOpciones}{}

\portadasAuxiliares{Anexo III - Especificación de diseño}
%%%%%%%%%%%%%%%%%%%%%%%%%%%%%%%%%%%%%%%%%%%%%%%%%%%%%%%%%%%%%%%%%%
%%%%%%%%%%%%%%%%%%%%%%%%%%%%%%%%%%%%%%%%%%%%%%%%%%%%%%%%%%%%%%%%%%
\chapter{Especificación de diseño}
%%%%%%%%%%%%%%%%%%%%%%%%%%%%%%%%%%%%%%%%%%%%%%%%%%%%%%%%%%%%%%%%%%
%%%%%%%%%%%%%%%%%%%%%%%%%%%%%%%%%%%%%%%%%%%%%%%%%%%%%%%%%%%%%%%%%%


%Introducción
\section{Introducción}

En este apartado se pretende dar una primera aproximación a la solución final del problema. Partiendo del análisis realizado en el apartado anterior, ahora se va a especificar cómo van a interactuar los objetos entre sí para dar la solución al problema, en contraposición a la anterior etapa de análisis dónde sólo se especificaba la funcionalidad. La frontera entre la finalización del análisis y el comienzo del diseño es difusa ya que el modelo va evolucionando y refinándose en cada paso.

En este apartado se marcará el camino a seguir hacia la solución final, y se tomarán decisiones muy importantes dentro de la arquitectura, los datos o la interfaz. Los detalles de diseño son muy importantes para conseguir unos factores de calidad externos e internos que permitan obtener un producto final de calidad.

Como se ha comentado con anterioridad, se va a realizar un diseño orientado a objetos. Esta metodología permite abordar el problema de una manera eficaz, y trabajar con clases y objetos que representan las abstracciones de los entes que se manifiestan en el dominio del problema, permitiéndonos así tener una comprensión mucho más clara de cómo funciona el sistema final, y de cómo están relacionadas sus piezas.

El diseño es una pieza clave para el correcto desarrollo de un proyecto software, ya que facilita llevar a cabo una estructuración modular, permitiendo identificar cada elemento del programa. Esto hace posible encontrar todos aquellos puntos peligrosos a los que el desarrollo podría enfrentarse en la fase de construcción.

En este apartado se definirá la estructura de paquetes de la aplicación y las relaciones que tienen entre ellos. Después, se desglosará cada paquete con los diagramas de clases, permitiendo ver y comprender el diseño con más detalle. Estos diagramas no incluirán todas las clases y atributos, sino que, por temas de legibilidad, sólo contendrán aquellas que sean más importantes para el funcionamiento del programa, ocultando información superflua.

\newpage

\section{Ámbito del software}
El objetivo del sistema es la construcción de una aplicación que permita trabajar con imágenes radiográficas, pudiendo entrenar un clasificador tanto desde cero (es decir, con una serie de imágenes etiquetadas) como a partir de un fichero \arff{}, así como analizar esas radiografías para determinar si existen defectos.

Se diseñará una interfaz gráfica que deberá disponer de las características que se definieron en la fase de análisis previa. Dicha interfaz deberá ser intuitiva y fácil de usar, aportando al usuario la máxima información posible para que pueda utilizarla eficientemente.

Las restricciones de diseño a aplicar sobre el desarrollo son:
\begin{itemize}
 \item Arquitectura basada en objetos.
 \item Aplicación extensible.
 \item Interfaz sencilla e intuitiva.
 \item Máxima información de cara al usuario.
\end{itemize}

\newpage

\section{Diseño Arquitectónico}
El diseño arquitectónico tiene como objetivo desarrollar la estructura modular que representa las relaciones entre los módulos, combinando la estructura del programa con la de los datos a través de interfaces que permiten el flujo de éstos.

La arquitectura de la aplicación depende del problema a tratar. La mayoría de las veces se utilizan varios estilos arquitectónicos para la solución.

\subsection{Estilos arquitectónicos utilizados}
Tras un estudio de distintos estilos arquitectónicos, hemos considerado que los más convenientes para nuestra aplicación son los que se muestran a continuación.

\subsubsection{Arquitectura en capas}
Este tipo de arquitectura distribuye jerárquicamente los roles y las responsabilidades. De esta forma, se proporciona una división de los problemas a resolver. Los roles indican cómo una capa se relaciona con otra, mientras que las responsabilidades indican su funcionalidad \cite{arquitectura}. Entre sus características, podemos destacar:

\begin{itemize}
\item Descomposición de los servicios.
\item Las capas pueden permanecer en una misma máquina o en equipos distintos.
\item Los componentes de cada capa se comunican con otras capas mediante interfaces.
\item Se diferencia claramente la funcionalidad de cada capa.
\item Se trata de una abstracción a muy alto nivel, por lo que no se conocen tipos de datos, atributos, métodos e implementaciones.
\item Las capas inferiores no tienen dependencias
\end{itemize}

Como principales ventajas, tenemos la abstracción, el aislamiento, el rendimiento y la testeabilidad, al ser cada capa independiente de las demás.

Esta arquitectura debería usarse cuando:

\begin{itemize}
\item Se tienen construidas capas de otra aplicación y pueden ser reutilizables.
\item La aplicación es muy compleja y el diseño requiere la separación de funcionalidad.
\item Se quiere implementar reglas de negocio complicadas.
\item Se deben soportan diferentes tipos de clientes.
\end{itemize}

En \ver{arq} podemos ver cómo se estructura.

\figura{0.65}{imgs/arquitecturacapas.png}{Arquitectura en capas \cite{arquitectura}}{arq}{}

\subsection{Arquitectura final de la aplicación}
Hemos estructurado nuestra aplicación siguiendo la arquitectura en capas, en la cual distinguimos:

\begin{itemize}
\item La capa de presentación o interfaz es la capa encargada de interactuar con el usuario. Permite, por ejemplo, que éste pueda visualizar los resultados de los análisis de imágenes y se encarga también de recibir los datos de entrada del usuario.
\item La capa de lógica de negocio contiene todas las reglas del dominio de nuestra aplicación. Esta capa en ocasiones recibe otros nombres como dominio, modelo de negocio, modelo de dominio, etc. En nuestro caso, es la capa del modelo, que se encarga de manejar la lógica de la aplicación.
\item La capa de acceso a datos se encarga de obtener los datos alojados en sistemas de almacenamiento persistente. En nuestro caso, abre las imágenes y se encarga de gestionar físicamente los archivos \arff{}.
\end{itemize}

\subsection{Patrones de diseño utilizados}
Hemos realizado una clasificación de los patrones de diseño de acuerdo a la capa lógica a la que pertenecen: Acceso a Datos, Dominio o Presentación \cite{carlos_lopez_nozal_apuntes_2012}.

\subsubsection{Patrones en la capa de Modelo del Dominio}

\paragraph{Fachada}\mbox{} \\
\indent Hemos utilizado el patrón Fachada para separar la parte de interfaz de usuario de la capa de dominio. Esta fachada encapsula la lógica de la aplicación y va a ser la encargada de manejar los procesos de análisis y entrenamiento, controlando el comportamiento de las ventanas. En la imagen \ver{fachada} se puede ver su estructura. Tiene las siguientes características:

\figura{0.5}{imgs/dclasesfachada.png}{Patrón Fachada - Diagrama de clases \cite{carlos_lopez_nozal_apuntes_2012}}{fachada}{}

\begin{itemize}
\item \textbf{Problema}

\begin{itemize}
\item Existe mucha dependencia entre las clases que representan una abstracción y las clases clientes. Las dependencias añaden una complejidad notable a los clientes.
\item Se quiere simplificar las clases clientes.
\item Se quiere poner una barrera entre una clase cliente y un conjunto de clases y relaciones que implementan una abstracción.
\end{itemize}

\item \textbf{Solución}

\begin{itemize}
\item Se proporciona un objeto adicional reutilizable que oculta gran parte de la complejidad del trabajo de las clases. El cliente sólo tiene que relacionarse con este nuevo objeto.
\item Los clientes se comunican con el subsistema a través de la fachada, que reenvía las peticiones a los objetos del subsistema apropiados y puede realizar también algún trabajo de traducción.
\item Los clientes que usan la fachada no necesitan acceder directamente a los objetos del sistema.
\end{itemize}

\item \textbf{Conclusiones}

\begin{itemize}
\item Los clientes no necesitan conocer cómo se relacionan, ni cómo se crean las clases que proporcionan el servicio, solo se comunican a través del los objetos Fachada.
\item Reduce o elimina el acoplamiento entre las clases cliente y las clases que implementan la abstracción. Se podrían cambiar las clases que implementan la abstracción sin ningún impacto en las clases clientes. Ayuda a dividir un sistema en capas y reduce dependencias de compilación.
\item Los clientes que necesiten acceder directamente a los objetos que implementan la abstracción, pueden acceder a ellos.
\end{itemize}
\end{itemize}

En nuestro caso, como ya hemos dicho, utilizamos una clase llamada Fachada que es la que separa las capas y la que encapsula la lógica \ver{nuestrafachada}

\figuraConPosicion{0.5}{imgs/nuestrafachada.png}{Diagrama de clases de nuestra Fachada}{nuestrafachada}{}{H}

\paragraph{Singleton}\mbox{} \\
\indent Hemos utilizado el patrón Singleton para poder instanciar únicamente un objeto de tipo Fachada, que es el encargado de manejar la lógica de negocio. En \ver{dclasessingleton} podemos ver cómo se estructura este patrón. Tiene las siguientes características:

\figuraConPosicion{0.5}{imgs/dclasessingleton.png}{Patrón Singleton - Diagrama de clases \cite{carlos_lopez_nozal_apuntes_2012}}{dclasessingleton}{}{H}


\begin{itemize}
\item \textbf{Problema}

Algunas clases deberían tener sólo una instancia. Estas clases están generalmente relacionadas con el manejo de un determinado recurso.

\item \textbf{Solución}

\begin{itemize}
\item Involucra una única clase.
\item La clase Singleton tiene una variable estática que referencia a la única instancia de la clase.
\item Para prevenir que los clientes creen más instancias de la clase se declara el constructor privado.
\item La instancia de la clase se puede crear cuando la clase se carga.
\item Para permitir el acceso a la instancia, la clase proporciona un método estático, típicamente llamado getInstance().
\end{itemize}

\item \textbf{Conclusiones}

\begin{itemize}
\item Sólo existe una instancia de la clase Singleton.
\item Otras clases que utilizar la instancia de la clase Singleton lo deben hacer invocando el método getInstance().
\end{itemize}
\end{itemize}

\paragraph{Estrategia}\mbox{} \\
Hemos utilizado el patrón estrategia para encapsular los diferentes algoritmos que usa nuestra aplicación. Hemos necesitado tres estrategias: una para los tipos de ventana, otra para los algoritmos de extracción de características y otra para los algoritmos de preprocesamiento. Con esto buscamos que se puedan añadir fácilmente nuevos algoritmos.

En \ver{dclasesestrategia} podemos ver cómo se estructura este patrón. Tiene las siguientes características:

\figuraConPosicion{0.75}{imgs/dclasesestrategia.png}{Patrón Estrategia - Diagrama de clases \cite{carlos_lopez_nozal_apuntes_2012}}{dclasesestrategia}{}{H}

\begin{itemize}
\item \textbf{Problema}

\begin{itemize}
\item Muchas clases relacionadas difieren solo en su comportamiento.
\item Se necesitan distintas variantes de un algoritmo.
\item Un algoritmo usa datos que el cliente no debería conocer. Se quiere evitar exponer estructuras complejas y dependientes del algoritmo.
\item Una clase define múltiples comportamientos y estos se representan con múltiples sentencias condicionales en sus operaciones.
\end{itemize}

\item \textbf{Solución}

\begin{itemize}
\item Define una familia de algoritmos, encapsula cada uno de ellos y los hace intercambiables. Permite que un algoritmo varíe independientemente de los clientes que lo usan.
\item Se declara una interfaz común para todos los algoritmos soportados
\item Los clientes que usan la fachada no necesitan acceder directamente a los objetos del sistema.
\end{itemize}
\end{itemize}

Como ya hemos dicho, nosotros tenemos tres estrategias, por lo que necesitamos tres superclases que contendrán las operaciones comunes de cada estrategia. De ellas heredarán una serie de clases, que son las que implementan los algoritmos concretos, como puede ser el desplazamiento de una ventana deslizante o la implementación de las características de Haralick. En \ver{nuestraestrategia} se puede ver una simplificación de nuestras clases.

\figuraConPosicion{0.75}{imgs/nuestraestrategia.png}{Diagrama de clases de nuestras Estrategias}{nuestraestrategia}{}{H}

\newpage

\subsection{Diagrama de clases de diseño}

\figuraConPosicionSinMarco{1.1}{imgs/Diagramadeclases.png}{Diagrama de clases}{diagramaclases}{}{h}

\begin{itemize}
\item \textbf{Paquete interfaz:} contiene las clases que implementan la interfaz gráfica de la aplicación.
\item \textbf{Paquete modelo:} contiene las clases que componen la lógica de la aplicación. La clase Fachada es el punto de entrada y la encargada de controlar toda la lógica.
	\begin{itemize}
	\item \textbf{Subpaquete Preprocesamiento:} en este subpaquete se encuentra la estructura del patrón estrategia para los algoritmos de preprocesamiento.
	\item \textbf{Subpaquete Ventana:} contiene la estructura correspondiente al patrón estrategia para los diferentes tipos de ventana.
	\item \textbf{Subpaquete Feature:} en él se encuentran las clases que conforman la estructura del patrón estrategia para los diferentes algoritmos de extracción de características.
	\end{itemize}
\item \textbf{Paquete datos:} contiene las clases que se encargan de abrir imágenes y gestionar los ficheros \arff{}.
\item \textbf{Paquete utils:} contiene las clases de utilería de la aplicación.
\end{itemize}


\section{Diseño de la interfaz}
En este apartado, se cierra la parte de diseño de la interfaz gráfica, y se resaltan los aspectos más relevantes de este proceso.

\subsection{Ventana principal}

\figuraConPosicion{1}{imgs/ventanaprincipal.png}{Diseño de la ventana principal}{disventppal}{}{H}

Desde la ventana principal podemos acceder a todos los casos de uso mediante botones y menú. Se ha preferido esta distribución a, por ejemplo, distintas ventanas, para tenerlo todo en un mismo lugar. Para evitar confusiones, se ha elegido una estrategia de habilitación/deshabilitación de los botones según se puedan hacer o no determinadas acciones.

\subsection{Ventana de opciones avanzadas}

\figuraConPosicion{0.65}{imgs/opcionesavanzadas.png}{Diseño de la ventana de opciones avanzadas}{disventavanzadas}{}{H}

Desde esta ventana se permite cambiar todas las opciones del programa. Se accede a ella mediante el menú de opciones de la ventana principal.

\subsection{Ventana de Precision \& Recall}

\figuraConPosicion{0.35}{imgs/ventprecisionrecall.png}{Diseño de la ventana de precision \& recall}{disventaprecrecall}{}{H}

En esta ventana se muestran los resultados de precision \& recall. Se accede a ella mediante el botón de la ventana principal, que sólo se puede pulsar cuando ha acabado un análisis.

\newpage

\section{Diseño procedimental}
En esta sección se detallarán aquellos casos de uso que sean más complejos y que por ello
requieran de un estudio más individualizado de los mismos. Se utilizarán diagramas de secuencia
para hacer más fácil el seguimiento de las interacciones entre las distintas clases.

Los casos de uso analizados son:

\begin{itemize}
\item RF-01 Abrir imagen \ver{dsecuenciaAbriImagen}.
\item RF-03 Entrenar clasificador (opción entrenamiento con ARFF) \ver{dsecuenciaEntrenar1}.
\item RF-04 Entrenar clasificador (opción entrenamiento con imágenes), parte 1 \ver{dsecuenciaEntrenar2parte1}, parte 2 \ver{dsecuenciaEntrenar2parte2}, parte 3 \ver{dsecuenciaEntrenar2parte3} y parte 4 \ver{dsecuenciaEntrenar2parte4}.
\item RF-05 Analizar imagen, parte 1 \ver{dsecuenciaAnalizarparte1}, parte 2 \ver{dsecuenciaAnalizarparte2} y parte 3 \ver{dsecuenciaAnalizarparte3}.
\item RF-10 Cambiar opciones \ver{dsecuenciaOpciones}.
\item RF-13 Calcular <<Precision \& Recall>> \ver{dsecuenciaPrecision}.
\end{itemize}

\figuraConPosicionSinMarco{1}{imgs/dsecuenciaAbriImagen.png}{Diagrama de secuencia: Abrir imagen}{dsecuenciaAbriImagen}{}{H}

\figuraApaisadaSinMarco{.8}{imgs/dsecuenciaEntrenar1.png}{Diagrama de secuencia: Entrenar clasificador (opción entrenamiento con ARFF)}{dsecuenciaEntrenar1}{}

\figuraApaisadaSinMarco{.8}{imgs/dsecuenciaEntrenar2parte1.png}{Diagrama de secuencia: Entrenar clasificador (opción entrenamiento con imágenes, parte 1)}{dsecuenciaEntrenar2parte1}{}

\figuraApaisadaSinMarco{.8}{imgs/dsecuenciaEntrenar2parte2.png}{Diagrama de secuencia: Entrenar clasificador (opción entrenamiento con imágenes, parte 2)}{dsecuenciaEntrenar2parte2}{}

\figuraApaisadaSinMarco{.8}{imgs/dsecuenciaEntrenar2parte3.png}{Diagrama de secuencia: Entrenar clasificador (opción entrenamiento con imágenes, parte 3)}{dsecuenciaEntrenar2parte3}{}

\figuraApaisadaSinMarco{.8}{imgs/dsecuenciaEntrenar2parte4.png}{Diagrama de secuencia: Entrenar clasificador (opción entrenamiento con imágenes, parte 4)}{dsecuenciaEntrenar2parte4}{}

\figuraApaisadaSinMarco{.8}{imgs/dsecuenciaAnalizarparte1.png}{Diagrama de secuencia: Analizar imagen (parte 1)}{dsecuenciaAnalizarparte1}{}

\figuraApaisadaSinMarco{.8}{imgs/dsecuenciaAnalizarparte2.png}{Diagrama de secuencia: Analizar imagen (parte 2)}{dsecuenciaAnalizarparte2}{}

\figuraApaisadaSinMarco{.8}{imgs/dsecuenciaAnalizarparte3.png}{Diagrama de secuencia: Analizar imagen (parte 3)}{dsecuenciaAnalizarparte3}{}

\figuraConPosicionSinMarco{0.9}{imgs/dsecuenciaOpciones.png}{Diagrama de secuencia: Cambiar opciones}{dsecuenciaOpciones}{}{H}

\figuraConPosicionSinMarco{1}{imgs/dsecuenciaPrecision.png}{Diagrama de secuencia: Calcular <<Precision \& Recall>>}{dsecuenciaPrecision}{}{H}

\section{Referencia cruzada a los requisitos}
De cara a la trazabilidad del sistema, es interesante incluir matrices que relacionen los requisitos funcionales con los elementos de diseño. De esta forma puede observarse rápidamente si se han satisfecho todos los requisitos de la aplicación.

\subsection{RF-01 Abrir imagen}
\begin{itemize}
\item[] es.ubu.XRayDetector.interfaz.PanelAplicacion
\item[] es.ubu.XRayDetector.modelo.Fachada
\item[] es.ubu.XRayDetector.utils.Graphics
\end{itemize}

\subsection{RF-02 Entrenar clasificador}
\begin{itemize}
\item[] es.ubu.XRayDetector.interfaz.PanelAplicacion
\item[] es.ubu.XRayDetector.modelo.Fachada
\end{itemize}

\subsection{RF-03 Entrenar clasificador con ARFF existente}
\begin{itemize}
\item[] es.ubu.XRayDetector.modelo.Fachada
\item[] es.ubu.XRayDetector.datos.GestorArff
\end{itemize}

\subsection{RF-04 Entrenar clasificador generando nuevo ARFF}
\begin{itemize}
\item[] es.ubu.XRayDetector.interfaz.PanelAlicacion
\item[] es.ubu.XRayDetector.modelo.Fachada
\item[] es.ubu.XRayDetector.utils.Propiedades
\item[] es.ubu.XRayDetector.modelo.preprocesamiento.Saliency
\item[] es.ubu.XRayDetector.modelo.ventana.VentanaAleatoria
\item[] es.ubu.XRayDetector.modelo.ventana.VentanaDeslizante
\item[] es.ubu.XRayDetector.datos.GestorArff
\end{itemize}

\subsection{RF-05 Analizar imagen}
\begin{itemize}
\item[] es.ubu.XRayDetector.interfaz.PanelAplicacion
\item[] es.ubu.XRayDetector.utils.Propiedades
\item[] es.ubu.XRayDetector.modelo.Fachada
\item[] es.ubu.XRayDetector.modelo.preprocesamiento.Saliency
\item[] es.ubu.XRayDetector.modelo.ventana.VentanaDeslizante
\item[] es.ubu.XRayDetector.modelo.ventana.VentanaRegiones
\item[] es.ubu.XRayDetector.utils.Graphics
\item[] es.ubu.XRayDetector.utils.AutoLocalThreshold
\item[] es.ubu.XRayDetector.utils.ParticleAnalyzer
\item[] es.ubu.XRayDetector.utils.Thresholder
\end{itemize}

\subsection{RF-06 Calcular características}
\begin{itemize}
\item[] es.ubu.XRayDetector.modelo.ventana.VentanaAbstracta
\item[] es.ubu.XRayDetector.modelo.feature.Standard
\item[] es.ubu.XRayDetector.modelo.feature.Lbp
\item[] es.ubu.XRayDetector.modelo.feature.Haralick
\item[] es.ubu.XRayDetector.utils.Differentials\_
\end{itemize}

\subsection{RF-07 Seleccionar defecto}
\begin{itemize}
\item[] es.ubu.XRayDetector.interfaz.PanelAplicacion
\item[] es.ubu.XRayDetector.utils.Graphics
\item[] es.ubu.XRayDetector.modelo.Fachada
\end{itemize}

\subsection{RF-08 Exportar log}
\begin{itemize}
\item[] es.ubu.XRayDetector.PanelAplicacion
\end{itemize}

\subsection{RF-09 Abrir ayuda}
\begin{itemize}
\item[] es.ubu.XRayDetector.interfaz.BarraMenu
\item[] es.ubu.XRayDetector.interfaz.PanelAplicacion
\end{itemize}

\subsection{RF-10 Cambiar opciones}
\begin{itemize}
\item[] es.ubu.XRayDetector.interfaz.BarraMenu
\item[] es.ubu.XRayDetector.utils.Propiedades
\end{itemize}

\subsection{RF-11 Guardar imagen analizada}
\begin{itemize}
\item[] es.ubu.XRayDetector.interfaz.PanelAplicacion
\end{itemize}

\subsection{RF-12 Guardar imagen binarizada}
\begin{itemize}
\item[] es.ubu.XRayDetector.interfaz.PanelAplicacion
\item[] es.ubu.XRayDetector.modelo.Fachada
\end{itemize}

\subsection{RF-12 Calcular <<Precision \& Recall>>}
\begin{itemize}
\item[] es.ubu.XRayDetector.interfaz.PanelAplicacion
\item[] es.ubu.XRayDetector.interfaz.PrecisionRecallDialog
\item[] es.ubu.XRayDetector.modelo.Fachada
\end{itemize}

\section{Pruebas}
En este apartado se recogen las pruebas de integración del sistema y aspectos relevantes del diseño de las mismas.

Las pruebas de integración testean los módulos de la aplicación de forma combinada, es decir, como un grupo. Estas pruebas se realizan testeando un conjunto de elementos unitarios.

\subsection{Pruebas de integración}
FALTA

\newpage

\section{Entorno tecnológico del sistema}
Este entorno establece el equipo físico, el equipo lógico y las comunicaciones del sistema software. Los detalles de la parte física se indican en el siguiente apartado.

Necesitaremos tener instalada la máquina virtual de Java. La aplicación es monolítica, por lo que sólo estará desplegada en un único equipo.

Es recomendable tener instalado también un navegador web para poder ver los informes que se exporten.

Las dependencias entre la aplicación y las librerías que utiliza se muestran en \ver{componentes}

\figuraConPosicion{0.55}{imgs/componentes.png}{Diagrama de componentes}{componentes}{}{H}



\section{Plan de desarrollo e implantación}
La aplicación se implantará en una misma máquina que proporcione todas las funcionalidades requeridas, ya que, como hemos dicho, es monolítica.

En la imagen \ver{despliegue} se muestra el diagrama de despliegue que representa la implantación de la aplicación en la máquina cliente.

\figuraConPosicion{0.55}{imgs/despliegue.png}{Diagrama de despliegue}{despliegue}{}{H}

\portadasAuxiliares{Anexo IV - Manual del programador}
%%%%%%%%%%%%%%%%%%%%%%%%%%%%%%%%%%%%%%%%%%%%%%%%%%%%%%%%%%%%%%%%%%
%%%%%%%%%%%%%%%%%%%%%%%%%%%%%%%%%%%%%%%%%%%%%%%%%%%%%%%%%%%%%%%%%%
\chapter{Manual del programador}
%%%%%%%%%%%%%%%%%%%%%%%%%%%%%%%%%%%%%%%%%%%%%%%%%%%%%%%%%%%%%%%%%%
%%%%%%%%%%%%%%%%%%%%%%%%%%%%%%%%%%%%%%%%%%%%%%%%%%%%%%%%%%%%%%%%%%


%Introducción
\section{Introducción}
Este anexo explica en detalle la fase de implementación del proyecto.

Se realiza la documentación de las librerías utilizadas y las creadas específicamente para la aplicación, un manual del programador para que otras personas puedan trabajar en el proyecto en un futuro y una descripción de las pruebas unitarias del sistema.

\section{Documentación de las bibliotecas}\label{bib}

\subsection{Bibliotecas de Java}
Java es un lenguaje de programación orientado a objetos, desarrollado por Sun Microsystems a principios de los años 90.

El lenguaje en sí mismo toma mucha de su sintaxis de C y C++, pero tiene un modelo de objetos más simple y elimina herramientas de bajo nivel, que suelen inducir a muchos errores, como la manipulación directa de punteros o memoria.

En concreto, nosotros hemos utilizado \textit{Java 7}, por lo que es necesario descargar el \textit{Development Kit} correspondiente a esta versión.

\subsection{ImageJ}
ImageJ permite ser usado programáticamente mediante llamadas a un JAR. Esto nos ha permitido trabajar con las imágenes de forma sencilla, por ejemplo, estableciendo regiones de interés en una imagen y extrayendo información de sus píxeles.

En concreto, hemos usado la versión 1.47n, que se corresponde con el archivo \textit{ij.jar}.

\subsection{Weka}
\weka{}\cite{weka} es una plataforma de software para aprendizaje automático y minería de datos, diseñada por la Universidad de Waikato. Está escrita en \java{}.

Nosotros hemos utilizado \weka{} usando su JAR, lo que permite realizar llamadas a sus métodos dentro de nuestro código, como para, por ejemplo, construir y usar clasificadores. Esto permite realizar estas tareas de una forma muy sencilla, sin necesidad de programar estos métodos manualmente.

\subsection{JavaHelp}
JavaHelp es una expansión de Java que facilita la programación de las ventanas de ayuda en las aplicaciones java.

Con JavaHelp se pueden crear las ventanas típicas de ayuda de las aplicaciones informáticas, en las que sale en el lado izquierdo un panel con varias pestañas: índice de contenidos, búsqueda, temas favoritos, índice alfabético, etc., mientras que en el lado derecho sale el texto de la ayuda.

Las ventanas de ayuda pueden lanzarse directamente con la pulsación de botones en la aplicación, o bien por medio de la pulsación de la tecla F1, mostrando la ayuda correspondiente a la ventana sobre la que estamos trabajando.

Las ventanas de ayuda de JavaHelp se configuran por medio de varios ficheros en formato XML. Los textos de ayuda que se quieran mostrar se escribirán en ficheros con formato HTML.

JavaHelp no se incluye en la JDK, ni en la JRE, sino que debe conseguirse como un paquete externo.

La biblioteca correspondiente al módulo de JavaHelp se corresponde con el archivo jh.jar.


\subsection{Apache Commons IO}
Esta librería permite realizar algunas operaciones con ficheros, como es la fusión de uno o más ficheros de texto, o la exportación de un cierto texto a un fichero externo de una forma muy sencilla, evitándonos problemas.

Hemos usado la versión 2.4, y se corresponde con el archivo \textit{commons-io-2.4}.

\subsection{EJML}
Esta librería ha sido usada sólo para el cálculo de una de las caracteristicas de Haralick que requería previamente el cálculo de unos autovalores. Estos cálculos son computacionalmente muy exigentes, y la solución que dieron los desarrolladores anteriores no era buena. Esta librería realiza cálculos con matrices de forma muy eficiente, pero pese a ello, los cálculos que necesitamos ralentizan la ejecución de nuestra aplicación.

Hemos usado la versión 0.21, que se corresponde con el archivo \textit{ejml-0.21}.


\section{Código fuente}
El código fuente de la aplicación XRayDetector se puede encontrar en la carpeta XRayDetector/src.

Dentro de la carpeta src, los archivos pertenecientes al código fuente se encuentran organizados en carpetas correspondientes a los diferentes paquetes que conforman la estructura de ficheros Java de la aplicación.

Esta estructura está organizada de la siquiente forma:

FALTA

\subsection{Recursos necesarios por el código fuente}
La carpeta \textit{res} contiene los archivos para la correcta ejecución y funcionamiento de la aplicación. Se encuentra estructurada en subcarpetas, que son:

\begin{itemize}
\item Arff: contiene ficheros \arff{} de ejemplo.
\item Ayuda: Se corresponde con el módulo de ayuda en línea que utiliza la aplicación mediante \textit{JavaHelp}.
\item Config: contiene un único fichero (\textit{config.properties}) con las opciones elegidas por el usuario.
\item Img: contiene, por un lado, imágenes de ejemplo para usar la aplicación, y, por otro lado, los iconos necesarios para la interfaz de la aplicación (en la carpeta \textit{app}).
\item Log: contiene el archivo generado por el gestor de log integrado en la aplicación. Además, cuando el usuario exporta el contenido del log de la interfaz, se guarda automáticamente en la carpeta \textit{html} dentro de esta misma carpeta.
\item Model: contiene clasificadores de ejemplo para poder usar la aplicación.
\end{itemize}

La carpeta \textit{lib} contiene las bibliotecas necesarias para la compilación y ejecución de la aplicación. Estas librerías se corresponden con los archivos .JAR, descritos en el apartado \ref{bib}.

El código fuente de las pruebas se encuentra en la carpeta XRayDetector/test.

La documentación interna de las clases tras ser generada es almacenada en el directorio XRayDetector/docs/javadoc.

La documentación correspondiente al docheck referente a la documentación interna de las clases tras ser generada es almacenada en el directorio XRayDetector/docs/docCheck.

\section{Manual del programador}
Este manual pretende ser una guía de referencia para futuros programadores de la aplicación, facilitándoles en la medida de lo posible, la creación de nuevos componentes.

\subsection{Agregar nuevos elementos a la aplicación}
A continuación se va a mostrar un ejemplo para añadir nuevos componentes a la aplicación, creando nuevos tipos de ventana, nuevos algoritmos de extracción de características y nuevos algoritmos de preprocesamiento.

\subsubsection{Creación de nuevos elementos en las estrategias}
Como ya hemos visto, para las ventanas, extracción de características y preprocesamiento se han usado sendos patrones estrategia. Por lo tanto, el primer paso es simplemente incluir la nueva clase en esta estructura, heredando de la superclase que corresponda. Después, hay que implementar los métodos abstractos de estas superclases, que son realmente donde va a estar la funcionalidad específica.

En la extracción de características, es necesario guardar un vector con los nombres de cada descriptor en particular (\textit{headVector}) y otro, del mismo tamaño, con los valores de cada descriptor, en formato \textit{double}.

Para integrar una nueva ventana en el funcionamiento de la aplicación, es necesario añadir una opción en las opciones avanzadas y, después, los métodos que llaman a las ventanas (ejecutar entrenamiento, ejecutar análisis) deberán controlarlas.

Para integrar un nuevo algoritmo de extracción de características, habría que llamar a sus métodos dentro e la clase VentanaAbstracta, que es la que contiene los cálculos. Además, se hace necesario añadir los nombres de los descriptores en la cabecera de los \arff{}.

Para integrar un nuevo algoritmo de preprocesamiento, sería necesario crear una nueva opción, ya que de momento sólo hay uno y no se ha implementado esta opción, aunque la estructura está pensaba para albergar nuevos algoritmos.

\subsection{Modificación del módulo de ayuda en línea de la aplicación y del motor de búsqueda}
Para realizar el módulo de ayuda online de la aplicación, hemos utilizado la biblioteca JavaHelp, la cual ya hemos reseñado en este mismo anexo de la memoria.

Para poder usar JavaHelp, necesitaremos los siguientes ficheros:

\begin{itemize}
\item Ficheros html: en estos ficheros escribiremos la ayuda de la aplicación. Se usa una codificación html estándar, y podemos poner la información de ayuda correspondiente a las ventanas de la aplicación.
\item Mapa de JavaHelp: este fichero contiene los nombres de los html junto con la clave que le damos a cada uno de ellos.

Tendremos un mapID por cada html que queremos que se muestre. En nuestro caso, es el fichero <<mapa.jhm>> contenido en el directorio XRayDetector/res/ayuda.
\item Tabla de contenidos: en él incluiremos los capítulos y subcapítulos de los que consta nuestra ayuda. En nuestro caso, se trata del fichero <<toc.xml>> que se encuentra en el directorio de ayuda.
\item Fichero HelpSet: es el fichero principal de la ayuda. En él se indica qué se mostrará en la ayuda. En nuestro caso, sólo queremos que se muestren la tabla de contenidos y un motor de búsqueda.

Hay que indicar, además, cuál es el archivo de mapas. Es el fichero <<ayuda.hs>> del directorio de ayuda.
\item Motor de búsqueda: para realizar el motor de búsqueda, debemos tener descomprimida la biblioteca de JavaHelp.

Además de meter la correspondiente parte del motor en el fichero del HelpSet, debemos introducir el siguiente comando mediante línea de comandos, estando en el directorio de la ayuda:

\begin{verbatim}

java -jar path_java_help/jh2.0/javahelp/bin/jhindexer.jar

\end{verbatim}

Donde el \textit{path\_java\_help} es el directorio en el que hemos descomprimido la biblioteca de JavaHelp.
Una vez realizado, se habrá creado una carpeta llamada \textit{JavaHelpSearch}, que es la que contiene el motor de búsqueda.
\end{itemize}

\section{Pruebas unitarias}
FALTA

 


\portadasAuxiliares{Anexo V - Manual del usuario}
%%%%%%%%%%%%%%%%%%%%%%%%%%%%%%%%%%%%%%%%%%%%%%%%%%%%%%%%%%%%%%%%
%%%%%%%%%%%%%%%%%%%%%%%%%%%%%%%%%%%%%%%%%%%%%%%%%%%%%%%%%%%%%%%%
\chapter{Manual del usuario}
%%%%%%%%%%%%%%%%%%%%%%%%%%%%%%%%%%%%%%%%%%%%%%%%%%%%%%%%%%%%%%%%
%%%%%%%%%%%%%%%%%%%%%%%%%%%%%%%%%%%%%%%%%%%%%%%%%%%%%%%%%%%%%%%%

\section{Documento de instalación y configuración}

\subsection{Requisitos}
El único requisito necesario para la ejecución de XRayDetector es tener instalada la máquina virtual de Java en el equipo.

\subsection{Instalación y ejecución de la aplicación}
Para realizar la instalación de X-Ray Detector, basta con copiar la carpeta de «XRayDetector»
en el directorio deseado.

Para ejecutar la aplicación haz doble clic en el archivo «ejecutar.bat».

\section{Manual de usuario}

\subsection{Ventana principal de la aplicación}
Al iniciar la aplicación aparecerá la ventana de \ver{ventanappal}, que se corresponde con la ventana principal de XRayDetector.

\figuraConPosicion{0.95}{imgs/ventanaprincipal.png}{Ventana principal de XRayDetector}{ventanappal}{}{H}

A continuación, se procede a describir los elementos que componen la ventana principal, que se encuentra dividida en múltiples secciones.

\subsubsection*{Control}
Aquí nos podemos encontrar los controles principales de la aplicación.

\begin{itemize}
\item \textbf{Abrir imagen:} Permite cargar una imagen en la aplicación.
\item \textbf{Entrenar clasificador:} Entrenar un clasificador a partir de un conjunto de imágenes o de un archivo ARFF.
\item \textbf{Analizar imagen:} Analiza la imagen en busca de defectos.
\end{itemize}

\figuraConPosicion{0.4}{imgs/Control.png}{Menú control}{control}{}{H}

\subsubsection*{Progreso}
Indica el progreso que lleva la tarea que se está realizando la aplicación hasta que se complete.

\begin{itemize}
\item \textbf{Barra de progreso:} Indica cuál es el progreso actual de la tarea que se está realizando.
\item \textbf{Botón Stop:} Detiene la tarea que se está realizando actualmente.
\end{itemize}

\figuraConPosicion{0.4}{imgs/Progreso.png}{Menú progreso}{progreso}{}{H}

\subsubsection*{Nivel de tolerancia}
Indica el nivel de tolerancia que se utilizará para ajustar los bordes al defecto detectado, a mayor valor mayor precisión. Este valor puede ajustarse desplazando la barra de desplazamiento a izquierda y derecha.

\figuraConPosicion{0.4}{imgs/tolerancia.png}{Nivel de tolerancia}{tolerancia}{}{H}

\subsubsection*{Analizar resultados}
Permite analizar y exportar los resultados obtenidos tras el proceso de detección.

\begin{itemize}
\item \textbf{Guardar imagen analizada:} Permite guardar una copia de la imagen mostrada en el visor al disco duro.
\item \textbf{Guardar imagen binarizada:} Guarda una copia del defecto detectado sobre la imagen a través de una imagen binarizada en el disco duro.
\item \textbf{Calcular precision and recall:} Calcula los valores de precision and recall de la imagen sobre los resultados obtenidos.
\end{itemize}

\figuraConPosicion{0.5}{imgs/Resultados.png}{Analizar resultados}{resultados}{}{H}

\subsubsection*{Log}
Muestra al usuario información referente al estado de la aplicación y los resultados tras ejecutar diversas operaciones en XRayDetector.

\begin{itemize}
\item \textbf{Limpiar log:} Borra el contenido actual del log.
\item \textbf{Exportar log:} Exporta el contenido actual del log a un archivo HTML.
\end{itemize}

\figuraConPosicion{0.6}{imgs/Exportarlog.png}{Control de Log}{log}{}{H}

\subsubsection*{Visor}
Aquí se muestra la imagen cargada en la aplicación, así como el resultado del proceso de detección de defectos. Una vez analizada la imagen, se marcarán los defectos en la imagen y podrán ser seleccionados a partir directamente sobre la misma o sobre la tabla de resultados con la lista de defectos y sus características.

\figuraConPosicion{0.6}{imgs/Visor.png}{Visor}{visor}{}{H}


\subsubsection*{Tabla resultados}
Aquí se muestra una lista de los defectos detectados durante el proceso de detección, así como una colección de características geométricas asociadas al defecto. Si se selecciona una fila en la tabla de resultados, el defecto se coloreará en el visor indicando con qué defecto se corresponde. De la misma forma, si se selecciona un defecto sobre el visor mediante la combinación de la tecla \textit{control} y click del ratón, se resaltará la fila de la tabla a la que corresponde.

\figuraConPosicion{1}{imgs/tablaresultados.png}{Tabla de resultados}{tablaresultados}{}{H}


\subsubsection*{Barra de menús}
Aquí se encuentran algunas opciones adicionales que pueden realizarse sobre la aplicación.

\begin{itemize}
\item \textbf{Menú Archivo:} Permite salir de la aplicación mediante la opción Salir.
\item \textbf{Menú Opciones:} Permite acceder a la configuración de opciones avanzadas de la aplicación.
\item \textbf{Menú Ayuda:} Permite acceder a la ayuda en línea de la aplicación, así como información relativa a la misma.
\end{itemize}

\figuraConPosicion{0.6}{imgs/Barramenus.png}{Barra de menús}{barramenus}{}{H}

En los siguientes apartados, iremos explicando en detalle el funcionamiento completo de la aplicación, a partir de los principales componentes ya vistos.

\subsection{Configuración de la aplicación}
Mediante el menú \textbf{Opciones}, de la barra de menú y seleccionando \textbf{Opciones Avanzadas}, podemos acceder a la configuración de las opciones avanzadas de la aplicación que nos permitirá cambiar diversos parámetros sobre la misma. Vamos a explicar qué significa cada opción.

\figuraConPosicion{0.6}{imgs/opcionesavanzadas.png}{Opciones avanzadas}{opav}{}{H}

\begin{itemize}
\item \textbf{Tamaño de la ventana:} Permite especificar el tamaño de ventana que se utilizará durante los procesos de entrenamiento y detección de defectos. Los tamaños que la aplicación permite seleccionar son: 12$\times$12, 16$\times$16, 24$\times$24 y 32$\times$32.

\item \textbf{Salto de la ventana:} Permite especificar en un porcentaje cuanto avanzará la ventana respecto a su tamaño salto a salto. Este valor puede modificarse mediante la barra de desplazamiento que puede ajustarse entre unos valores comprendidos entre el 10\% y el 100\% del tamaño de la ventana.

\item \textbf{Tipo de detección:} Especifica el tipo de detección que se llevará a cabo durante el proceso de análisis:

\begin{enumerate}
\item Normal: La detección se realiza sin tener en cuenta los falsos positivos. El resultado de la detección no se filtra por lo que se devuelven todas las posibles ventanas marcadas como defecto.
\item Normal + Umbrales locales: La detección se realiza igual que en el modo normal, pero luego se filtran aquellos píxeles marcados como defecto haciendo una intersección de los mismos con el resultado del filtro de umbrales locales. Aquellos píxeles marcados como defectuosos que también están marcados como defectuosos en el filtro de umbrales locales son los que se mantienen.
\item Blancos en umbrales locales: Primero se calculan los umbrales locales de la imagen y se saca una lista de píxeles en las regiones candidatas a albergar defectos. Durante el proceso de detección, se van considerando píxeles de cada región teniendo en cuenta el tamaño de la misma, centrando en los que sí se consideren una ventana y calculando sus características.
\end{enumerate}

\item \textbf{Ventana de entrenamiento:} Especifica el tipo de ventana utilizada durante el proceso de entrenamiento:

\begin{enumerate}
\item Deslizante: La ventana va recorriendo la imagen de forma secuencial sacando las características para crear el clasificador. El salto de la ventana viene determinado por el salto de la ventana indicado por el usuario.
\item Aleatoria: La ventana va cogiendo muestras de forma aleatoria de la imagen para extraer sus características y construir el clasificador a partir de los datos obtenidos.
\end{enumerate}


\item \textbf{Tipo de clasificación:} Especifica el tipo de clasificación que se va a establecer cuando una ventana se ha analizado:

\begin{enumerate}
\item Clases Nominales: Las ventanas se etiquetan indicando si son defectuosas o no mediante TRUE o FALSE.
\item Regresión: Las ventanas se etiquetan indicando el numero de píxeles defectuosos que se han encontrado.
\end{enumerate}

\item \textbf{Heurística ventana defectuosa:} Las ventanas se marcarán como defectuosas en función al tipo de heurística seleccionada.

\begin{enumerate}
\item Porcentaje píxeles malos en la ventana: Si el porcentaje de píxeles defectuosos detectados en la ventana es superior a un tanto por ciento del total de píxeles de la ventana, siendo este porcentaje definido por el usuario mediante la barra de desplazamiento de porcentaje de píxeles defectuosos, la ventana es marcada como defectuosa.
\item Porcentaje de vecinos defectuosos respecto al píxel central: Si el porcentaje de píxeles defectuosos detectados respecto a los vecinos del píxel central de la ventana es superior a un tanto por ciento del total de vecinos del píxel central, siendo este porcentaje definido por el usuario mediante la barra de desplazamiento de porcentaje de píxeles defectuosos, la ventana es marcada como defectuosa.
\end{enumerate}

\item \textbf{Porcentaje pixeles defectuosos:} Indica el porcentaje que se tomará como referencia en la heurística seleccionada por el usuario para determinar el umbral por el cual una ventana será clasificada como defectuosa o no.

\item \textbf{Características:} Determina que características se seleccionarán para construir el clasificador:

\begin{enumerate}
\item Todas: Se utilizan todas las características extraídas para construir el clasificador.
\item Las mejores: Se utilizan aquellas características que se consideran mejores para construir el clasificador, ya que pueden existir características que no aporten información relevante al clasificador para discriminar los datos.
\end{enumerate}

\end{itemize}

\subsection{Entrenar un clasificador}
El primer paso para poder analizar imágenes es tener un clasificador entrenado que sea capaz de decidir cuándo una ventana es defectuosa o no. En cualquier momento, se puede entrenar un clasificador para usarlo posteriormente en la aplicación para realizar el proceso de detección de defectos. Para ello hacemos click en el botón \textbf{Entrenar Clasificador} del panel de control de la aplicación.

\figuraConPosicion{0.6}{imgs/botonentrenar.png}{Botón entrenar}{botonentrenar}{}{H}

A continuación, se ofrece la posibilidad de construir un clasificador utilizando 2 métodos diferentes:

\figuraConPosicion{0.6}{imgs/tipoentrenamiento.png}{Elección del tipo de entrenamiento}{tipoentrenamiento}{}{H}

\begin{itemize}
\item Generando un archivo ARFF nuevo: Si se selecciona esta opción, la aplicación abrirá un explorador de archivos y solicitará al usuario que especifique donde se encuentra el directorio con las imágenes que se utilizarán para el proceso de entrenamiento y con las que se generará un archivo ARFF nuevo, a partir del cual se construirá el clasificador.

\textbf{NOTA:} Las imágenes que se utilicen para el proceso de entrenamiento deben tener asociadas una máscara para poder enseñar al clasificador a identificar los defectos.

\item A partir de un archivo ARFF existente: Si se selecciona esta opción, la aplicación abrirá un explorador de archivos y solicitará al usuario que especifique donde se encuentra el archivo ARFF que se utilizará para el proceso de entrenamiento y con el que se construirá el clasificador.
\end{itemize}

La aplicación guardará el modelo generado en la carpeta \textit{res/model}, en formato \textit{.model}.

Una vez construido el clasificador, la aplicación estará lista para comenzar el proceso de análisis con el fin de detectar los defectos en la imagen.

\textbf{NOTA:} Este proceso lleva un tiempo y puede demorar varios minutos entrenar un clasificador a partir de un conjunto de datos de tamaño considerable.

El proceso puede ser detenido en cualquier momento haciendo click en el botón \textbf{Stop}, es decir, el botón del aspa roja contenido en la sección \textit{Progreso} de la interfaz.

\subsection{Abrir una imagen}
Para poder analizar una imagen, es necesario primero abrirla y que la cargue la aplicación. Para ello, hacemos click en el botón \textbf{Abrir Imagen} del panel de control de la aplicación.

\figuraConPosicion{0.6}{imgs/botonabrir.png}{Botón abrir imagen}{botonentabrir}{}{H}

A continuación, la aplicación nos muestra un explorador de archivos donde seleccionaremos la imagen a cargar en la aplicación. Hay que tener en cuenta las restricciones de este explorador, ya que solo permte escoger archivos JPG, JPEG, BMP y PNG. Una vez seleccionada hacemos click en el botón \textbf{Aceptar}.

\figuraConPosicion{0.9}{imgs/escogerimagen.png}{Explorador archivos para abrir imagen}{elegirimagen}{}{H}

Una vez cargada la imagen, se mostrará en el visor de imágenes de la aplicación, y estará lista para ser procesada.

\subsection{Analizar imagen}
Una vez que tenemos la imagen abierta en el visor de la aplicación y que tenemos un clasificador debidamente entrenado a partir de un conjunto de datos de muestra, se procede al proceso de análisis para iniciar la detección de defectos sobre la imagen. Para ello, hacemos click en el botón \textbf{Analizar Imagen} del panel de control de la aplicación.

\figuraConPosicion{0.6}{imgs/botonanalizar.png}{Botón analizar imagen}{botonanalizar}{}{H}

Este botón sólo se podrá pulsar mientras se haya cargado una imagen en el visor.

XRayDetector da la opción de analizar sólo una región de la imagen. Para ello, simplemente hay que hacer click en un punto de la imagen y arrastrar el ratón, manteniendo pulsado el botón del mismo. Se generará un recuadro amarillo, que delimita la región a analizar. Si no se selecciona ninguna región en el visor de la aplicación, se mostrará un mensaje indicando que el proceso puede demorarse un tiempo en completar dicha operación.

\figuraConPosicion{0.6}{imgs/avisoanalizar.png}{Aviso al analizar imagen}{avisoanalizar}{}{H}

A continuación, la aplicación inicia el proceso de análisis en busca de defectos sobre la imagen. Durante este proceso de análisis, se verá cómo se van dibujando unas ventanas amarillas, que representan las regiones que se están analizando en ese momento. Si alguna es clasificada como defecto, se quedarán dibujadas en color verde.

\figuraConPosicion{0.6}{imgs/analizando.png}{Análisis en ejecución}{analizando}{}{H}

También se irá mostrando el progreso de la aplicación mediante una barra de progreso. Junto a esta barra se encuentra el botón de \textit{Stop} (el aspa roja), que permite detener el proceso en cualquier momento.

Una vez que ha finalizado el proceso de detección, y si no ha habido ningún error, se muestran los defectos detectados con su contorno en el visor.

\figuraConPosicion{0.6}{imgs/dibujados.png}{Defectos detectados}{dibujados}{}{H}

Los resultados obtenidos pueden observarse tanto en el visor como en la tabla de resultados que se encuentra bajo el visor. Esta tabla contiene una lista con todos los defectos encontrados, junto con sus características geométricas asociadas a cada defecto.

Si hacemos click sobre un defecto en el visor de la imagen (manteniendo la tecla \textit{control} pulsada, lo que provocará que el icono del cursor cambie a una mano), o bien sobre la lista de defectos de la tabla de resultados, se coloreará el defecto seleccionado en el visor, y se seleccionará la fila correspondiente en la tabla de resultados asociada al defecto marcado.

\figuraConPosicion{1}{imgs/resdeteccion.png}{Resultados coloreados}{resdeteccion}{}{H}

Además podemos cambiar el nivel de tolerancia de detección sobre el defecto mediante la barra desplazadora. A menor valor, mayor tolerancia y menor precisión, y viceversa.

\subsection{Analizar y exportar los resultados}
Tras el proceso de análisis de la imagen, es posible analizar y exportar los resultados obtenidos. Para ello, a través del panel \textbf{Analizar Resultados}, nos da a elegir diversas opciones:

\begin{itemize}
\item \textbf{Guardar imagen analizada:} Permite guardar una copia de lo que se muestra en el visor, en la ruta que el usuario especifique mediante un explorador de archivos, como imagen JPG.
\item \textbf{Guardar defectos binarizados:} Permite guardar el resultado de la detección con los píxeles que han sido marcados como defectuosos mediante una imagen binarizada. El resultado se guarda en la ruta que el usuario especifique mediante un explorador de archivos.
\item \textbf{Calcular Precision \& Recall:} La aplicación calcula a partir de los resultados obtenidos tras el proceso de análisis y detección de defectos los parámetros Precision and Recall, para determinar cómo de efectiva ha sido la detección. Al hacer click en el botón \textbf{Calcular Precision \& Recall}, la aplicación solicitará que se le indique donde se encuentra la máscara asociada a la imagen analizada mediante un explorador de archivos.
\end{itemize}

\figuraConPosicion{0.6}{imgs/resprecrecall.png}{Resultados de Precision \& Recall}{resprecrecall}{}{H}

\subsection{Exportar y limpiar Log}
Si se desea, puede exportarse el log de la aplicación haciendo click en el botón \textbf{Exportar} del panel Log.

\figuraConPosicion{0.5}{imgs/botonexportar.png}{Botón exportar}{botonexportar}{}{H}

Al hacer click en el mismo, la aplicación generará un archivo index.html, en la carpeta \textit{/XRayDetector/res/log/html/}.

También se puede borrar el contenido de este log mediante el botón \textbf{Limpiar Log}.

\subsection{Ayuda}
Esta aplicación dispone de un módulo de ayuda en línea, que puede ser consultado en cualquier momento. Puede accederse a la ayuda en línea de la aplicación, bien mediante la tecla de acceso rápido F1, o a través del menú Ayuda, Ayuda en Línea en la barra de menús de la aplicación.

\figuraConPosicion{1}{imgs/Ayuda.png}{Módulo de ayuda en línea}{ayuda}{}{H}

Esta ventana se compone de de 2 partes principales. En la parte izquierda nos encontramos con las pestañas de la tabla de contenidos, que se representa mediante el símbolo de un libro, y el motor de búsqueda de resultados de la ayuda, cuya representación viene definida por el símbolo de una lupa.

\figuraConPosicion{0.3}{imgs/lupa.png}{Pestañas de la ayuda en línea}{lupa}{}{H}

\subsubsection{Tabla de contenidos}
Seleccionando el icono del libro en las pestañas del módulo de ayuda, se mostrará en el lado izquierdo de la ventana, el árbol de contenidos perteneciente a la ayuda completa de la aplicación. Esta opción es la visualizada en el módulo de ayuda por defecto, y aparecerá abierta la entrada correspondiente a la ayuda  que hayamos accedido.

En la parte derecha de la ventana aparece el cuadro de ayuda con la descripción correspondiente a la entrada seleccionada en el árbol de contenidos, donde el usuario puede consultar toda la información acerca del uso de la aplicación y de los componentes que conforman la ventana desde la que solicitó la ayuda en línea.


\figuraConPosicion{1}{imgs/AyudaTablacontenidosbusqueda.png}{Ejemplo búsqueda tabla de contenidos}{AyudaTablacontenidosbusqueda}{}{H}

\subsubsection{Motor de búsqueda}
Seleccionando el icono de la lupa en las pestañas del módulo de ayuda, se mostrará en el lado izquierdo de la ventana un campo de texto que permite al usuario buscar una palabra concreta en la ayuda en línea, mostrando debajo del campo de búsqueda los sitios en los que se han encontrado las coincidencias con la palabra buscada y el número de coincidencias que se han encontrado en ese apartado del árbol de contenidos.

\figuraConPosicion{1}{imgs/AyudaMotorbusquedaejemplo.png}{Ejemplo de motor de búsqueda}{AyudaMotorbusquedaejemplo}{}{H}

En la parte derecha de la ventana aparece el cuadro de ayuda con las entradas coincidentes con la palabra buscada, resaltando los lugares donde se encuentren las coincidencias. En la imagen anterior se puede ver un ejemplo de uso del motor de búsqueda en el que se ha introducido la palabra <<análisis>>, como palabra a buscar dentro del árbol de contenidos. Como resultado se puede observar que se han encontrado 6 lugares donde se hace referencia al análisis, y dentro de los cuales, aparece 1 coincidencia en el apartado <<Ventana de Entrenamiento y Detección de Defectos>> del árbol de contenidos, y a la derecha aparecen remarcados los lugares en los que se muestra la palabra <<análisis>> encontrada en el módulo de ayuda a través del motor de búsqueda.

\subsubsection{Otras ayudas}
La ayuda en línea no es el único tipo de ayuda que se ha incluido en la aplicación. Cada componente de la interfaz (botones, barras de desplazamiento, etc) tienen asociada una ayuda emergente, que se muestra al mantener unos segundos el cursor del ratón sobre ellos. Esta ayuda proporciona una explicación muy breve sobre el componente en cuestión de forma muy rápida.


\subsection{Salir de la aplicación}
Para salir de la aplicación se debe hacer click sobre la opción \textbf{Salir} del menú \textbf{Archivo} de la barra de menús de la aplicación, o mediante el botón superior derecho de la ventana de la aplicación representado con una \textbf{X}.

%
% APÉNDICES %%%%%%%%%%%%%%%%%%%%%%%%%%%%%%%%%%%%%%%%%%%%%%%%%%%%%%%%%%%%%
\backmatter
\appendix

% Añadir entrada en el índice: Apéndices
\addcontentsline{toc}{chapter}{Apéndices}

\portadasAuxiliares{Apéndice A - Guía rápida de Version One}
%%%%%%%%%%%%%%%%%%%%%%%%%%%%%%%%%%%%%%%%%%%%%%%%%%%%%%%%%%%%%%%%%%%
%%%%%%%%%%%%%%%%%%%%%%%%%%%%%%%%%%%%%%%%%%%%%%%%%%%%%%%%%%%%%%%%%%%
\chapter{Comparativas}
%%%%%%%%%%%%%%%%%%%%%%%%%%%%%%%%%%%%%%%%%%%%%%%%%%%%%%%%%%%%%%%%%%%
%%%%%%%%%%%%%%%%%%%%%%%%%%%%%%%%%%%%%%%%%%%%%%%%%%%%%%%%%%%%%%%%%%%

\section{Introducción}
Como hemos podido ver en varias partes de la presente memoria, este proyecto es una evolución de uno comenzado anteriormente. Por ello, se hace necesario realizar una comparativa entre los dos proyectos, para tener una idea más clara de cuáles han sido las mejoras y modificaciones.

En este apéndice se explican, primero, las modificaciones realizadas respecto al proyecto anterior (\ref{modificaciones}), seguido de una comparativa de métricas de software (\ref{metricas}) y terminando con una comparativa de rendimiento entre las dos aplicaciones (\ref{rendimiento}).

\section{Modificaciones}\label{modificaciones}
Las modificaciones que se han hecho respecto al proyecto anterior han sido:

\begin{enumerate}
\item Rediseño de la estructura de la aplicación, usando una arquitectura en capas y patrones de diseño que permitan ampliar la aplicación de una forma sencilla.
\item Refactorización de las clases de cálculo de características, buscando dividir los métodos tan largos que tenían en métodos más pequeños.
\item Modificación de las clases de cálculo de características para que puedan usarse los métodos por separado (por ejemplo, si sólo quiero calcular la media ahora es posible).
\item Rediseño de la interfaz de usuario, buscando que sea más intuitiva y funcional. Ahora el usuario sólo puede realizar ciertas operaciones en un determinado momento, ya que activamos y desactivamos los botones.
\item Rediseño de las clases de la interfaz, buscando dividir el único método que había antes en varios métodos. También se han movido los \textit{listeners} de los botones a clases internas, en vez de ser clases anónimas en medio del código.
\item Implementación de multihilo. Se particionan las imágenes al analizar y al entrenar, de tal forma que cada hilo sólo considerará una partición de la imagen, lo que aumenta el rendimiento.
\item Implementación de una nueva forma de calcular los autovalores, sustituyendo la clase \textit{Matrix} que había antes por llamadas a la biblioteca \textit{EJML}, lo que hace pasar de ser inviable el cálculo de los mismos a ser sólo un poco lento.
\item Sustitución de los cálculos de la primera y segunda derivadas por llamadas a la clase \textit{Differentials\_} de ImageJ.
\item Implementación de nuevas estrategias de etiquetado de ventanas: píxel central, píxel central más región de vecinos y porcentaje del total de píxeles, de las cuales sólo nos hemos quedado con las dos últimas por ser las que mejor rendimiento daban. Con esto se pretende mejorar la precisión.
\item Implementación de nuevas estrategias de detección de defectos: detección normal más superposición con filtro de umbrales locales para descartar falsos positivos y lista de píxeles en regiones blancas en la imagen de umbrales locales. Lo que se busca con esto es, primero, aumentar la precisión y, segundo, aumentar la rapidez.
\item Implementación de un botón para guardar una binarización de los defectos encontrados, que es el paso previo para poder dibujarlos después. Esto permite tener un mayor conocimiento de cómo de bien está funcionando la detección.
\item Implementación del cálculo de características geométricas de los defectos dibujados y muestra de las mismas en una tabla en la interfaz. Esto permite conocer mejor las características de los defectos y, en un futuro, permitirá clasificar los defectos en diversos tipos.
\item Interactividad con los defectos dibujados: se permite seleccionar un defecto en el visor (mediante la combinación de la tecla \textit{control} con click del ratón), lo que resaltará la fila de la tabla de características geométricas que le corresponde. Esto también resaltará el propio defecto, dibujándolo en un tono rojizo transparente. De la misma forma, si se selecciona una fila de la tabla, se resaltará el defecto al que corresponde. Todo esto le da más valor a la interfaz.
\item Implementación del cálculo de <<Precision \& Recall>>, lo que permite conocer cómo de bien se están detectando los defectos. En un futuro, esto también servirá para comparar nuevos métodos de detección con los que ya existen de una forma numérica, más exacta.
\item Implementación de un log de excpeciones.
\item Se ha cambiado el log de la interfaz para que sea texto HTML, lo que permite usar colores. Esto aumenta la intuitividad de la aplicación, ya que el usuario enseguida ve si cierto mensaje implica que se ha realizado algo con éxito o si ha habido un error (colores verde y rojo).
\item La exportación del log de la interfaz también se hace en formato HTML, lo que permite exportar los colores.
\item Implementación de un aviso si no se selecciona región para analizar. Antes se obligaba a seleccionar una región, pero quizás el usuario en ocasiones prefiera analizar la imagen completa.
\item Opciones del programa en fichero de opciones.
\item Implementación de un módulo de ayuda en línea, lo que permite acceder directamente a la ayuda de lo que estamos usando en este momento, además de tener un buscador.
\item Mejora de la documentación de código, eliminando errores gramaticales en aquellos métodos que se han reutilizado y documentando todos los elementos del código, cosa que antes no pasaba.
\end{enumerate}


\section{Métricas}\label{metricas}
\subsection{Métricas: introducción}
Las métricas de código son una herramienta muy útil para evaluar la calidad del código de una aplicación. Como uno de los objetivos era mejorar el código del proyecto anterior, hemos visto necesario hacer una pequeña comparativa objetiva entre ambos proyectos. Para ello se necesita usar una medida objetiva: las métricas.

En concreto, las métricas que se han usado son las que permiten calcular los \textit{plugins} que ya vimos en la memoria: RefactorIT y SourceMonitor. Estas métricas son orientadas a objeto y se suelen dividir en métricas CK (Chidamber y Kemerer), métricas LK (Lorenz y Kidd) y métricas de R. Martin \cite{carlos_lopez_nozal_apuntes_2012}. Hemos hecho un resumen de las que nos han parecido más representativas.

Para consultar más información sobre estas métricas se recomienda usar la ayuda de RefactorIT o de SourceMonitor, ya que vienen muy bien explicadas.

\subsection{Métricas de RefactorIT}
En la tabla \vertabla{refactorit} hemos incluido un resumen de alguna de las métricas que nos han parecido más interesantes, seguido de una explicación de las mismas y otras que no hemos incluido en la tabla.

\tablaSmallPosicion{Métricas RefactorIT}{p{5cm}  c  c}{refactorit}{
  \multicolumn{1}{c}{Métrica} & \multicolumn{1}{c}{Nueva versión} & \multicolumn{1}{c}{Versión antigua}\\
  }
 {
 V(G) máxima & 31 & 67\\
 NP máximo & 10 & 17\\
 WMC medio & 16.4 & 27.1\\
 RFC medio & 28 & 28\\
 Dn media & 0.244 & 0.465\\
 }{H}
 
Los resultados de las métricas completas pueden consultarse en los archivos HTML incluidos en docs/métricas/RefactorIT.

Respecto a los resultados, lo primero que vemos es la diferencia en V(G), es decir, la complejidad ciclomática. Esto es muy importante, pues uno de los objetivos era mejorar el rendimiento, cosa que se ha conseguido parcialmente mejorando el código. En general, no sólo vemos una mejoría en el máximo, sino que la media parece menor, viendo la tabla completa.

En cuanto a NP (número de parámetros), también vemos que se ha disminuido notablemente el máximo. En general, se ha disminuido en toda la aplicación, gracias al uso de un fichero de opciones. Aún asi, sigue siendo un poco alto, ya no que hemos podido meter todo en este fichero.

Si observamos la fila de WMC (\textit{Weighted Methods per Class}, métodos ponderados por clase), también vemos una disminución en la media. De nuevo, vuelve a ser muy importante, ya que esta métrica representa la suma de las complejidades de los métodos de una clase. Sí que es cierto que sigue habiendo valores altos, pero en muchas ocasiones es debido al uso de los patrones de diseño ya explicados. Por ejemplo, el uso del patrón fachada hace que esta clase sea muy compleja, pero es normal. Lo mismo pasa con las superclases de las estrategias, ya que en ellas se ha implementado la lógica común a las subclases, que tiende a ser alta en nuestro caso. Esto hace que sea grande en esas clases pero que se reduzca en las otras.

Respecto a RFC (\textit{Response For a Class}, respuesta para una clase), vemos que los valores se han mantenido. Nuevamente, lo consideramos como normal por el uso de patrones (sobre todo por el patrón fachada), ya que muchos métodos responderán a llamadas de esta clase.

La Dn media (distancia a la \textit{main sequence}, una métrica que mide cómo de balanceado está un subsistema respecto a su estabilidad y abstracción) vemos que también es bastante menor en nuestro caso. Esto quiere decir que, en general, nuestros subsistemas no son demasiado abstractos para su estabilidad, ni demasiado inestables para su abstracción.

Si miramos las tablas completas, podremos ver que la métrica DIT (\textit{Depth in Tree}, profundidad en el árbol de herencia), se puede ver que en nuestro caso hay valores más altos. Esto no quiere decir que sea malo. Simplemente significa que nosotros hemos usado más herencia que los anteriores desarrolladores, generando un diseño más complejo. Esta complejidad es asumible, ya que viene dada por el uso de los patrones estrategia, lo cual es bueno, ya que van a permitir la extensibilidad de la aplicación.

Relacionada con la métrica anterior está NOC (\textit{Number of Children in Tree}, número de hijos en el árbol), que en el caso del proyecto anterior siempre es cero. Esto quiere decir que ellos no usaron nunca la herencia.

En cuanto a otras métricas, se puede ver que en nuestro proyecto las dependencias cíclicas entre paquetes (CYC y DCYC) son siempre cero, lo cual es bueno. El que no existan dependencias cíclicas entre paquetes da una idea de que el diseño no es malo, ya que las dependencias van sólo en una dirección. Esto evita problemas de integración y de extensibilidad. En el proyecto anterior no siempre son cero, por lo que se pueden dar problemas.

Respecto a LCOM (\textit{Lack of Coherence}, carencia de cohesión), los valores son bastante parecidos, si bien es cierto que ellos llegan a tener valores de uno, lo cual quizás sea demasiado. En nuestro caso, los valores altos vienen condicionados, de nuevo, por el uso de patrones. En el caso de la fachada, es la que se encarga de controlar el comportamiento general, por lo que es permisible cierto grado de acoplamiento entre métodos. En los patrones estrategia, la superclase engloba el comportamiento común, por lo que también es normal que el valor sea elevado. También es cierto que es posible que en futuras iteraciones sea recomendable dividir alguna de estas clases (sobre todo la fachada) si estos valores se disparan.

Por último, vemos que los valores de NOA (\textit{Number of Attributes}, número de variables clase) son ligeramente altos en nuestro caso. De nuevo, vuelve a ponerse de manifiesto el uso de los patrones de diseño.

\subsection{Métricas de SourceMonitor}
En primer lugar, vemos en \ver{kiviat1} un diagrama de Kiviat que muestra los valores de las métricas que ha calculado esta herramienta para nuestro proyecto.

\figuraConPosicion{1}{imgs/kiviat1.png}{Métricas SourceMonitor para nuestro proyecto}{kiviat1}{}{H}

En \ver{kiviat2}, vemos el mismo diagrama para el proyecto antiguo.

\figuraConPosicion{1}{imgs/kiviat2.png}{Métricas SourceMonitor para el proyecto antiguo}{kiviat2}{}{H}

En el caso de nuestro proyecto, vemos que casi todos los valores están dentro de los rangos que la herramienta considera como normales, lo cual en principio es bueno. Sí que es cierto que la complejidad máxima se dispara un poco, pero lo consideramos normal, ya que el proyecto tiene una complejidad inherente bastante alta. De todas formas, la complejidad media se mantiene dentro de los límites, lo cual parece indicar que es un caso aislado.

La profundidad media de bloque, que hace referencia a la profundidad de bloques anidados (sentencias \textit{if}, bucles, etc), vemos que es bastante alta, pero aún así es menor que el valor máximo del intervalo. También lo consideramos normal, ya que el código se complica en ocasiones, sobre todo con el cálculo de características.

Los métodos por clase y el número de sentencias por método también se encuentran dentro de esos valores normales, lo que indica que el código es, en principio, fácil de leer y reutilizable.

La métrica referente a los comentarios no la consideramos relevante, puesto que en la fecha en la que se sacó el informe aún no estaba terminada toda la documentación.

Respecto al proyecto del año pasado, vemos que todas las métricas menos una están fuera de los intervalos. Son muy notables las diferencias de complejidades (tanto la máxima como la media), lo que parece indicar que nuestro código debería ser más eficiente. Además, esto coincide con lo visto en las métricas de RefactorIT.

El porcentaje de comentarios parce también demasiado alto, lo que puede indicar un defecto de código, ya que demasiados comentarios pueden marcar las zonas donde se puede partir un método en otros métodos más pequeños. Relacionado con esto están las métricas de método por clase y sentencias por método. Vemos que los métodos por clase son pocos, cosa que quizás en un principio pueda parecer positivo. Realmente, en este caso, tener pocos métodos indica que estos métodos son muy largos. Esto parece confirmarse con las sentencias por método, que son demasiadas. Esto hace que el código sea muy poco reutilizable y muy difícil de leer, cosa que hemos podido comprobar a la hora de mejorar el código.

La métrica de la profundidad de bloque confirma también todo lo dicho sobre los métodos demasiado largos, ya que en muchos casos se usa una gran cantidad de sentencias condicionales y bucles dentro del mismo método, lo que dispara la profundidad del mismo. Dividiendo los métodos, cosa que hemos hecho, se reduce esta profundidad.


\section{Rendimiento}\label{rendimiento}
FALTA

\portadasAuxiliares{Apéndice B - Licencia GNU GPL}
\include{apendiceB}

%
%Bibliografía
\bibliografia{referencias}

%
% Otras referencias
\bibliografiaOtras{otrasreferencias}

\end{document}
